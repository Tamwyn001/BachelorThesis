\documentclass[../main.tex]{main.tex}
\begin{document}
\section{Altermagnetism}
The reader might already be familiar with ferromagnets and all the regular magnetic models. Taking into account the spin and
wave vector of each particle in a site, one can derive some symmetries under transform operations. For exemple a
ferromagnet is symmetric under spin-flip (\rem{means time reversal?}) and a rotation. In the exemple of the antiferromagnetism,
we have two sub-lattice with opposite spins. In such systems the spin compensate eachother resulting in a null magnetism.
The system is symmetric under spin-flip and translation.\\

Altermagnets implement two or more sub-lattice which might not be related between eachother using the crystal's symmetries.
Therfore summing up the spin doesn't result in a trivial expression like the antiferromagnetic material. In fact the overall
spin projection might almost be zero but not exactly. \rem{always integer spin, read e mails}. In a more formal way, we can
distinguish two types of altermagnet. \\

In the first type, the altermagnet's lattice site have different distance to the neighbours depending on the linking axis and the 
spin of the particle in the site.
\begin{figure}[H]
    \centering
    \begin{tikzpicture}
        \coordinate (o) at (0,0);
        \foreach \sign in {-1, 1}
        {
        \foreach \dir/\col/\title in {2/TamGreen/t+m, 0.75/red/t-m}
            {
            \pgfmathparse{ifthenelse(\dir==2, \dir*\sign, 0)} % Calculate x-coordinate
            \let\x=\pgfmathresult
            \pgfmathparse{ifthenelse(\dir==2, 0, \dir*\sign)} % Calculate y-coordinate
            \let\y=\pgfmathresult
            
            % Use the computed coordinates in a let statement
            \coordinate (a) at (\x,\y);               
            \path[-, color = \col] (o) -- node[midway]{\(\title\)} (a) ;
            \filldraw[color=black, fill = white , thick] (a) circle (0.1);
                
            }
        }
        \filldraw[color=black, fill = white , thick](0,0) circle (0.2);
        \node[anchor=center] at (o) {\(\scriptsize\uparrow\)};
    \end{tikzpicture}
    \begin{tikzpicture}
        \coordinate (o) at (0,0);
        \foreach \sign in {-1, 1}
        {
        \foreach \dir/\col/\title in {2/TamGreen/t+m, 0.75/red/t-m}
            {
            \pgfmathparse{ifthenelse(\dir==2, 0,\dir*\sign )} % Calculate x-coordinate
            \let\x=\pgfmathresult
            \pgfmathparse{ifthenelse(\dir==2, \dir*\sign ,0)} % Calculate y-coordinate
            \let\y=\pgfmathresult
            
            % Use the computed coordinates in a let statement
            \coordinate (a) at (\x,\y);               
            \path[-, color = \col] (o) -- node[midway]{\(\title\)} (a) ;
            \filldraw[color=black, fill = white , thick] (a) circle (0.1);
                
            }
        }
        \filldraw[color=black, fill = white , thick](0,0) circle (0.2);
        \node[anchor=center] at (o) {\(\scriptsize\downarrow\)};
    \end{tikzpicture}
\end{figure}
\end{document}