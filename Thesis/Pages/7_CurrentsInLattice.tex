\documentclass[../main.tex]{subfile}
\begin{document}
\subsection{Currents}
Using the charge conservation and the Heisenberg picture we are goning to derive an expression
for the current in the lattice. The system we are taking into account is two dimensional.\\

The charge conservation reads
\[
    \partial_t \rho_i = - \nabla \bm{j}_i
\]
and identifies the time variation of the charge density on the site $i$ as the negative divergence of the current density.
Now, performing some transformations, we bring this expression in a more useful form. The goal here is to integrate on both 
side over our two-dimensional surface $\Omega$. For the charge density this yields to the charge at a site:
\[
    \int_{\Omega} \partial_t \rho_i \dd \bm{r} = \partial_t Q_i .
\]
For the current density we can use the Gauß law to change the integration set:
\[
    \int_{\Omega} \nabla \bm{j}_i \dd \bm{r} = \int_{\partial \Omega} \bm{j}_i \bm{n} = \sum_{n} J_{i,n} a = \sum_{n} I_{i,n}
\]
where $\partial \Omega$ is the boundary of $\Omega$. The normal vector $\bm{n}$ points in the 2D-plane, outward from the boundary.
Asuming we have a square lattice, we can asign to each lattice site a square
unit cell with side lenght $a$. The sum over $n$ happens to be over all the side.

Now introducing the Heisenberg picture with $\hbar = 1$ we get
\[
    \partial_t Q_i = \im [H, Q_i].
\]
Finlay we can introduce the second quantisation in the charge:
\[
    Q_i = \sum_{\sigma} c_{i\sigma}^{\dagger}  c_{i\sigma} = \sum_{\sigma} n_{i,\sigma}
\]
which is quite trivial, summing over all the particle at a site leads the charge of the site. After putting all together, this yields
\begin{equation}
 I_{i}^{+x} +  I_{i}^{+y} +  I_{i}^{-x} +  I_{i}^{-y} = -\im \left[H, \sum_{\sigma} n_{i,\sigma}\right]
\end{equation}
This mean the last step to perform is to compute the commutator of the different terms of the Hamiltonian with the chrage at a site $i$.\\
We reming here that our Hamiltonian contains a chemical potentail, a hopping, a superconduting and an altermagnetic term.

\paragraph{The hopping term} $~$ We set remeber the use of a constant hopping amplitude $t_{ij} = t$.
\[
    \left[ \sum_{\langle ij\rangle\sigma} c_{i\sigma}^{\dagger}c_{j\sigma} , \sum_{\sigma} n_{l\sigma}\right] 
    = \sum_{\langle ij\rangle\sigma\sigma'}  c_{i\sigma}^{\dagger}c_{j\sigma}n_{l\sigma'} - n_{l\sigma'}c_{i\sigma}^{\dagger}c_{j\sigma}
\]
We can then introduce a useful trick that involves the commutator $[n_{\mu},c_{\nu}] = -\delta_{\mu\nu}c_{\mu}$
\begin{equation*}
    \begin{aligned}
    c_{i\sigma}^{\dagger}c_{j\sigma}n_{i,\sigma'} &= c_{i\sigma}^{\dagger} \bigl( \underbrace{c_{j\sigma}n_{l\sigma'} 
    -n_{l\sigma'}c_{j\sigma}}_{-[n_{l\sigma'},c_{j\sigma}]} + n_{l\sigma'}c_{j\sigma}\bigr) \\
    & = c_{i\sigma}^{\dagger} \left(\delta_{\sigma'\sigma}\delta_{lj}c_{i\sigma'} + n_{l\sigma'}c_{j\sigma}\right).
    \end{aligned}
\end{equation*}
Following the same schema we derive the other part of the commutator. Here the expressions involves
$[n_{\mu}, c^{\dagger}_{\nu}] = \delta_{\mu\nu}c^{\dagger}_{\mu}$:
\begin{equation*}
    \begin{aligned}
        n_{l\sigma'}c_{i\sigma}^{\dagger}c_{j\sigma} &= \bigl( \underbrace{n_{l\sigma'}c_{i\sigma}^{\dagger} 
        - c_{i\sigma}^{\dagger}n_{l\sigma'}}_{[n_{l\sigma'}, c^{\dagger}_{i\sigma}]} + c_{i\sigma}^{\dagger}n_{l\sigma'} \bigr)c_{j\sigma} \\
        &= \left(\delta_{\sigma'\sigma}\delta_{li}c_{i\sigma'}^{\dagger} + c_{i\sigma}^{\dagger}n_{l\sigma'} \right)c_{j\sigma}
    \end{aligned}
\end{equation*}
After substracting the second term from the first one we are left with
\begin{equation}
    \left[ \sum_{\langle ij\rangle\sigma} c_{i\sigma}^{\dagger}c_{j\sigma} , \sum_{\sigma} n_{i,\sigma}\right] 
    = \rem{\frac{1}{2}}\sum_{\langle ij\rangle\sigma\sigma'} \delta_{\sigma'\sigma} \left( \delta_{lj}c_{i\sigma}^{\dagger} c_{j\sigma'} - \delta_{li}c_{i\sigma'}^{\dagger}c_{j\sigma}\right).
\end{equation}
 \rem{DO we have to avoid summing twice over the nearest neigbhours?}
Because of the squared lattice we can summerise the neigbour set $\langle ij\rangle$ to $\left\{ i + \delta_x , i - \delta_x, i + \delta_y, i - \delta_y   \right\}$ involving
$\delta_{\text{axis}}$ the displacemnt from the site to neigbhour one along the given axis.
We obtain after summing up over the $\sigma'$ and writing explicitely everything we obtain 
\begin{equation*}
    \begin{aligned}
        =\frac{1}{2} \sum_{\sigma} 
       &~(c_{i\sigma}^{\dagger}c_{l-\delta_x,\sigma} -c_{i\sigma'}^{\dagger}c_{l+\delta_x,\sigma})  
       +(c_{i\sigma}^{\dagger}c_{l+\delta_x,\sigma} -c_{i\sigma'}^{\dagger}c_{l-\delta_x,\sigma})\\
       &+(c_{i\sigma}^{\dagger}c_{l-\delta_y,\sigma} -c_{i\sigma'}^{\dagger}c_{l+\delta_y,\sigma})
       +(c_{i\sigma}^{\dagger}c_{l+\delta_y,\sigma} -c_{i\sigma'}^{\dagger}c_{l-\delta_y,\sigma}).
\end{aligned}
\end{equation*}
For the chemical potential term it is useful to introduce that the commutator between two number operator vanishes. 
Since the charge and the chenical potential operators involves only number operator, we find that this term dont't take part
to the current.

\paragraph{Superconduting term} The superconduting term has a particlar behaviour. if one can sovle the gap (self consistently), this term 
doesn't contribute to the current as we are going to see. We first form the commutator between the Hubbard term and the charge operator:
\begin{equation*}
        \left[\sum_i\left(\Delta_i c_{i\uparrow}^{\dagger}c_{i\downarrow}^{\dagger} + 
        \Delta_i^{\ast} c_{i\uparrow}c_{i\downarrow}\right) , \sum_{\sigma} n_{l,\sigma}\right]
\end{equation*}
again make the already introded commutator appear, we obtain
\begin{equation*}
    \begin{aligned}
    &= \sum_{i\sigma} \delta_{il}\left(\Delta_i^{\dagger} (\delta_{\sigma\downarrow} + \delta_{\sigma\uparrow})c_{i\downarrow}c_{i\uparrow} 
    -\Delta_i(\delta_{\sigma\downarrow} + \delta_{\sigma\uparrow})c_{i\downarrow}^{\dagger} c_{i\uparrow}^{\dagger}\right)\\
    &= 2 \left( \Delta_i^{\dagger} c_{l\downarrow}c_{l\uparrow} - \Delta_i c_{l\downarrow}^{\dagger}c_{l\uparrow}^{\dagger}\right)
    \end{aligned}
\end{equation*}
This expression mixes creation and annihilation operators and makes it hard to recognise a current. One can view this term as a 
source \rem{need citation?}meaning we havw a new term $\mathcal{C}_i$ that appears in the equation  
\[
    -\partial_t Q_i = \mathcal{C}_i +  \sum_n I_{i,n}  .
\]
To know the contribution to the current we can investigate the rate of charge generation of this term. This is achieved by taking the quantum
expectation and the thermal average of the system.
\begin{align*}
    &~2\left(\Delta_l^{\dagger}\langle c_{l\downarrow}c_{l\uparrow}\rangle - \Delta_l\langle c_{l\downarrow}^{\dagger}c_{l\uparrow}^{\dagger}\rangle\right)\\
    =&~ 2 \Delta_l \sum_{n} v_{nl\downarrow}u_{nl\uparrow}^{\ast} \bigl(1-f\left(1/2 E_n\right)\bigr) + 2 \Delta_l^{\dagger} \sum_{n} u_{nl\downarrow}v_{nl\uparrow}^{\ast} f\left(1/2 E_n\right)\\
    =& ~ \frac{2}{U_i}\left(\Delta_l^{\dagger}\Delta_l -\Delta_l \Delta_l^{\dagger} \right) = 0
\end{align*}
Solving $\Delta$ self-consistently will lead to $\Delta\Delta^{\dagger} = \Delta^{\dagger}\Delta$ and therfore we find that the value vanishes.

\paragraph{Altermagnetic term}
As we introduced it in a previous discussion, the altermagnetic term is more complicated than the last one treated an need more work. In essence we describe an 
advanced hopping term, that changes regarding of the hopping axis.
We can first bring the commutator where the matrix element $\bigl(\bm{m}_{ij}\cdot\bm{\sigma}\bigr)_{\sigma\sigma'} $ is a scalar:
\begin{equation*}
    \left[\sum_{\langle i j \rangle\sigma\sigma'}\bigl(\bm{m}_{ij}\cdot\bm{\sigma}\bigr)_{\sigma\sigma'} 
        \cdot c_{i\sigma}^{\dagger}c_{j\sigma'}, \sum_{\tilde{\sigma}} n_{l,\tilde{\sigma}}\right] 
    =\sum_{\substack{\langle i j \rangle \\\sigma\sigma'\tilde{\sigma}}} 
        \bigl(\bm{m}_{ij}\cdot\bm{\sigma}\bigr)_{\sigma\sigma'}  \left(c_{i\sigma}^{\dagger}c_{j\sigma'}n_{l\tilde{\sigma}} 
        -n_{l\tilde{\sigma}}c_{i\sigma}^{\dagger}c_{j\sigma'} \right).
\end{equation*}
Using the same transformation we made earlyer to introduce the commutator between $n,c$ and $n,c^{\dagger}$ we obtain after summing over the $\tilde{\sigma}$
\[
    = \sum_{\substack{\langle i j \rangle \\\sigma\sigma'}}  \bigl(\bm{m}_{ij}\cdot\bm{\sigma}\bigr)_{\sigma\sigma'}  
        \delta_{\sigma\sigma'} \left(\delta_{lj}c_{i\sigma}^{\dagger}c_{j\sigma} - \delta_{li}c_{i\sigma}^{\dagger}c_{j\sigma}\right)
\]
introducing a new set made of  $\bm{\delta}$s as we did before. The summation over the $\bm{\delta}$ results in the following \rem{combersom} expression
\begin{equation*}
    \begin{aligned}
        = \sum_{\sigma}\Biggl[&
                \left(c_{l-\delta_x,\sigma}^{\dagger}c_{l\sigma} - c_{l\sigma}^{\dagger}c_{l+\delta_x,\sigma} \right) 
                    \sum_{\sigma'}\bigl(\bm{m}_{l,l+\delta_x}\cdot\bm{\sigma}\bigr)_{\sigma\sigma'}\\
                +&\left(c_{l+\delta_x,\sigma}^{\dagger}c_{l\sigma} - c_{l\sigma}^{\dagger}c_{l-\delta_x,\sigma} \right) 
                    \sum_{\sigma'}\bigl(\bm{m}_{l,l-\delta_x}\cdot\bm{\sigma}\bigr)_{\sigma\sigma'}\\
                +&\left(c_{l-\delta_y,\sigma}^{\dagger}c_{l\sigma} - c_{l\sigma}^{\dagger}c_{l+\delta_y,\sigma} \right) 
                    \sum_{\sigma'}\bigl(\bm{m}_{l,l+\delta_y}\cdot\bm{\sigma}\bigr)_{\sigma\sigma'}\\
                +&\left.\left(c_{l+\delta_y,\sigma}^{\dagger}c_{l\sigma} - c_{l\sigma}^{\dagger}c_{-+\delta_y,\sigma} \right) 
                    \sum_{\sigma'}\bigl(\bm{m}_{l,l-\delta_y}\cdot\bm{\sigma}\bigr)_{\sigma\sigma'}\right]
    \end{aligned}
\end{equation*} 

\paragraph{Side currents}


\paragraph{Total currents}
Until now we are able to describe how the current flows trought each faces of the unit cell. For this we can simply split the term we derived.
From this we now aim to derive the current flows on each axis.
This can ealsy be done asuming that the current flowing in the $-x$ direction subtracted from the current in the positive $x$ direction forms the total current in $\bm{e}_x$.
From this we get for $r\in\{x,y\}$:
\[
    I_{i}^{r} = I_{i}^{+r} + I_{i}^{-r}.
\]
The real current that we can measure can be obtained by taking the quantum expection value and the thermal average of the currents. Further we also introduce the 
BdG-transformed operators with the eigenvalues \ref{eq:BdG_transf_c} and \ref{eq:BdG_transf_c_dagg}. \\

For the seak of readability we are going to stick with this $r$-notation. In fact the total currents takes a disproportionate size on the page so we abstract a bit.
\begin{equation*}
    \begin{aligned}
    \langle I_l^r\rangle = \sum_{\sigma}\sum_{n\in\mathcal{N}_+} \bigl(1-f(E_n)) 
    &\left[ \sum _{\sigma'}\bigl(\bm{m}_{l,l+\delta_r}\cdot\bm{\sigma}\bigr)_{\sigma\sigma'} \left(u_{nl\sigma} u_{n,l+\delta_r,\sigma}^{\ast}  + u_{n,l-\delta_r,\sigma}u_{nl\sigma}^{\ast}\right)\right.\\
    +&\left.\sum _{\sigma'}\bigl(\bm{m}_{l,l-\delta_r}\cdot\bm{\sigma}\bigr)_{\sigma\sigma'} \left(u_{n,l+\delta_r,\sigma} u_{nl\sigma}^{\ast} + u_{nl\sigma}v_{n,l-\delta_r,\sigma}^{\ast}\right)\right]
    \end{aligned}
\end{equation*}
\end{document}