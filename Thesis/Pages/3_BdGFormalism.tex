\documentclass[../main.tex]{subfile}
\begin{document}

\section{Bogoliubov-de Gennes Formalism}
The Bogoliubov-de Gennes transformation allows us to express the hamiltonian in a diagonal way and express our quantities 
by looking at the eigenvectors of the hamiltonian. The resulting matrix is expressed in a huge space and is very sparse.\\

To give a taste of it, it will allow us to rewrite our hamiltonian as following
\begin{align}
    H = E_0 - \frac{1}{2} \check{c}^\dagger \check{H} \check{c}, \label{eq:BdG_intro_H}
\end{align}
involving $\check{c} = (\hat{c}_1,..,\hat{c}_N)$, where each $\hat{c}_i$ is a
 vector containing the creation and annihilation operators of a lattice site $i$:
$\hat{c}_i = (c_{i,\uparrow} ,c_{i\downarrow}, c_{i,\uparrow}^{\dagger} ,c_{i,\downarrow}^{\dagger})$.\\

As we see we just describe each site with the four possbile $c$-operators. This means for each lattice site,
we have a $4\times4$-submatrix that reflects the possible combinations of creation and anihinaltion operators of both spins. 
For the readability we are going to drop the comma between the site and spin indices.\\
For exemple if one has (without loss of generality) a chemical potential at the site $i$, 
 then the hamiltonian is discribed in the following way:
\[
    H_{\text{chem},i} = \sum_{\sigma} \mu_i c_{i,\sigma}^{\dagger} c_{i,\sigma}
\]
If we want to discribe it in therm of $\hat{c}_i$ we have:
\[
    H_{\text{chem},i} = \hat{c}_i^{\dagger}\cdot \mu_i \mathbb{I}_4 \cdot\hat{c}_i = \begin{pmatrix}
        c_{i,\uparrow}^{\dagger} \\c_{i\downarrow}^{\dagger}\\ c_{i\uparrow} \\c_{i\downarrow}\\
    \end{pmatrix}\cdot \mu_i
     \begin{pmatrix}
        1 & 0 & 0 & 0\\
        0 & 1 & 0 & 0\\
        0 & 0 & 1 & 0\\
        0 & 0 & 0 & 1
    \end{pmatrix}\begin{pmatrix}c_{i,\uparrow} ,c_{i\downarrow}, c_{i\uparrow}^{\dagger} ,c_{i\downarrow}^{\dagger}\end{pmatrix}
\]
Depending on the interaction we wish to describe, we can figure out what combination of operatos we want and design the $4\times4$ matrix accordingly.
To achieve a full description of the system we can consider the inderaction between to site $i,j$ as a $4\times4$ matrix involving the $\hat{c}_i^{\dagger}$ and $c_j$ operators.
Then we can build a huge matrix $\check{H}$ based on $4\times4$ matrices at $\check{H}_{i,j}$ and the vector we multiply it to is juste the $\hat{c}_i^{\dagger}$ and $c_j$ operators 
stack above one and other forming the above-introduced $\check{c}$ vector. As a result, one gets the first formula introduced in this section \ref{eq:BdG_intro_H}.
We can then compute the eigenvalues and -vectors express the quantities we're interested in. This is what we call the Bogoliubov-de Gennes transformation.\\

Now that the motivation is clear, we need to bring our Hamiltonian in a form that involves the fermionic operators $c_{i\sigma}$ and $c_{i\sigma}^{\dagger}$.
\subsection{Tigh Binding Model}
Our goal is now to fix our particle on lattice sites and describe their interactions. We are thefore going to translate our wavefunctuion fornalism in an on site plus
nearest neighbour description.\\

For the generalities, asume we have the Hamiltonian in the second quantisation formalism:
\begin{align*}
    H = & \sum_{\sigma\sigma'} \int \phi_{\sigma}^{\dagger}(\bm{r}) H_{\sigma\sigma'}(\bm{r}) \psi_{\sigma'}(\bm{r}) \dd^3r\\
    + & \sum_{\sigma\sigma'} \int \int \phi_{\sigma}^{\dagger}(\bm{r})\phi_{\sigma'}^{\dagger}(\bm{r}') V_{\sigma\sigma'}(\bm{r},\bm{r}') \phi_{\sigma'}(\bm{r}')\phi_{\sigma}(\bm{r}) \dd^3r'\dd^3r 
\end{align*}
We introcude a basis of so called Wannier orbitals $w(\bm{r} - \bm{R}_i)$ with $\bm{R}_i$ an atom location. The should be large in the neighbourhood of $\bm{R}_i$ and vanishes when the distance tends to infinity.
They are therfore called ``localised''. The basis is complete, the orbitals verify the orthonormality condition:
\[
    \int w^{\ast}(\bm{r} - \bm{R}_i) w(\bm{r} - \bm{R}_j) \dd^3r = \delta_{ij}.
\]
therfore we can define some field operator in this basis, based on creation and annihilation operators that acts on a lattice site $i$:
\begin{equation}\label{eq:Wannier_fieldOp}
    \phi_{\sigma}(\bm{r}) := \sum_{i} w(\bm{r} - \bm{R}_i) c_{i\sigma} ~~~~~~ \phi_{\sigma}^{\dagger}(\bm{r}) := \sum_{i} w^{\ast}(\bm{r} - \bm{R}_i) c_{i\sigma}^{\dagger}
\end{equation}
which is not a continuous description anymore. 
Inserting these operator back into our above Hamiltonian and using the othonornality allows us to have an on site/nearest neighbour Hamiltonian. Taking for
instance the first part of the Hamiltonian:
\begin{align*}
    H = & \sum_{\sigma\sigma'} \int \psi_{\sigma}^{\dagger}(\bm{r}) H_{\sigma\sigma'}(\bm{r}) \psi_{\sigma'}(\bm{r}) \dd^3r\\
    =& \sum_{ij\sigma\sigma'} c_{i\sigma}^{\dagger} c_{j\sigma'}\int w^{\ast}(\bm{r} - \bm{R}_i) H_{\sigma\sigma'}(\bm{r}) w(\bm{r} - \bm{R}_j) \dd^3r\\
    := & \sum_{i\sigma\sigma'} \epsilon_i^{\sigma\sigma'} c_{i\sigma}^{\dagger} c_{i\sigma'} - \sum_{\langle ij\rangle\sigma\sigma'} t_{ij}^{\sigma\sigma'} c_{i\sigma}^{\dagger} c_{j\sigma'} + ..~.   
\end{align*}
In the last line we include a local energy term $\epsilon$ and the so called hopping term $t_{ij}$, wich is the interaction with the nearest neighbour sites $j$ of $i$.
For a more precise description one could consider more neighbour. The spin depent term can be use to describe spin orbit coupeling or spin-flip processes.\\

We now aim to define the useful process for this thesis using this formalism.

\subsubsection{Non-interacting electrons}
The two main components of the non-interacting system Hamiltonian $H_N$ are the chemical potential $\mu_i$ wich is specific to each site and the hopping term $t_{ij}$.
The chemical potential is modulated by the number of particles on the site $i$ and the hopping term gives the amptidudes of moving a electron from site $i$ to $j$.
We asume it as spin-independant here.\\
\begin{equation}\label{eq:Ham_Normal}
    H_N = - \sum_{i \sigma} \mu_i c_{i\sigma}^{\dagger} c_{i\sigma} - \sum_{\langle ij\rangle \sigma} t_{ij} c_{i\sigma}^{\dagger} c_{j\sigma}
\end{equation}
where $\langle ij\rangle$ is a communly-used notation to sum over $i$ and its nearest neighbours $j$, skiping $i=j$. We label it the normal
Hamiltonian.\\

The hopping amplitude can be computed from the overlap of the orbitals under a kinetik operator $-\nabla^2/(2m)$, which explains the meaning ``hopping'':
\begin{align*}
    t_{ij} =& -\int w^{\ast}(\bm{r} - \bm{R}_i) \frac{\nabla^2}{2m} w(\bm{r} - \bm{R}_j) \dd^3r\\
           =& +\frac{1}{2m}\int \left(\nabla w(\bm{r} - \bm{R}_i)\right)^{\ast} \left(\nabla w(\bm{r} - \bm{R}_j)\right) \dd^3r.
\end{align*}
We used a partial integration considering the boundary conditions of the Wannier orbitals $w(\pm \infty) = 0$. Therfore one part of the partial integration vanishes
and we integrate/differtiate the integrands in the other integral, leading to two $\nabla$s. Further we see that $t_{ij} = t_{ji}^{\ast}$ by swaping the two integrands.\\

\subsubsection{Superconductivity}
Previous study of ours on the superconductivity have led us to the following Hamiltonian:
\[
    H_S = - \int U(\bm{r}) \psi_{\downarrow}^{\dagger}(\bm{r})\psi_{\uparrow}^{\dagger}(\bm{r})\psi_{\uparrow}(\bm{r})\psi_{\downarrow}(\bm{r}) \dd^3r
\]
on which we can apply a mean field approximation $\Delta(\bm{r}) = U(\bm{r})\langle \psi_{\uparrow}(\bm{r})\psi_{\downarrow}(\bm{r})\rangle$.
This yields to a commun BCS-Hamiltonian for regular superconductors.\\
\[
    H_S = - \int \biggl(\Delta(\bm{r}) \psi_{\downarrow}^{\dagger}(\bm{r})\psi_{\uparrow}^{\dagger}(\bm{r}) + \Delta(\bm{r})^{\ast}\psi_{\uparrow}(\bm{r})\psi_{\downarrow}(\bm{r})\biggr) \dd^3r.
\]  
we see that we the second integrand is just the complexe conjugate of the first one. 
To spare some place, we are going to focus ourselves on the first one and denoted its homologue with $h.c.$ ``hermitian conjugate''.\\
We insert $\ref{eq:Wannier_fieldOp}$ and obtain:
\begin{align*}
    H_S =& - \sum_{ij} c_{i\downarrow}^{\dagger}c_{i\uparrow}^{\dagger} \int \Delta(\bm{r}) w^{\ast}(\bm{r} - \bm{R}_i) w^{\ast}(\bm{r} - \bm{R}_j) \dd^3r + \text{h.c.}\\
    :=& -\sum_{ij} \Delta_{ij} c_{i\downarrow}^{\dagger}c_{i\uparrow}^{\dagger} + \text{h.c.}
\end{align*}
$\Delta(\bm{r})$ is an order parameter and doesn't vary to much in the coherence lenght, which is much bigger than the attomic lenght.
Thefore we can say that the orbitals varry faster than the gap. Morovere these orbitals are peakd in the neighbourhood of the atomic location $\bm{R}_i$ and $\bm{R}_j$.
Achieving the integral we get $\Delta_{ij} = \Delta_i \delta_{ij}$. We can from then reintroduce the h.c. and we get
\begin{equation}\label{eq:Ham_Supercond_Tight}
    H_S = -\sum_{i} \Delta_i c_{i\downarrow}^{\dagger}c_{i\uparrow}^{\dagger} + \Delta_i^{\ast} c_{i\uparrow}c_{i\downarrow}.
\end{equation}
We however we're missing the mean field term $E_0$:
\[
    E_0 = \int U \langle \psi_{\downarrow}^{\dagger}\psi_{\uparrow}^{\dagger}\rangle \langle\psi_{\uparrow}\psi_{\downarrow}\rangle \dd^3r. = \int U \frac{\Delta^{\ast}}{U}\frac{\Delta}{U} \dd^3r = \int \frac{|\Delta|^2}{U} \dd^3r.
\]
and after aplying the tight binding formalism we get:
\[
    E_0 = \sum_{i} \frac{|\Delta_i|^2}{U},
\]
wich is a term we can add to the Hamiltonian \ref{eq:Ham_Supercond_Tight}. Form these equations we have the final 
Hamiltonian for the superconducting system:
\[
    H = E_0 + H_N + H_S.
\]  
\subsection{A more symmertic Hamiltonian}
As we introduced it while motivating the Bogoliubov-de Gennes formalism, we aspire to describe each state as a vector-matrix-vector product of
\[
    \hat{c}_i = \left(c_{i\uparrow}, c_{i\downarrow},c_{i\uparrow}^{\dagger}, c_{i\downarrow}^{\dagger}\right).
\]
However using the form we have in the superconducting \ref{eq:Ham_Supercond_Tight} and normal \ref{eq:Ham_Normal} 
Hamiltonian will later not act as a fermionic operator upon the transformation we're about to
do. We need to rewrite the Hamiltonian in a more symmertic way to later respect the anticommutation relations.\\

\paragraph{The chemical potential} term can be expressed using the anticommutation relations of the fermionic operators $[c_{i\sigma}^{\dagger},c_{i\sigma}]_+ = 1$:
\begin{equation}\label{eq:SymHam_muTerm}
    \sum_{i\sigma} \mu_i c_{i\sigma}^{\dagger} c_{i\sigma} = \frac{1}{2}\sum_{i\sigma} \mu_i \left(c_{i\sigma}^{\dagger}c_{i\sigma} - c_{i\sigma}c_{i\sigma}^{\dagger} + 1\right)
\end{equation}
The trick we used is quite straight forward but not obvious: 
\[
    c^{\dagger}c ~=~ \frac{1}{2}c^{\dagger}c + \frac{1}{2}c^{\dagger}c ~=~ \frac{1}{2}c^{\dagger}c + \underbrace{\frac{1}{2}c^{\dagger}c + \frac{1}{2}cc^{\dagger}}_{\frac{1}{2}\anticommu{c^{\dagger}}{c} = \frac{1}{2}} - \frac{1}{2}cc^{\dagger} \tag{$\mathfrak{Tr}$1} \label{eq:Trick1}
\]
\paragraph{The hopping term} can in the same way be expressed as:
\[
    \sum_{\langle ij\rangle\sigma} t_{ij} c_{i\sigma}^{\dagger} c_{j\sigma} = \frac{1}{2}\sum_{\langle ij\rangle \sigma} t_{ij} \left(c_{i\sigma}^{\dagger}c_{j\sigma} - c_{j\sigma}c_{i\sigma}^{\dagger}\right).
\]
we can take the liberty to reorder the indicies in a term of a sum and use the fact that $t_{ij} = t_{ji}^{\ast}$:
\begin{equation}\label{eq:SymHam_tij}
    \sum_{\langle ij\rangle\sigma} t_{ij} c_{i\sigma}^{\dagger} c_{j\sigma} = \frac{1}{2}\sum_{\langle ij\rangle \sigma} t_{ij}c_{i\sigma}^{\dagger}c_{j\sigma} - t_{ji}c_{i\sigma}c_{j\sigma}^{\dagger} = \frac{1}{2}\sum_{\langle ij\rangle \sigma} t_{ij}c_{i\sigma}^{\dagger}c_{j\sigma} - t_{ji}^{\ast}c_{i\sigma}c_{j\sigma}^{\dagger}.
\end{equation}
\paragraph{The superconducting term} for its part takes the form:
\begin{equation}\label{eq:SymHam_Delta}
\sum_{i} \Delta_i c_{i\downarrow}^{\dagger}c_{i\uparrow}^{\dagger} + \Delta_i^{\ast} c_{i\uparrow}c_{i\downarrow}
        = \frac{1}{2}\sum_{i} \Delta_i \left(c_{i\downarrow}^{\dagger}c_{i\uparrow}^{\dagger}-c_{i\uparrow}^{\dagger}c_{i\downarrow}^{\dagger}\right)+ \Delta_i^{\ast} \left(c_{i\uparrow}c_{i\downarrow}-c_{i\downarrow}c_{i\uparrow}\right).
\end{equation}
We then finish this section by using Eq.\ref{eq:SymHam_muTerm}, \ref{eq:SymHam_tij} and \ref{eq:SymHam_Delta} in the Hamiltonian and obtain the following form:
\begin{align*}
    H = E_0 ~-~& \frac{1}{2} \sum_{i\sigma} \mu_i\left(c_{i\sigma}^{\dagger}c_{i\sigma} - c_{i\sigma}c_{i\sigma}^{\dagger}\right)\\
    -& \frac{1}{2}\sum_{\langle ij\rangle \sigma} t_{ij}c_{i\sigma}^{\dagger}c_{j\sigma} - t_{ji}^{\ast}c_{i\sigma}c_{j\sigma}^{\dagger}\\
    -& \frac{1}{2}\sum_{i} \Delta_i \left(c_{i\downarrow}^{\dagger}c_{i\uparrow}^{\dagger}-c_{i\uparrow}^{\dagger}c_{i\downarrow}^{\dagger}\right) +
    \Delta_i^{\ast} \left(c_{i\uparrow}c_{i\downarrow}-c_{i\downarrow}c_{i\uparrow}\right).
\end{align*}
The constant term $\frac{1}{2} \sum_{i\sigma} \mu_i$ of the normal Hamiltonian just vanished in the $E_0$. \rem{right?}
We can now rewrite the Hamiltonian in a more compact way:
\begin{align}
    H = E_0 - \frac{1}{2}\sum_{i,j} \hat{c}_i^{\dagger} \hat{H}_{ij} \hat{c}_j  \label{eq:BdG_sys_H}
\end{align}
where the on site matrix reads
\begin{align}
    \hat{H}_{ij} = \begin{pmatrix}
        \mu_i \mathbb{I}_2 \delta_{ij} + t_{ij} & -\im\sigma_2\Delta_i \delta_{ij}\\
        \im\sigma_2\Delta^{\ast}_i \delta_{ij} & -\mu_i \mathbb{I}_2 \delta_{ij} - t_{ij}^{\ast}
    \end{pmatrix} = \begin{pmatrix}
        H_{ij} & \Delta_{ij}\\
        \Delta_{ij}^{\dagger} & -H_{ij}^{\ast}
    \end{pmatrix}    \label{eq:H^_ij} 
\end{align}
where we use $\mathbb{I}_n$ as an $n$-dimensional identity matrix. We havn't explicitely removed the 
$t_{ij}$ if we're not concidering nearest neighbours.
At this point it's intersting to note that if we wish to build some periodic boundary conditions, 
it might be the case that a site on side is neighbour with a site on the other side.\\  

We can further compress our $\hat{c}_i$ operator by introducing 
\[
    \check{c} = (\hat{c}_1,..,\hat{c}_N)
\] 
along with the system Hamiltonian-matrix $\check{H}_{ij} := \hat{H}_{ij}$ wich allows us to rewrite the Hamiltonian $\ref{eq:BdG_sys_H}$ as:
\begin{align}
    H = E_0 - \frac{1}{2} \check{c}^\dagger \check{H} \check{c}.
\end{align}
\subsection{Eigenvalues}
We now have a look at the following eigenvalue problem, wich later helps from the diagonalization of the Hamiltonian:
\[
    \check{H} \check{\chi}_n = E_n \check{\chi}_n
\]
$n$ runs over the number of the eigenvalue and $\check{\chi}_n$ is the corresponding eigenvector.
we can decompose the $\check{\chi}_n$ to reflect each lattice site: $\check{\chi}_n = (\hat{\chi}_{n1},..,\hat{\chi}_{nN})$. 
This means $\chi_{n,i}$ refers to a $4\times4$ block, i.e. the on the submatrix we had earlier talked about.
Therfore this $\chi_{n,i}$ contains four values, grouped in two vectors of length two, one for each spin: $\chi_{n,i} = (u_{ni}, v_{ni})$.
Further $u_{ni} = (u_{ni\uparrow}, u_{ni\downarrow})$ couples to the two first components $(c_{i\uparrow},c_{i\downarrow})$ we had in $\hat{c}$ and 
similarly $v_{ni} = (v_{ni\uparrow}, v_{ni\downarrow})$ to the two last components $(c_{i\uparrow}^{\dagger},c_{i\downarrow}^{\dagger})$ of
the four operator $\hat{c}$.\\ 

We can simplify the eigenvalue problem by taking a look only at a site $i$. We then only sum up over $i$.th row of $\check{H}_{ij}$ with the components of $\check{\chi}_n$:
\[
    \sum_{j\in\natset{N}}\hat{H}_{ij} \hat{\chi}_{nj} = E_n \hat{\chi}_{ni}.
\]
We remember that $\check{H}_{ij}$ represent a complex scalar and $\hat{H}_{ij}$ is a $4\times4$ matrix with complexe entries. So it follows by reintroducing \ref{eq:H^_ij} the following set of equations:
\begin{equation}
    \left\{
    \begin{aligned}
        &\sum_{j\in\natset{N}}H_{ij} u_{nj} + \Delta v_{nj} = &E_n u_{nj}\\
        &\sum_{j\in\natset{N}}\Delta^{\dagger} u_{nj} - H_{ij}^{\ast} v_{nj} =~& E_n v_{nj}.
    \end{aligned}
    \right.~~
    \stackrel{(1)}{\longrightarrow}~~
    \left\{
        \begin{aligned}
            &\sum_{j} H_{ij} u_{nj} + \Delta v_{nj} ~=& E_n u_{nj}\\
            &\sum_{j} H_{ij} v_{nj}^{\ast} + \Delta^{\dagger} u_{nj}^{\ast}= &-E_n v_{nj}^{\ast}.
        \end{aligned}
        \right.
\end{equation}
Where in (1) we took the conjugate of the second equation and used $\Delta^{\dagger} = -\Delta^{\ast}$.
This is an important result, beacause it shows that if $\check{\chi}_n = (u_{n1}, v_{n1},u_{n2}, v_{n2},.. )$ 
is an eigenvector with eigenvalue $E_n$, then  
so should be $(v_{n1}^{\ast}, u_{n1}^{\ast},v_{n2}^{\ast}, u_{n2}^{\ast},.. )$ with the eigenvalue $-E_n$.\\

This leads to a symmetry in the energy spectrum of $H = E_0 \pm \frac{1}{2}\check{c}^{\dagger}\check{H}\check{c}^{\dagger}$.
This flexibility allows us to choose the version of $H$ with the positive sign, which is more communly used.\\

\subsection{Diagonalization}
Our goal is now to express the Hamiltonian relative to its energy eigenvalues, which is more practicle to work with.
As we have seen in the last section, eigenvectors $\chi_n$ allows us to compute the energies. Therfore we are going 
to diagonalize the Hamiltonian by using the eigenvectors $\chi_n$ to express the Hamiltonian according to its eigenvalues.\\

First we define a row-vector of our eigenstate $\check{X} = [\check{\chi}_{\pm 1},..,\check{\chi}_{\pm 2N}]$ and introduce 
a diagonal matrix $\check{D} = \text{diag}(E_{\pm 1},..,E_{\pm 2N})$ with the eigenvalues.
Then we can write the Hamiltonian as:
\[
    \check{H} = \check{X}\check{D}\check{X}^{-1} = \check{X}\check{D}\check{X}^{\dagger} 
\]
we can then transform the Hamiltonian with $\check{c} := \check{X}\check{\gamma}$
\begin{align*}
    H = E_0 - \frac{1}{2}\check{c}^{\dagger}\check{H}\check{c} =& E_0 - \frac{1}{2}\check{\gamma}^{\dagger}\check{X}^{\dagger}\check{H}\check{X}\check{\gamma}\\
    =& E_0 - \frac{1}{2}\check{\gamma}^{\dagger}\underbrace{\check{X}^{\dagger} \check{X}}_{=\mathbb{I}}\check{D}\underbrace{\check{X}^{-1}\check{X}}_{=\mathbb{I}}\check{\gamma}\\
    =& E_0 - \frac{1}{2}\check{\gamma}^{\dagger}\check{D}\check{\gamma}\\
    =& E_0 - \frac{1}{2}\sum_{n\in\mathcal{N}}\\
\end{align*}
where $\mathcal{N}= \{\pm n : n \in \natset{N}\}$
Reagranging the transformation of $\check{c}$ we get $\gamma = \check{X}^{\dagger}\check{c}$   
Now that we've made the structure of the involved variables clear in the last section, we find the expression of the $\gamma$ wich is $2N$-dimensional:
\begin{align*}
    \gamma_n &= \sum_{i} \left( u_{ni\uparrow}^{\ast}c_{i\uparrow}+ v_{ni\uparrow}^{\ast}c_{i\uparrow}^{\dagger} +  u_{ni\downarrow}^{\ast}c_{i\downarrow}+ v_{ni\downarrow}^{\ast}c_{i\downarrow}^{\dagger}\right)\\
        &=  \sum_{i\sigma} \left( u_{ni\sigma}^{\ast}c_{i\sigma}+ v_{ni\sigma}^{\ast}c_{i\sigma}^{\dagger}\right)
\end{align*}
and due to the symmerty we saw erlier, 
\[
    \gamma_{-n} =  \sum_{i\sigma} \left( v_{ni\sigma}c_{i\sigma}+ v_{ni\sigma}c_{i\sigma}^{\dagger}\right)
\]
for $n\in \natset{N}$. We now take a look at the conjugate transpose of $\gamma_{-n}$. Because scalar are dimension $1\times1$ we have 
$(uc^{\dagger})^{\dagger} = (c^{\dagger})^{\dagger} u^{\dagger} =  c^{\dagger} u^{\ast} = u^{\ast}c$ and it follows:
\[
    \gamma_{-n}^{\dagger} = \sum_{i\sigma} \left( v_{ni\sigma}^{\ast}c_{i\sigma}^{\dagger}+ u_{ni\sigma}^{\ast}c_{i\sigma}\right) = \gamma_n.
\]
Using this we can link each $\gamma_i$ to the corresponding eigenvalue $E_i$:  $\gamma_n$ to the corresponding eigenvalue $E_n$ and $\gamma_{n-}$ to the corresponding eigenvalue $E_{-n} = -E_n$.
We recall that we had $2N$ degrees of freedom $c_{i\sigma}$ due
to the spins and after the transformation we get $4N$ degrees into $\hat{c}_i$. But because our energies $E_n$ and $E_{-n}$ are realted to eachother, we can keep the positive $2N$ eigenvalues
and this maintain the total number of degree of freedom.\\

We can split the sum over the $n\in\mathcal{N}$ in two parts: $\mathcal{N}_+ = \{n \in \mathcal{N} : n> 0 \}$, $\mathcal{N}_- = \{n \in \mathcal{N} : n<0 \}$
\begin{align*}
    H &= E_0 + \frac{1}{2}\sum_{n\in\mathcal{N}_+} E_n \gamma^{\dagger}_n\gamma_n + \frac{1}{2}\sum_{n\in\mathcal{N}_-} E_n \gamma^{\dagger}_n\gamma_n\\
      &= E_0 + \frac{1}{2}\sum_{n\in\mathcal{N}_+} E_n \gamma^{\dagger}_n\gamma_n + \frac{1}{2}\sum_{n\in\mathcal{N}_+} E_{-n} \gamma^{\dagger}_{-n}\gamma_{-n}\\
      &= E_0 + \frac{1}{2}\sum_{n\in\mathcal{N}_+} E_n \gamma^{\dagger}_n\gamma_n - \frac{1}{2}\sum_{n\in\mathcal{N}_+} E_n \gamma^{\dagger}_{-n}\gamma_{-n}\\
      &= E_0 + \frac{1}{2}\sum_{n\in\mathcal{N}_+} E_n \gamma^{\dagger}_n\gamma_n - \frac{1}{2}\sum_{n\in\mathcal{N}_+} E_n \gamma_{n}\gamma^{\dagger}_{n}\\
      &= E_0 + \frac{1}{2}\sum_{n\in\mathcal{N}_+} E_n \left( \gamma^{\dagger}_n\gamma_n -\gamma_{n}\gamma^{\dagger}_{n}\right)
\end{align*}
where with used the energy symmetry and $\gamma^{\dagger}_{-n} =\gamma_{n}, \gamma_{-n} =\gamma^{\dagger}_{n}$.\\

Using this knowledge, we can express a final formula for the Hamiltonian by using the anticommutation properties of the fermionic $\gamma$-operators:
 $\anticommu{\gamma^{\dagger}_n}{\gamma_{n}} = 1$, so using the trick \ref{eq:Trick1} and bringing the $\frac{1}{2}$ prefactor in the sum:
\begin{align}
    H =& E_0 - \sum_{n\in \natset{N}} E_n \left( \gamma_n^{\dagger}\gamma_n - \frac{1}{2}\right).
\end{align}
This is the final form of the Hamiltonian in the Bogoliubov-de Gennes formalism. As a user one should build the Hamiltonian 
and computes its eigenvalues,-vector and transform them into the $\gamma$ operators.\\

\subsubsection{Superconducting Gap}
We already covered how the superconducting gap $\Delta$ is a relevant property of the Meissner state. We now aim to use the  mean field 
theorie in order to find the gap. This requires a self consistency equation, which we can be derived from the Hamiltonian.\\ 

The gap was defined as $\Delta(\bm{r}) := U(\bm{r}) \langle \psi_{\uparrow}(\bm{r})\psi_{\downarrow}(\bm{r})\rangle$. Back to the tight binding formalism,
the gap now depends on the lattice site $i$ and reads
$\Delta_i = \langle c_{i\uparrow}c_{i\downarrow}\rangle$ and we can express $c_{i\sigma}$ in terms of the $\gamma$-operators:\\
\begin{center}
\begin{minipage}{0.4\textwidth}
\begin{equation}
    \begin{aligned}\label{eq:BdG_transf_c}
    c_{i\sigma} =& \sum_{n\in\mathcal{N}} u_{ni\sigma}\gamma_n \\
        =& \sum_{n\in\mathcal{N}_+} u_{ni\sigma}\gamma_n + u_{-n,i\sigma}\gamma_{-n}\\
        =& \sum_{n\in\mathcal{N}_+} u_{ni\sigma}\gamma_n + v_{ni\sigma}^{\ast}\gamma_{n}^{\dagger}
    \end{aligned}
\end{equation}
\end{minipage}\hspace{0.05\textwidth}
\begin{minipage}{0.03\textwidth}
    \begin{tikzpicture}
        \coordinate (a) at (0,1);
        \coordinate (b) at (0,-1);
        \draw[-] (b) -- (a);
        \filldraw[color=black, fill = white , thin] (a) circle (0.05);
        \filldraw[color=black, fill = white , thin] (b) circle (0.05);
    \end{tikzpicture}
\end{minipage}
\begin{minipage}{0.45\textwidth}
    \begin{equation}
        \begin{aligned}\label{eq:BdG_transf_c_dagg}
        c_{i\sigma}^{\dagger} =& \sum_{n\in\mathcal{N}_+}( u_{ni\sigma}\gamma_n)^{\dagger} +( v_{ni\sigma}^{\ast}\gamma_{n}^{\dagger})^{\dagger} \\
            =& \sum_{n\in\mathcal{N}_+} \gamma_n^{\dagger}u_{ni\sigma}^{\dagger} + \gamma_{n}(v_{ni\sigma}^{\ast})^{\dagger} \\
            =& \sum_{n\in\mathcal{N}_+} u_{ni\sigma}^{\ast}\gamma_n^{\dagger} + v_{ni\sigma}\gamma_{n} \\
        \end{aligned}
    \end{equation}
\end{minipage}
\end{center}

where we used the symmetry of the eigenvectors. We can now compute expectation value involved in the gap:
\begin{align*}
    \langle c_{i\uparrow}c_{i\downarrow} \rangle =& \sum_{n,m\in\mathcal{N}_+} \left\langle\left(u_{ni\uparrow}\gamma_n + v_{ni\uparrow}^{\ast}\gamma_{n}^{\dagger}\right)\left(u_{mi\downarrow}\gamma_m + v_{mi\downarrow}^{\ast}\gamma_{m}^{\dagger}\right)\right\rangle\\
     =&  \sum_{n,m\in\mathcal{N}_+} \left\langle\left(u_{ni\uparrow} u_{mi\downarrow}\gamma_n\gamma_m + u_{ni\uparrow} v_{mi\downarrow}^{\ast}\gamma_n\gamma_{m}^{\dagger} + v_{ni\uparrow}^{\ast}u_{mi\downarrow}\gamma_{n}^{\dagger} \gamma_m+  v_{ni\uparrow}^{\ast} v_{mi\downarrow}^{\ast}\gamma_{n}^{\dagger}\gamma_{m}^{\dagger}  \right)\right\rangle\\
     \stackrel{(\ast)}{=}&  \sum_{n\in\mathcal{N}_+} \left\langle u_{ni\uparrow} v_{ni\downarrow}^{\ast}\gamma_n\gamma_{n}^{\dagger}\right\rangle + \left\langle v_{ni\uparrow}^{\ast}u_{ni\downarrow}\gamma_{n}^{\dagger}\gamma_{n}\right\rangle\\     
     =&  \sum_{n\in\mathcal{N}_+} u_{ni\uparrow} v_{ni\downarrow}^{\ast}\left\langle\gamma_n\gamma_{n}^{\dagger}\right\rangle + v_{ni\uparrow}^{\ast}u_{ni\downarrow}\left\langle \gamma_{n}^{\dagger}\gamma_{n}\right\rangle\\
     =&  \sum_{n\in\mathcal{N}_+} u_{ni\uparrow} v_{ni\downarrow}^{\ast} \left(1-f(E_n)\right) + v_{ni\uparrow}^{\ast}u_{ni\downarrow}f(E_n)
\end{align*}
where $f$ is the Fermi-Dirac distribution. In $(\ast)$ we notice no $\gamma\gamma$ or $\gamma^{\dagger}\gamma^{\dagger}$ terms in the Hamiltonian, so their expectation 
value is zero
\footnote{This is like the expectation value of killing twice a fermion in a state. It is not possbile, because we cant annihilate a state that has a possession number of zero.
And in the same way due to the Pauli-principle we can't have more than one particle in the same state, so $\langle \gamma^{\dagger} \gamma^{\dagger} \rangle = 0$. Here we 
additionaly removed the indicies, in fact the diagonalty also takes place on the indicies so that we end just with $n$.
The Hamiltonian is diagonal in $\gamma\gamma^{\dagger}$ and $\gamma^{\dagger}\gamma$ \rem{right?}}.

The expectation value $\langle a\hat{A}\rangle_{\Phi}$ of a scalar times an operator reads $\langle \Phi|a \hat{A}|\Phi\rangle_{\Phi} = a \langle \Phi|\hat{A}|\Phi\rangle_{\Phi} = a \langle \hat{A}\rangle_{\Phi}$. 
To convince onselves, we just take a look at the first quantisation expression of this braket. This result leads to the self consistency equation:
\begin{equation}\label{eq:SelfConsitentDelta}
    \Delta_i = U_i\sum_{n\in\mathcal{N}_+} u_{ni\uparrow} v_{ni\downarrow}^{\ast} \left(1-f(E_n)\right) + u_{ni\downarrow} v_{ni\uparrow}^{\ast}f(E_n)
\end{equation}
We plan to solve this equation numerically, inserting some guess in the Hamiltonian, diagonalize it, update $\Delta_i$ and reinsert it into H and repeat until we reach a fixpoint.\\
\end{document}

