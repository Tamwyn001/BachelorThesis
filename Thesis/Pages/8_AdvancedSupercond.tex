\documentclass[../main.tex]{subfile}
\begin{document}
\section{An advanced superconductivity}
In this section we are going to highlight how the neighbour-depending potential can lead to a new kind of superconductivity.
These new superconductive states are label $s$, $p$ and extended $d$-wave superconductivity.

To begin this new study we first need a desciption of the Hamiltonian in the reciproquial space. 
To achieve this we use a descrete Fourier transformation \rem{write the explicit coeficient as well as the transformation}.

\begin{equation*}
    H = \sum_{\bm{k},\sigma} \epsilon_{\bm{k}} c_{\bm{k}\sigma}^{\dagger} c_{\bm{k}\sigma} 
    - V \sum_{\bm{k}} \left( F_{\bm{k}} c^{\dagger}_{\bm{k}\uparrow} c^{\dagger}_{-\bm{k}\downarrow}
    - F_{\bm{k}}^{\ast} c_{-\bm{k}\downarrow} c_{\bm{k}\uparrow} \right)
\end{equation*}
Involving 
\begin{equation*}
    \epsilon_{\bm{k}} = -2t \left( \cos(k_x) + \cos(k_y) \right) - \mu\\
    F_{\bm{k}} = - V \left(F^{x+} e^{\im k_x} + F^{x-}e^{-\im k_x} + F^{y+} e^{\im k_y} + F^{y-}e^{-\im k_y} \right). 
\end{equation*}
having $F{ij} = \langle c_{i\uparrow} c_{j\downarrow}$ and $F^{r\pm} = F_{i, \pm r}$. We do as well see that the correlation functions is the same
for each lattice site $i$.\\

In a similar way as before, we aim to use the BdG-formalism to solve the Hamiltonian thanks to the eigenvalues and -vector of the system.
To achieve this earlier discussions outligned the need to bring the Hamiltonian in a matrix shape.\\

--------------------------------------------\\

The description of the creatation and annihilation operators with a step in the momentum space for the $y$-direction can be expressed as follow:
\begin{align}
    c_{xy\sigma} = \frac{1}{\sqrt{N_y}} \sum_{k_y} c_{x k_y\sigma} e^{\im k_y y} \label{eq:C_Foruier}\\
    c^{\dagger}_{xy\sigma} = \frac{1}{\sqrt{N_y}} \sum_{k_y} c^{\dagger}_{x k_y\sigma} e^{-\im k_y y}\label{eq:C_dagg_Foruier}
\end{align}
For the readebility we are going to use $k_y \rightarrow k$. Further to an index $i$ with can associate $(x,y)$. An index $j$ can be associated with $(x',y')$.
We first want to transform the hopping term:
\begin{equation*}
    H_{\text{hop}} = -\sum_{\langle ij\rangle \sigma} t_{ij} \sum_{kk'} c^{\dagger}_{x k\sigma} c_{x' k'\sigma} e^{\im (k' y' - k y)}
\end{equation*}
Here we can use the neighbour shift properties $y' = y + \delta_y$ where $\delta_y = \pm 1$. Doing so we have 
\[
    e^{\im (k' y' - k y)} = e^{\im (k' (y+\delta_y) -  k y)} = e^{\im (k' -  k)y}e^{\im k'\delta_y}
\]
No wee need to express the neighbourhood sum. Precily we have
\begin{align*}
        H_{\text{hop}} =& -\sum_{\langle ij\rangle \sigma} t_{ij} \sum_{kk'} c^{\dagger}_{x k\sigma} c_{x' k'\sigma}  e^{\im (k -  k')y}e^{\im k'\delta_y}\\
                        =& -\sum_{xy \sigma} \sum_{kk'}  \biggl( t_{x,x+1} \underbrace{c_{x k\sigma}^{\dagger} c_{x+1 k'\sigma}e^{\im k'\cdot (0)}}_{\text{$+x$ hopping, no $\delta_y$}} +  t_{x,x-1} \underbrace{c_{x k\sigma}^{\dagger} c_{x-1 k'\sigma} e^{\im k'\cdot (0)}}_{\text{$-x$ hopping, no $\delta_y$}}\\
                        & ~~~~~~~~~~~~+t_{x,y-1} \underbrace{c_{x k\sigma} c_{x k'\sigma}^{\dagger} e^{\im k'\cdot (-1)}}_{\delta_y = -1} + t_{x,y+1} \underbrace{c_{x k\sigma}^{\dagger} c_{x k'\sigma}e^{\im k'\cdot (1)}}_{\delta_y = 1}\biggr)e^{\im (k -  k')y}
\end{align*}
As we see the $y$ direction is now expressed in the $k$-index, which is unique for each lattice $y$-slice. The information is then conserved. 
We know that system have different material on the $x$ axis. This means $t_{x,y-1} = t_{x,x} = t_{x,y+1}$ because the material are isotropic but every material has a 
different hopping term. Beside we can use the following realtion $1/N_y \sum_{y} e^{\im (k -  k')y} = \delta_{kk'}$. Perfoming both expression leads after a summation over $k'$ to
\begin{equation*}
    H_{\text{hop}} = -\sum_{x k \sigma} t_{x,x+1} c_{x k\sigma}^{\dagger} c_{x+1 k\sigma} + t_{x,x-1} c_{x k\sigma}^{\dagger} c_{x+1 k\sigma} +  t_{x,x} c_{x k\sigma}^{\dagger} c_{xk\sigma}\left(e^{\im k} + e^{-\im k}\right)
\end{equation*}
And now we can reintroduce an arbitray second coordinate $x'$ to describe the neighbours.
\begin{equation}
    H_{\text{hop}} = -\sum_{xx' k \sigma} t_{x,x'} c_{x k\sigma}^{\dagger} c_{x' k\sigma} \left(\delta_{x+1,x'} + \delta_{x-1,x'} + \delta_{x,x'} 2 \cos(k)\right)
\end{equation}
The chemical potential term is more easly given. In fact the number opertor yields to use two operators $c^{\dagger}c$ at a same coordinate $i$.
\begin{equation}
    H_{\mu} = -\mu \sum_{x kk' \sigma} c_{x k\sigma}^{\dagger} c_{x k'\sigma} e^{\im (k-k')y} = \sum_{xx'k\sigma} \mu c_{xk\sigma}^{\dagger} c_{xk'\sigma} \delta_{xx'}     
\end{equation}
We finaly have for the terms involving $c^{\dagger}_{xx'\sigma} c_{xx'\sigma}$:
\begin{equation*}
    H_{\text{hop}} + H_{\mu} = \sum_{xx'k\sigma} \epsilon_{xx'k\sigma} c_{xk\sigma}^{\dagger} c_{x'k\sigma}
\end{equation*}
using
\[
    \epsilon_{xx'k\sigma}= - t_{xx'} \left(\delta_{x+1, x'} + \delta_{x-1,x'}\right) - \left(t_{xx'}2\cos(k) + \mu\right) \delta_{xx'}
\]

Moving on to the potential term we have to introduce a new notation. $i\pm\hat{x} = i\pm 1$  $i\pm\hat{y} = i\pm N_x$. \rem{Check if $-$ or $+$.}.
For the brevety we use $f(a) = V_{i,a}F_{i,a} c^{\dagger}_{a\downarrow} c^{\dagger}_{i\uparrow}$
\begin{align*}
    H_V =& V \sum_{\langle ij\rangle} F_{ij} c^{\dagger}_{i\downarrow} c^{\dagger}_{j\uparrow} + \text{h.c.} \\
    =& \sum_{i} f(i-1) + f(i+1) + f(i+N_x) + f(i-N_x) + \text{h.c.}
\end{align*}
We can now insert our fourier transformation introduced in Eq.\ref{eq:C_dagg_Foruier} and Eq.\ref{eq:C_Foruier} to obtain
\begin{align*}
    H_V =& \sum_{xy} \frac{1}{N_y} \sum_{kk'} \left( V_{x,x+1} F_{x}^{x+} c^{\dagger}_{x+1,k,\downarrow}c^{\dagger}_{x,k',\uparrow} + V_{x,x-1} F_{x}^{x-} c^{\dagger}_{x+1,k,\downarrow} c^{\dagger}_{x,k',\uparrow} \right.\\
        &+ \left.V_{x,x}\left( F_{x}^{y+} e^{-\im k}  - F_{x}^{y-} e^{\im k}\right) c^{\dagger}_{x,k,\downarrow} c^{\dagger}_{x,k',\uparrow} \right) e^{-\im(k + k')y} + \text{h.c.}\\
\end{align*}
Defining 
\begin{equation*}
    \begin{aligned}
    F_{xx'k} =& - V_{x,x+1} F_{x}^{x+} c^{\dagger}_{x+1,k,\downarrow}c^{\dagger}_{x,k',\uparrow} + V_{x,x-1} F_{x}^{x-} c^{\dagger}_{x+1,k,\downarrow} c^{\dagger}_{x,k',\uparrow} \\
        &+ V_{x,x}\left( F_{x}^{y+} e^{-\im k}  - F_{x}^{y-q} e^{\im k}\right)
    \end{aligned}
\end{equation*}
we obtain a BdG-transformed Hamiltonian
\begin{align*}
    H =& \sum_{xx'k} D_{xk}^{\dagger} H_{xx'k} D_{x'k}\\
      =& \sum_{xx'k} \begin{pmatrix}
        c^{\dagger}_{xk\uparrow} & c_{x,-k,\downarrow}
      \end{pmatrix}
        \begin{pmatrix}
            \epsilon_{xx'k\uparrow} & F_{xx'k}\\
            -F_{xx'k}^{\ast} & -\epsilon_{xx'k\downarrow}
        \end{pmatrix}
        \begin{pmatrix}
            c_{x'k\uparrow}\\
            c^{\dagger}_{x',-k,\downarrow}
        \end{pmatrix}
\end{align*}
The summation over all $x,x'$ can be represented in a new matrix.
\[
    H = \sum_k D_k^{\dagger} H_k D_k
\]
involving the $4Nx \times 4Nx$ matrix $H_k$ and the $4Nx$-dimensional vector $D_k$.
\[
    H_k = \begin{pmatrix}
        H_{11k} &\dots & H_{1N_xk}\\
        \vdots&\ddots&\\
        H_{N_x1k} & & H_{N_xN_xk}
    \end{pmatrix}
\]
As before the $y$ information is stored in the $k$-index, which is unique for each lattice $y$-slice. This said, we 
can diagonalize $N_y$ times a $4N_x \times 4N_x$ matrix. $H_k$ represent the interaction of a $y$-line with itself.
The eigenvalues are the same for each $y$-slices and physical
quantities are going to be expressed with this summation over $k$ and the eigenvalues, -vectors of each $k$ ($y$-slice).\\

On the other hand the vector we use to carry the creation and annihilation operators is given as 
\[
    D_k^{\dagger} = \begin{pmatrix}
        c^{\dagger}_{1k\uparrow} & c_{1,-k,\downarrow}& \dots &c^{\dagger}_{N_x k\uparrow} & c_{N_x,-k,\downarrow}
    \end{pmatrix}
\]
The eigenvalues equation is similar to Eq.\ref{eq:BdG_eigenVal_H}
\begin{equation}\label{eq:BdG_eigenVal_H_k}
    H_k \mathfrak{X}_{nk} = E_{nk} \mathfrak{X}_{nk}
\end{equation}
The eigenvectors and -values are given as
\begin{equation*}
    \mathfrak{X}_{nk} = \begin{pmatrix}
        \mathfrak{x}_{n1k}\\
        \vdots\\
        \mathfrak{x}_{nN_xk}\\
    \end{pmatrix},~~ \mathfrak{x}_{nN_xk} = \begin{pmatrix}
        u_{nik}\\
        v_{nik}
    \end{pmatrix}
\end{equation*}
if we stick to the formalism we already derived in the earlier Sec.\label{sec:Diagonalization} we obtain similar 
eigenvectors where $u_{nik}$ corresponds to $c$ 
\end{document}