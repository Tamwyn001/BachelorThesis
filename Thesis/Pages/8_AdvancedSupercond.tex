\documentclass[../main.tex]{subfile}
\begin{document}
\section{An advanced superconductivity}
\subsection{Site depentend potential}
In this section we are going to highlight how the neighbour-depending potential can lead to a new kind of superconductivity.
These new superconductive states are label $s$, $p$ and extended $d$-wave superconductivity. The derivations we are going to make are closely based on
the work of A.H. Mjøs and J. Linder in \cite{Mjos2019}.\\

The Hamiltonian beeing the groundstone of this 
disscusssion, we're going to beggin with it. It differs slightly from the symmertic one we derived in Eq.\ref{eq:Ham_Symm_Supercond_1}. This extended version contains a neighbour-depending potential $H_V$ that is going to 
produce the new superconductive $\Delta$-part of the Hamiltonian. This potential is as well attractive such that $V>0$, similar to the BCS theorie.
\begin{equation}
    H = -t\sum_{\langle ij\rangle \sigma} c_{i\sigma}^{\dagger}c_{j\sigma} - \mu \sum_{i\sigma} c_{i\sigma}^{\dagger}c_{i\sigma} - \frac{V}{2} \sum_{\langle ij\rangle \sigma} n_{i\sigma}n_{j\bar{\sigma}}.
\end{equation}
Here the $\bar{\sigma}$-notation means the opposite spin of $\sigma$. In other words the attraction finds only place between particles of opposite spin, mirroring the formation process of Cooper-pairs.
We have a factor of one half to avoid counting twice the neighbours.\\

Trained eyes will recognise that $H_V$ is not quadratic in the creation and annihilation operators which makes the Hamiltonian impossible to write in the BdG-formalism. For this reason we can 
use the so called Hartree-Fock mean field approximation $H_V \rightarrow H^{HF}_V$ defined as:
\begin{equation}
    H^{HF}_V = -\frac{1}{2} \sum_{\langle ij\rangle \sigma} V_{ij} \left(F_{ij}^{\sigma \bar{\sigma}} c^{\dagger}_{j\bar{\sigma}}c^{\dagger}_{i\sigma} + \text{h.c.} + |F_{ij}^{\sigma \bar{\sigma}}|^2\right)
\end{equation}
\rem{Propagate the minus in the equations.}
involving the pairing amplitude $F_{ij}^{\sigma \bar{\sigma}} = \langle c_{i\sigma}c_{j\bar{\sigma}}\rangle$ that we already introduced. The $V_{ij}$ are the neighbour-depending potential.
The last term is a constant energy term which we can discard during the diagonalization process. If one wanted to compute the free energy this must be included there. From this point simplifications
can be made to reach out final Hamiltonian.\\

Using the fermionic property $[c_{i\sigma}^{\dagger},c_{j\sigma}^{\dagger}]_+ = 0$ we have $c_{i\sigma}^{\dagger} c_{j\sigma}^{\dagger} = -c_{i\sigma}^{\dagger} c_{j\sigma}^{\dagger}$
which leads to $\langle  c_{i\sigma}^{\dagger} c_{j\sigma}^{\dagger} \rangle = -\langle c_{i\sigma}^{\dagger} c_{j\sigma}^{\dagger}\rangle$. We can use this in the last step of the following simplification:
\begin{equation}\label{eq:HF_V}
    \begin{aligned}
        H^{HF}_V =& \frac{1}{2} \sum_{\langle ij\rangle \sigma} V_{ij} \left(F_{ij}^{\sigma \bar{\sigma}} c^{\dagger}_{j\bar{\sigma}}c^{\dagger}_{i\sigma} + \text{h.c.}\right)\\
            =& \frac{1}{2}\sum_{\langle ij\rangle} V_{ij} \left(F_{ij}^{\uparrow \downarrow} c^{\dagger}_{j\downarrow}c^{\dagger}_{i\uparrow} + F_{ij}^{\downarrow\uparrow} c^{\dagger}_{j\uparrow}c^{\dagger}_{i\downarrow} + \text{h.c.}\right)\\
            =& \sum_{\langle ij\rangle} V_{ij} \left(F_{ij}^{\uparrow \downarrow} c^{\dagger}_{j\downarrow}c^{\dagger}_{i\uparrow} + \text{h.c.}\right)
    \end{aligned}
\end{equation}
Using the raltion $F_{ij}^{\uparrow\downarrow} = -F_{ij}^{\downarrow\uparrow}$ and the symmerty of the potential $V_{ij} = V_{ji}$ we can add the two terms up and remove the one half factor.
The Hamiltonian is now ready to be diagonalized but first, we are going to discuss which advantages a Fourier transform of the Hamiltonian could bring us.\\
 
In an homogenous material we can consider a lattice and imagine some periodic boundary conditions in all directions. There is a translation invariance.
In heterostructures however the material may vary, let's asume
without loss of generality, in a direction. As we now, a periodic signal is a good candiade for a Fourier transform, which we can then express in a finite set of coeficient. This is very handfull.
However the direction we want to transform on has to show a periodicty. For this reason an homogenous two-dimensional lattice involves a Fourier transformation in two space dimensions while a heterostructure
we transform only in the direction where the material is the same. For instance a multilayer material in the $x$-direction can be described by combining a real space desciption in the $x$-axis 
combined to a Fourier transformation in the $y$-direction.
In this thesis we are going to focus ourselves on a multilayer material in the $x$-direction.\\

\subsection{In the vertical periodic boundary setup}
The description of the creation and annihilation operators with a step in the momentum space for the $y$-direction can be expressed as follow:
\begin{align}
    c_{xy\sigma} =& \frac{1}{\sqrt{N_y}} \sum_{k_y} c_{x k_y\sigma} e^{\im k_y y} \label{eq:C_Foruier}\\
    c^{\dagger}_{xy\sigma} =& \frac{1}{\sqrt{N_y}} \sum_{k_y} c^{\dagger}_{x k_y\sigma} e^{-\im k_y y}\label{eq:C_dagg_Foruier}
\end{align}
How do we find an expression for $k_y$? Well the periodicity (for ex. in the $c$ operator) yields 
$c_{xy\sigma} = c_{x,y+N_x,\sigma}$ this means the following condition must be fullfilled:
\begin{align*}
    c_{xy\sigma} = c_{x,y+N_x,\sigma} \Leftrightarrow &e^{\im k_y y} = e^{\im k_y (y+N_y)} \Leftrightarrow e^{\im k_y N_y} = 1\\
    \Leftrightarrow &k_y = \frac{2\pi n}{N_y}.
\end{align*}

We know that the momentum index should cover the entire first Brilloin zone. This covers the momentum from $-\pi/a$ to $\pi/a$ 
where $a$ is the lattice constant. Further due to the $2\pi/a$ periodicity we have the same $k_y$ at $-\pi/a$ and $\pi/a$.
For this reason we need to map $k_y $ in $[-\pi/a ;\pi/a)$.
This means we have $n\in \left[-N_y/2  ; N_y/2-1\right]$ including $0$. For $n=4$ we then have $k_y\in\{ -\pi, -\pi/2, 0,  \pi/2\}$. \\

For the readebility we are going to use $k_y \rightarrow k$.
Further to an index $i$  can associate $(x,y)$ and to $j$, $(x',y')$.
We first want to transform the hopping term:
\begin{equation*}
    H_{\text{hop}} = -\sum_{\langle ij\rangle \sigma} t_{ij} \sum_{kk'} c^{\dagger}_{x k\sigma} c_{x' k'\sigma} e^{\im (k' y' - k y)}
\end{equation*}
Here we can use the neighbour shift properties $y' = y + \delta_y$ where $\delta_y = \pm 1$. Doing so we have 
\[
    e^{\im (k' y' - k y)} = e^{\im (k' (y+\delta_y) -  k y)} = e^{\im (k' -  k)y}e^{\im k'\delta_y}
\]
No wee need to express the neighbourhood sum. Precily we have
\begin{align*}
        H_{\text{hop}} =& -\sum_{\langle ij\rangle \sigma} t_{ij} \sum_{kk'} c^{\dagger}_{x k\sigma} c_{x' k'\sigma}  e^{\im (k -  k')y}e^{\im k'\delta_y}\\
                        =& -\sum_{xy \sigma} \sum_{kk'}  \biggl( t_{x,x+1} \underbrace{c_{x k\sigma}^{\dagger} c_{x+1 k'\sigma}e^{\im k'\cdot (0)}}_{\text{$+x$ hopping, no $\delta_y$}} +  t_{x,x-1} \underbrace{c_{x k\sigma}^{\dagger} c_{x-1 k'\sigma} e^{\im k'\cdot (0)}}_{\text{$-x$ hopping, no $\delta_y$}}\\
                        & ~~~~~~~~~~~~+t_{x,y-1} \underbrace{c_{x k\sigma} c_{x k'\sigma}^{\dagger} e^{\im k'\cdot (-1)}}_{\delta_y = -1} + t_{x,y+1} \underbrace{c_{x k\sigma}^{\dagger} c_{x k'\sigma}e^{\im k'\cdot (1)}}_{\delta_y = 1}\biggr)e^{\im (k -  k')y}
\end{align*}
As we see the $y$ direction is now expressed in the $k$-index, which is unique for each lattice $y$-slice. The information is then conserved. 
We know that system have different material on the $x$ axis. This means $t_{x,y-1} = t_{x,x} = t_{x,y+1}$ because the material are isotropic but every material has a 
different hopping term. Beside we can use the following realtion $1/N_y \sum_{y} e^{\im (k -  k')y} = \delta_{kk'}$. Perfoming both expression leads after a summation over $k'$ to
\begin{equation*}
    H_{\text{hop}} = -\sum_{x k \sigma} t_{x,x+1} c_{x k\sigma}^{\dagger} c_{x+1 k\sigma} + t_{x,x-1} c_{x k\sigma}^{\dagger} c_{x+1 k\sigma} +  t_{x,x} c_{x k\sigma}^{\dagger} c_{xk\sigma}\left(e^{\im k} + e^{-\im k}\right)
\end{equation*}
And now we can reintroduce an arbitray second coordinate $x'$ to describe the neighbours.
\begin{equation}
    H_{\text{hop}} = -\sum_{xx' k \sigma} t_{x,x'} c_{x k\sigma}^{\dagger} c_{x' k\sigma} \left(\delta_{x+1,x'} + \delta_{x-1,x'} + \delta_{x,x'} 2 \cos(k)\right)
\end{equation}
The chemical potential term is more easly given. In fact the number opertor yields to use two operators $c^{\dagger}c$ at a same coordinate $i$.
\begin{equation}
    H_{\mu} = -\mu \sum_{x kk' \sigma} c_{x k\sigma}^{\dagger} c_{x k'\sigma} e^{\im (k-k')y} = -   \sum_{xx'k\sigma} \mu c_{xk\sigma}^{\dagger} c_{xk'\sigma} \delta_{xx'}     
\end{equation}
We finaly have for the terms involving $c^{\dagger}_{xx'\sigma} c_{xx'\sigma}$:
\begin{equation*}
    H_{\text{hop}} + H_{\mu} = \sum_{xx'k\sigma} \epsilon_{xx'k\sigma} c_{xk\sigma}^{\dagger} c_{x'k\sigma}
\end{equation*}
using
\[
    \epsilon_{xx'k\sigma}= - t_{xx'} \left(\delta_{x+1, x'} + \delta_{x-1,x'}\right) - \left(t_{xx'}2\cos(k) + \mu\right) \delta_{xx'}
\]

Moving on to the potential term we have to introduce a new notation. $i\pm\hat{x} = i\pm 1$ and $i\pm\hat{y} = i\pm N_x$. \rem{Write a single clear message at the beggining of the lattice introduction.}.
For the brevety we use $f(a) = V_{ia}F_{ia} c^{\dagger}_{a\downarrow} c^{\dagger}_{i\uparrow}$
\begin{align*}
    H_V =& \sum_{\langle ij\rangle} V_{ij} F_{ij} c^{\dagger}_{i\downarrow} c^{\dagger}_{j\uparrow} + \text{h.c.} \\
    =& \sum_{i} f(i-1) + f(i+1) + f(i+N_x) + f(i-N_x) + \text{h.c.}
\end{align*}
We can now insert our Fourier transformation introduced in Eq.\ref{eq:C_dagg_Foruier} and Eq.\ref{eq:C_Foruier} to obtain
\begin{align*}
    H_V =& \sum_{xy} \frac{1}{N_y} \sum_{kk'} \left( V_{x,x+1} F_{x}^{x+} c^{\dagger}_{x+1,k,\downarrow}c^{\dagger}_{xk'\uparrow} + V_{x,x-1} F_{x}^{x-} c^{\dagger}_{x+1,k,\downarrow} c^{\dagger}_{xk'\uparrow} \right.\\
        &+ \left.V_{xx}\left( F_{x}^{y+} e^{-\im k}  - F_{x}^{y-} e^{\im k}\right) c^{\dagger}_{xk\downarrow} c^{\dagger}_{xk'\uparrow} \right) e^{-\im(k + k')y} + \text{h.c.}\\
\end{align*}
Defining a more general form for the summand involving two site $i$ and $j$:
\begin{equation*}
    \begin{aligned}
    F_{xx'k} =& - V_{x,x'}\left( F_{x}^{x+} \delta_{x+1,x'} +  F_{x}^{x-} \delta_{x-1,x'}\right. \\
        &+ \left.\left( F_{x}^{y+} e^{-\im k}  - F_{x}^{y-} e^{\im k}\right)\delta_{xx'}\right),
    \end{aligned}
\end{equation*} 
where we got $k' = -k$ from the sum over the $y$-direction. We can rewrite the expression of $H_V$ as
\begin{align*}
    H_V =& \sum_{x} \sum_{kk'} F_{xx'k} c^{\dagger}_{xk\uparrow} c^{\dagger}_{x',-k,\downarrow} + F_{xx'k}^{\ast} c_{x',-k,\downarrow} c_{xk\uparrow}.
\end{align*}
Now using $H_{\text{hop}}, H_{\mu}$ and $H_{V}$ we can write the full Hamiltonian as
\begin{equation}
    \begin{aligned}
    H =& \sum_{xx'k} D_{xk}^{\dagger} H_{xx'k} D_{x'k}\\
      =& \sum_{xx'k} \begin{pmatrix}
        c^{\dagger}_{xk\uparrow} & c_{x,-k,\downarrow}
      \end{pmatrix}
        \begin{pmatrix}
            \epsilon_{xx'k\uparrow} & F_{xx'k}\\
            F_{xx'k}^{\ast} & -\epsilon_{xx'k\downarrow}
        \end{pmatrix}
        \begin{pmatrix}
            c_{x'k\uparrow}\\
            c^{\dagger}_{x',-k,\downarrow}
        \end{pmatrix}.
    \end{aligned}
\end{equation}
The summation over all $x,x'$ can be represented in a new matrix.
\[
    H = \sum_k D_k^{\dagger} H_k D_k
\]
involving the $4Nx \times 4Nx$ matrix $H_k$ and the $4Nx$-dimensional vector $D_k$.
\[
    H_k = \begin{pmatrix}
        H_{11k} &\dots & H_{1N_xk}\\
        \vdots&\ddots&\\
        H_{N_x1k} & & H_{N_xN_xk}
    \end{pmatrix}
\]
As before the $y$ information is stored in the $k$-index, which is unique for each lattice $y$-slice. This said, we 
can diagonalize $N_y$ times a $4N_x \times 4N_x$ matrix. $H_k$ represent the interaction of a $y$-line with itself.
The eigenvalues are the same for each $y$-slices and physical
quantities are going to be expressed with this summation over $k$ and the eigenvalues, -vectors of each $k$ ($y$-slice).\\

On the other hand the vector we use to carry the creation and annihilation operators is given as 
\[
    D_k^{\dagger} = \begin{pmatrix}
        c^{\dagger}_{1k\uparrow} & c_{1,-k,\downarrow}& \dots &c^{\dagger}_{N_x k\uparrow} & c_{N_x,-k,\downarrow}
    \end{pmatrix}\in\mathbb{H}^{2N_x}
\]

\subsubsection{BdG-transformation for a vertical foruier transform}
The eigenvalues equation is similar to Eq.\ref{eq:BdG_eigenVal_H}
\begin{equation}\label{eq:BdG_eigenVal_H_k}
    H_k \mathfrak{X}_{nk} = E_{nk} \mathfrak{X}_{nk}
\end{equation}
The eigenvectors and -values are given as
\begin{equation*}
    \mathfrak{X}_{nk} = \begin{pmatrix}
        \mathfrak{x}_{n1k}\\
        \vdots\\
        \mathfrak{x}_{nN_xk}\\
    \end{pmatrix},~~ \mathfrak{x}_{nxk} = \begin{pmatrix}
        u_{nxk}\\
        v_{nxk}
    \end{pmatrix}
\end{equation*}
if we stick to the formalism we already derived in the earlier Sec.\ref{sec:Diagonalization} we obtain similar 
eigenvectors where $u_{nik}$ corresponds to $c$. We are now going to transform the $c$ operators. First we need
to define $\mathfrak{X}_k = [\mathfrak{X}_{1k}, ..,\mathfrak{X}_{2N_xk}]\in\mathbb{H}^{2N_x\times 2N_x}$ storing the number of 
lattice sites times the number of spins (2) in the first dimension and 
the number of eigenvectors in the second dimension. Beside we have $\mathfrak{g}_k = (\mathfrak{g}_{1k}, .., \mathfrak{g}_{2N_xk})^T\in\mathbb{H}^{2N_x}$ along with $D_{k} = \mathfrak{X}_k\mathfrak{g}_k$,
delivering $\mathfrak{g}_k = \mathfrak{X}^{\dagger}_k D_{k}$ which is equivalent to:
\begin{equation}
    \mathfrak{g}_{nk} = \sum_{x\in{N_x}} u_{nxk}c_{xk\uparrow} + v_{nxk}c^{\dagger}_{x,-k,\downarrow}.
\end{equation}
Looking at $D_{k} = \mathfrak{X}_k\mathfrak{g}_k$ we can derive two very usefull expressions:
\begin{center}
    \begin{minipage}{0.4\textwidth}
    \begin{equation}\label{eq:BdG_transf_c_fourier}
        c_{xk\uparrow} = \sum_{n\in\natset{2N_x}} u_{nxk}\mathfrak{g}_{nk} 
    \end{equation}
    \end{minipage}\hspace{0.05\textwidth}
    \begin{minipage}{0.03\textwidth}
        \begin{tikzpicture}
            \coordinate (a) at (0,0.5);
            \coordinate (b) at (0,-0.5);
            \draw[-] (b) -- (a);
            \filldraw[color=black, fill = white , thin] (a) circle (0.05);
            \filldraw[color=black, fill = white , thin] (b) circle (0.05);
        \end{tikzpicture}
    \end{minipage}
    \begin{minipage}{0.45\textwidth}
        \begin{equation}
        \label{eq:BdG_transf_c_fourier_dagg}
            c_{x,-k,\downarrow}^{\dagger} =\sum_{n\in\natset{2N_x}} v_{nxk}\mathfrak{g}_{nk}. 
        \end{equation}
    \end{minipage}
    \end{center}

\subsubsection{Pairing amplitudes}
For the derivation of the pairing amplitudes on the axis $a\in\{\hat{\bm{x}},\hat{\bm{y}}\}$.
Doing so, we will use our BdG-transformation as we did earlier to be 
able to solve these parameters self-consistently using the eigenvectors and -values.\\

\begin{equation*}
    F_i^{\pm a} = \langle c_{i\uparrow} c_{i\pm a,\downarrow}\rangle =
    \frac{1}{N_y} \sum_{kk'} \langle c_{i_x k\uparrow} c_{i_x\pm a,k',\downarrow} \rangle e^{\im(k + k')i_y} e^{\pm\im k' \delta_y}
\end{equation*}
using $y\pm a = y'$. Here for the brevety we have to sacrifice the notation. This is unformal and might be confusing so we clarify the point in first place.\\
So $a$ describes a displacement in the $x$- or $y$-direction. On one hand $i+a$ refers to 
the lattice index when moving from $i$ with a translation $a$. On the other hand, having ``$i = (x,y)$'', $x+a$ refers to the $x$-coordinate of the lattice site $i+a$.
This is analogue for $y$.\\ 
This then means or $a= \hat{\bm{y}}$ we have $x\pm a = x$ and $k'_a = 0$.\\

The expectaion value is independant of the site vertical coordinate $y$. This means that we can achieve a site description by making the average of the value over a $x$-slice. In other words
we can write:
\begin{equation*}
    F^{\pm a}_i = 
    \frac{1}{N_y}\sum_{y\in\natset{N_y}} \frac{1}{N_y} \sum_{kk'} \langle c_{i_x k\uparrow} c_{i_x \pm a,k',\downarrow} \rangle e^{\im(k + k')\bm{r}_i} e^{\pm\im k'\delta_y}
\end{equation*}
which after the sumation over $i$ results as we covered earlier as $1/N_y \sum_{y} e^{\im(k + k')i_y} = \delta_{k,-k'}$
\begin{equation*}
    F^{\pm x}_i = 
    \frac{1}{N_y} \sum_{k} \langle c_{i_x k\uparrow} c_{i_x\pm a,k,\downarrow} \rangle e^{\mp\im k_a}
\end{equation*}
Now that we have simplifed the Fourier transform we can incorporate the BdG-transformation. The process is very similar to Eq.\ref{eq:transfo_c_up_c_down_BdG} and yields
\begin{equation}\label{eq:transfo_F_x}
    F^{\pm a}_i =\frac{1}{N_y}\sum_{k}\sum_{nn'} u_{ni_x k}v_{n',i_x \pm a,k}^{\ast} \langle \mathfrak{g}_{nk} \mathfrak{g}_{n'k}^{\dagger} \rangle e^{\mp\im k \delta_y}
\end{equation}
where we can as well write $i_x = x$. Recalling the use of $a$ to abstractise the axis we get:
\begin{align}
    F^{\pm x}_i =& \frac{1}{N_y} \sum_{nk} u_{ni_x k} v_{n,i_x\pm x,k}^{\ast}\left(1 - f(E_{nk})\right)\\
    F^{\pm y}_i =& \frac{1}{N_y} \sum_{nk} u_{ni_x k} v_{n i_x k}^{\ast}\left(1 - f(E_{nk})\right) e^{\mp \im k}.
\end{align}



\subsubsection{In the open boundary setup}
Here we need to reuse the equations of the Nambu-Spin formalism desribed in chapter \ref{sec:BdGFormalism}. As we saw we need to first
take care that the part of the Hamiltonian we want to introduce has \rem{a kind of symmerty} in the creation and annihilation operators.
If we transform the Hartree-Fock Hamiltonian as it is in Eq.\ref{eq:HF_V} we see that only the uper right quadrant of the matrix $H_{V,ij}^{HF}$ 
of $\check{c}_i^{\dagger} \cdot (H_{V}^{HF})_{ij}\cdot \check{c}_j $ is filled. We need to use the anticommutation relations to solve this.
This delivers the following expression:
\begin{equation}
    \begin{aligned}
    H_V^{HF} =& -\sum_{\langle ij\rangle} V_{ij} \left(F_{ij}^{\uparrow \downarrow} c_{j\downarrow}^{\dagger}c_{i\uparrow}^{\dagger} +
     F_{ij}^{\uparrow \downarrow\ast} c_{i\uparrow}^{\dagger}c_{j\downarrow}^{\dagger}\right)\\
     =& -\sum_{\langle ij\rangle} V_{ij} \left(F_{ji}^{\uparrow \downarrow} c_{i\downarrow}^{\dagger}c_{j\uparrow}^{\dagger} +
     F_{ij}^{\uparrow \downarrow\ast} c_{i\uparrow}^{\dagger}c_{j\downarrow}^{\dagger}\right)\\
     =& -\frac{1}{2}\sum_{\langle ij\rangle} V_{ij} \left[F_{ij}^{\uparrow \downarrow} \left(c_{i\downarrow}^{\dagger}c_{j\uparrow}^{\dagger} - c_{i\uparrow}^{\dagger}c_{j\downarrow}^{\dagger}\right)
     + F_{ij}^{\uparrow \downarrow\ast}  \left(c_{i\uparrow}^{\dagger}c_{j\downarrow}^{\dagger} - c_{i\downarrow}^{\dagger}c_{j\uparrow}^{\dagger}\right)\right]\\
    \end{aligned}
\end{equation} 
which is achieved using Eq.\ref{eq:Trick1}, $V_{ij} = {V_{ji}}$, \rem{$ F_{ij}^{\uparrow \downarrow} = -F_{ji}^{\downarrow \uparrow} = F_{ji}^{\uparrow \downarrow}$}
using the anticommutation realtion and a spin echange. We then use $F_{ij}^{\uparrow \downarrow}$ = $F_{ij}$ since all ambiguity has been removed.\\
Therefore the matrix we obtain is given as
\begin{equation}
    H_{V,ij}^{HF} = -\frac{1}{2}\begin{pmatrix}
        \mathcal{O} & & & -F_{ij}^{\uparrow \downarrow}\\
         & & F_{ij}^{\uparrow \downarrow} & \\
         & F_{ij}^{\uparrow \downarrow\ast}& & \\
         -F_{ij}^{\uparrow \downarrow\ast} & & & \mathcal{O}\\
    \end{pmatrix} = -\frac{1}{2}\begin{pmatrix}
        \mathcal{O} & -\im \sigma_2 F_{ij}\\
        \im \sigma_2 F_{ij}^{\ast} & \mathcal{O}
    \end{pmatrix}
\end{equation}
an takes only place between two neighbouring sites $i$ and $j$. For the accuracy one could multiply it with 
$(\delta_{i+1,j} + \delta_{i-1,j} + \delta_{i+N_x,j} + \delta_{i-N_x,j})$. This matrix can then be added to Eq.\ref{eq:H^_ij} replacing the on-site superconductive $\Delta$-term by
a neighbouring superconductive $F$-term.\\

Now that we have expressed the Hartree-Fock Hamiltonian in the correct maner, we can use our BdG-transformation to diagonalize the Hamiltonian and find a self-consistent solution 
for the $F$-parameters. Here we use the transformation between the creation and annihilation operators, and the eigenvectors, -values of the Hamiltonian given by Eq.\ref{eq:BdG_transf_c} 
and Eq.\ref{eq:BdG_transf_c_dagg}. 
\begin{equation}
    F_{ij} = \langle c_{i\uparrow}c_{j\downarrow}\rangle = \sum_{n\in \mathcal{N}_+} u_{ni\uparrow}v_{nj\downarrow}^{\ast}\left(1 - f(E_{nk})\right) + v_{ni\uparrow}^{\ast}u_{nj\downarrow}f(E_{nk})
\end{equation}
in a very similar way as Eq.\ref{eq:transfo_c_up_c_down_BdG}.


\subsection{Advanced order parameters}
In the more simple desciption of the superconductivity we outlied how the superconducting order parameter $\Delta$ depends on the pairing amplitude $F$.
Because the potential was isotropic, we simply had $\Delta = U \langle c_{i\uparrow}c_{i\downarrow}\rangle$. Here however the potential is anisotropic 
and we obtain a linear combination of the pairing amplitudes in the different directions of the lattice. Achieving different combinations of the $F$s 
we obtiain different superconducting states. Here is an exaustive list. \rem{repalce V with U for consistency?}
\begin{alignat}{3}
    &\Delta_{s,i} = V F_{s,i} &= \frac{V}{4} \left(F_{i}^{x+(S)} + F_{i}^{x-(S)} + F_{i}^{y+(S)} + F_{i}^{y-(S)}\right)\\
    &\Delta_{d,i} = V F_{d,i} &= \frac{V}{4} \left(F_{i}^{x+(S)} + F_{i}^{x-(S)} - F_{i}^{y+(S)} - F_{i}^{y-(S)}\right)\\
    &\Delta_{p_x,i} = V F_{p_x,i} &= \frac{V}{2} \left(F_{i}^{x+(T)} - F_{i}^{x-(T)}\right)\\
    &\Delta_{p_y,i} = V F_{p_y,i} &= \frac{V}{2} \left(F_{i}^{y+(T)} - F_{i}^{y-(T)}\right)
\end{alignat}
$F_s$, $F_d$, $F_{p_x}$ and $F_{p_y}$ are the pairing amplitudes for the $s$, $d$ (also called $d_{x^2-y^2}$ because of its expression), $p_x$ and $p_y$-wave superconductivity.
The $S$ and $T$ are the singlet and triplet expressions of the pairing amplitudes. More precily we define them as
\begin{align}
    F_{ij}^{(S)} = \frac{F_{ij} \rem{-} F_{ji}}{2} = \rem{F_{ij}}\\
    F_{ij}^{(T)} = \frac{F_{ij} \rem{+} F_{ji}}{2}
\end{align}
Where we shorten the expression using $F_{ij}^{\uparrow\downarrow(S)} = F_{ij}^{(S)}$.
\paragraph{Symmetry discussion} $~$ We see that these parameter depends on the spin and the momentum. Therfore it's a good idea to look at their respective
behaviour under exchange of these variables.\\

The discussion for the spin echange is quite straight forward.
\begin{align*}
    F_{ij}^{\uparrow\downarrow(S)} =& \frac{F_{ij}^{\uparrow\downarrow} + F_{ji}^{\uparrow\downarrow}}{2} = \frac{\langle c_{i\uparrow}c_{j\downarrow}\rangle + \langle c_{j\uparrow}c_{i\downarrow}\rangle}{2}\\
    F_{ij}^{\uparrow\downarrow(T)} =& \frac{F_{ij}^{\uparrow\downarrow} - F_{ji}^{\uparrow\downarrow}}{2} = \frac{\langle c_{i\uparrow}c_{j\downarrow}\rangle - \langle c_{j\uparrow}c_{i\downarrow}\rangle}{2}
\end{align*}
\rem{What's happening when we invert both spin?} using $\langle c_{i\uparrow}c_{j\downarrow}\rangle = -\langle c_{i\downarrow}c_{j\uparrow}\rangle$. Using the linearity of the expression we obtain
\begin{align*}
    F_{ij}^{\uparrow\downarrow(S)} = - F_{ij}^{\downarrow\uparrow(S)}\\
    F_{ij}^{\uparrow\downarrow(T)} = F_{ij}^{\downarrow\uparrow(T)}.
\end{align*}
This means that the singlet wave-pairing amplitude is antisymmetric under spin exchange and the triplet wave pairing is symmertic under spin excahnge. Their names
find place in the analogy of the wavefunction formalism.\\

For the momentum exchange we re going to take a look at the Fourier transformation of the pairing amplitudes. Using the transformation we made to 
reach Eq.\ref{eq:transfo_F_x} with any translation $ x \rightarrow \bm{r}$, $r\in \{\bm{e}_x,\bm{e}_y\}$ we obtain
\begin{align*}
    F_{i,i+ \bm{r}} = \frac{1}{N} \sum_{\bm{k}} \langle c_{\bm{k}\uparrow} c_{-\bm{k},\downarrow}\rangle e^{-\im\bm{k}\bm{r}}.
\end{align*}
Changing the sign of the momentum we obtain  $\frac{1}{N} \sum_{\bm{k}} \langle c_{-\bm{k},\uparrow} c_{\bm{k},\downarrow}\rangle e^{\im\bm{k}\bm{r}} =  F_{i,i- \bm{r}}$
because of the $\delta_{\bm{k},\bm{k}'}$ trick. For this reason $F_{i,i+ \bm{r}} +  F_{i,i- \bm{r}}$ is symmertic under momentum exchange while $F_{i,i+ \bm{r}} -  F_{i,i- \bm{r}}$ is antisymmetric.\\
Refering back to the order parameter definiton we see that the $s$ and $d$-wave are symmertic under momentum exchange, where the $p_x$ and $p_y$-wave are antisymmetric in such exchange.\\

\subsection{Implementation of the advanced superconductivity}

\rem{The individual F always change but the end result in $F_d$ can converge so we compute $F_{d,i}$ on each step to see if it converges.}
\end{document}  