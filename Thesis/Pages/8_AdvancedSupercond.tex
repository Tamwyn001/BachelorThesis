\documentclass[../main.tex]{subfile}
\begin{document}
\section{An advanced superconductivity}
\subsection{Site depentend potential}
In this section we are going to highlight how the neighbour-depending potential can lead to a new kind of superconductivity.
These new superconductive states are label $s$, $p$ and extended $d$-wave superconductivity. The derivations we are going to make are closely based on
the work of A.H. Mjøs and J. Linder in \cite{Mjos2019}.\\

The Hamiltonian beeing the groundstone of this 
disscusssion, we're going to beggin with it. It differs slightly from the symmertic one we derived in Eq.\ref{eq:Ham_Symm_Supercond_1}. This extended version contains a neighbour-depending potential $H_V$ that is going to 
produce the new superconductive $\Delta$-part of the Hamiltonian. This potential is as well attractive such that $V>0$, similar to the BCS theorie.
\begin{equation}
    H = -t\sum_{\langle ij\rangle \sigma} c_{i\sigma}^{\dagger}c_{j\sigma} - \mu \sum_{i\sigma} c_{i\sigma}^{\dagger}c_{i\sigma} - \frac{V}{2} \sum_{\langle ij\rangle \sigma} n_{i\sigma}n_{j\bar{\sigma}}.
\end{equation}
Here the $\bar{\sigma}$-notation means the opposite spin of $\sigma$. This means that the attraction founds only place between particle of opposite spin, mirroring the formation of Cooper-pairs.
We have a factor of one half to avoid counting twice the neighbours.\\

Trained eyes will recognise that $H_V$ is not quadratic in the creation and annihilation operators which makes the fermionic properties hard to prove \rem{is this correct?}. For this reason we can 
use the so called Hartree-Fock mean field approximation $H_V \rightarrow H^{HF}_V$ defined as:
\begin{equation}
    H^{HF}_V = \frac{1}{2} \sum_{\langle ij\rangle \sigma} V_{ij} \left(F_{ij}^{\sigma \bar{\sigma}} c^{\dagger}_{j\bar{\sigma}}c^{\dagger}_{i\sigma} + \text{h.c.} + |F_{ij}^{\sigma \bar{\sigma}}|^2\right)
\end{equation}
involving the pairing amplitude $F_{ij}^{\sigma \bar{\sigma}} = \langle c_{i\sigma}c_{j\bar{\sigma}}\rangle$ that we already introduced. The $V_{ij}$ are the neighbour-depending potential.
The last term is a constant energy term which we can discard during the diagonalization process. If one wanted to compute the free energy this must be included there. From this point simplifications
can be made to reach out final Hamiltonian.\\
\begin{equation}
    \begin{aligned}
        H^{HF}_V =& \frac{1}{2} \sum_{\langle ij\rangle \sigma} V_{ij} \left(F_{ij}^{\sigma \bar{\sigma}} c^{\dagger}_{j\bar{\sigma}}c^{\dagger}_{i\sigma} + \text{h.c.}\right)\\
            =& \frac{1}{2}\sum_{\langle ij\rangle} V_{ij} \left(F_{ij}^{\uparrow \downarrow} c^{\dagger}_{j\downarrow}c^{\dagger}_{i\uparrow} + F_{ij}^{\downarrow\uparrow} c^{\dagger}_{j\uparrow}c^{\dagger}_{i\downarrow} + \text{h.c.}\right)\\
            =& \sum_{\langle ij\rangle} V_{ij} \left(F_{ij}^{\uparrow \downarrow} c^{\dagger}_{j\downarrow}c^{\dagger}_{i\uparrow} + \text{h.c.}\right)
    \end{aligned}
\end{equation}
Using the raltion $F_{ij}^{\uparrow\downarrow} = -F_{ij}^{\downarrow\uparrow}$ and the symmerty of the potential $V_{ij} = V_{ji}$ we can add the two terms up and remove the one half factor. \rem{how?}
 The Hamiltonian is now ready to be diagonalized but first we know to operate a Fourier transform.\\
 
 In an homogenous material we can consider a lattice and imagine some periodic boundary conditions in all directions. There is a translation invariance.
 In heterostructures however the material may vary, let's asume
 without loss of generality, in a direction. The formalism for this new kind of superconductivity is usely described in the momentum space. However this can only work if we have
 a periodicity in the direction we want to transform on. For this reason an homogenous two-dimensional lattice involves a foruier transformation in two space dimensions while a heterostructure
 we transform only in the direction where the material is the same. For instance a multilayer material in the $x$-direction can be described using a fourier transformation in the $y$-direction.\\

 The description of the creatation and annihilation operators with a step in the momentum space for the $y$-direction can be expressed as follow:
 \begin{align}
     c_{i\sigma} =& \frac{1}{\sqrt{N}} \sum_{\bm{k}} c_{\bm{k}\sigma} e^{\im \bm{k} \bm{r}_i} \label{eq:C_Foruier_2D}\\
     c^{\dagger}_{i\sigma} =& \frac{1}{\sqrt{N}} \sum_{\bm{k}} c^{\dagger}_{\bm{k}\sigma} e^{-\im \bm{k} \bm{r}_i}\label{eq:C_dagg_Foruier_2D}
 \end{align}
 Using $N_xN_y = N$. Further to an index $i$ with can associate $\bm{r}_i = (x,y)$.
 We first want to transform the hopping term:
 \begin{equation*}
     H_{\text{hop}} = -\sum_{\langle ij\rangle \sigma} t_{ij} \sum_{\bm{k}\bm{k}'} c^{\dagger}_{\bm{k}\sigma} c_{\bm{k}'\sigma} e^{\im (\bm{k}' \bm{r}_j - \bm{k} \bm{r}_i)}
 \end{equation*}
 Here we can use the neighbour shift properties $ \bm{r}_j =  \bm{r}_i + \delta_j$ where  $\delta_j$ is a translation for $i$ to $j$. Doing so we have 
 \[
     e^{\im (\bm{k}' \bm{r}_j - \bm{k} \bm{r}_i)} = e^{\im (\bm{k}'- \bm{k} )\bm{r}_i}e^{\im \bm{k}'\delta_j}
 \]
in this disscusssion we are often going to use the following trick:
\begin{equation}
    \frac{1}{N} \sum_{i} e^{\im (\bm{k}' -  \bm{k})\bm{r}_i} = \delta_{\bm{k}\bm{k}'}
\end{equation}

 No wee need to express the neighbourhood sum. We said the material to be isotropic, this means $t_{ij} = t$. Precily we have
 \begin{align*}
         H_{\text{hop}} =& -\frac{1}{N}t\sum_{\langle ij\rangle \sigma} \sum_{\bm{k}\bm{k}'} c^{\dagger}_{\bm{k}\sigma} c_{\bm{k}'\sigma}  e^{\im (\bm{k}'- \bm{k} )\bm{r}_i}e^{\im \bm{k}'\delta_j}\\
                         =& -\frac{1}{N}t\sum_{i \sigma} \sum_{\bm{k}\bm{k}'} c_{\bm{k}\sigma}^{\dagger} c_{\bm{k}'\sigma}\left(\sum_{j} e^{\im \bm{k}'\delta_j}\right)e^{\im (\bm{k}'- \bm{k} )\bm{r}_i}\\
                         =& -t\sum_{\sigma\bm{k}} c_{\bm{k}\sigma}^{\dagger} c_{\bm{k}\sigma}\left(\sum_{j} e^{\im \bm{k}\delta_j}\right)
    \end{align*}
The chemical potential term is more easly given. In fact the number opertor yields to use two operators $c^{\dagger}c$ at a same coordinate $i$.
 \begin{equation}
     H_{\mu} = -\mu \sum_{i\bm{k}\bm{k}' \sigma} c_{\bm{k}\sigma}^{\dagger} c_{\bm{k}'\sigma} e^{\im (\bm{k}'-\bm{k})\bm{r}_i} = \sum_{\bm{k}\sigma} \mu c_{\bm{k}\sigma}^{\dagger} c_{\bm{k}\sigma}     
 \end{equation}
 We finaly have for the terms involving $c^{\dagger}_{\bm{k}\sigma} c_{\bm{k}\sigma}$:
 \begin{equation*}
     H_{\text{hop}} + H_{\mu} = \sum_{\bm{k}\sigma} \epsilon_{\bm{k}\sigma} c_{\bm{k}\sigma}^{\dagger} c_{\bm{k}\sigma}
 \end{equation*}
 using
 \[
     \epsilon_{\bm{k}\sigma}= - t\sum_{j} e^{\im \bm{k}\delta_j} + \mu.
 \]
 
 Moving on to the potential term we have to introduce a new notation. $i\pm\hat{x} = i\pm 1$  $i\pm\hat{y} = i\pm N_x$. \rem{Check if $-$ or $+$.}.
 For the brevety we use $f(a) = V_{i,a}F_{i,a} c^{\dagger}_{a\downarrow} c^{\dagger}_{i\uparrow}$
 \begin{align*}
     H_V =& V \sum_{\langle ij\rangle} F_{ij} c^{\dagger}_{i\downarrow} c^{\dagger}_{j\uparrow} + \text{h.c.} \\
     =& \sum_{i} f(i-1) + f(i+1) + f(i+N_x) + f(i-N_x) + \text{h.c.}
 \end{align*}
 We can now insert our fourier transformation introduced in Eq.\ref{eq:C_dagg_Foruier} and Eq.\ref{eq:C_Foruier} to obtain
 \begin{align*}
     H_V =& \sum_{xy} \frac{1}{N_y} \sum_{kk'} \left( V_{x,x+1} F_{x}^{x+} c^{\dagger}_{x+1,k,\downarrow}c^{\dagger}_{x,k',\uparrow} + V_{x,x-1} F_{x}^{x-} c^{\dagger}_{x+1,k,\downarrow} c^{\dagger}_{x,k',\uparrow} \right.\\
         &+ \left.V_{x,x}\left( F_{x}^{y+} e^{-\im k}  - F_{x}^{y-} e^{\im k}\right) c^{\dagger}_{x,k,\downarrow} c^{\dagger}_{x,k',\uparrow} \right) e^{-\im(k + k')y} + \text{h.c.}\\
 \end{align*}
 Defining 
 \begin{equation*}
     \begin{aligned}
     F_{xx'k} =& - V_{x,x+1} F_{x}^{x+} c^{\dagger}_{x+1,k,\downarrow}c^{\dagger}_{x,k',\uparrow} + V_{x,x-1} F_{x}^{x-} c^{\dagger}_{x+1,k,\downarrow} c^{\dagger}_{x,k',\uparrow} \\
         &+ V_{x,x}\left( F_{x}^{y+} e^{-\im k}  - F_{x}^{y-q} e^{\im k}\right)
     \end{aligned}
 \end{equation*} 

\rem{write the explicit coeficient as well as the transformation}. 

\begin{equation*}
    H = \sum_{\bm{k},\sigma} \epsilon_{\bm{k}} c_{\bm{k}\sigma}^{\dagger} c_{\bm{k}\sigma} 
    - V \sum_{\bm{k}} \left( F_{\bm{k}} c^{\dagger}_{\bm{k}\uparrow} c^{\dagger}_{-\bm{k}\downarrow}
    - F_{\bm{k}}^{\ast} c_{-\bm{k}\downarrow} c_{\bm{k}\uparrow} \right)
\end{equation*}
Involving 
\begin{align*}
    \epsilon_{\bm{k}} =& -2t \left( \cos(k_x) + \cos(k_y) \right) - \mu\\
    F_{\bm{k}} =& - V \left(F^{x+} e^{\im k_x} + F^{x-}e^{-\im k_x} + F^{y+} e^{\im k_y} + F^{y-}e^{-\im k_y} \right). 
\end{align*}
having $F{ij} = \langle c_{i\uparrow} c_{j\downarrow}$ and $F^{r\pm} = F_{i, \pm r}$. We do as well see that the correlation functions is the same
for each lattice site $i$.\\

In a similar way as before, we aim to use the BdG-formalism to solve the Hamiltonian thanks to the eigenvalues and -vector of the system.
To achieve this earlier discussions outligned the need to bring the Hamiltonian in a matrix shape.\\



--------------------------------------------\\
The description of the creatation and annihilation operators with a step in the momentum space for the $y$-direction can be expressed as follow:
\begin{align}
    c_{xy\sigma} = \frac{1}{\sqrt{N_y}} \sum_{k_y} c_{x k_y\sigma} e^{\im k_y y} \label{eq:C_Foruier}\\
    c^{\dagger}_{xy\sigma} = \frac{1}{\sqrt{N_y}} \sum_{k_y} c^{\dagger}_{x k_y\sigma} e^{-\im k_y y}\label{eq:C_dagg_Foruier}
\end{align}
For the readebility we are going to use $k_y \rightarrow k$. Further to an index $i$ with can associate $(x,y)$. An index $j$ can be associated with $(x',y')$.
We first want to transform the hopping term:
\begin{equation*}
    H_{\text{hop}} = -\sum_{\langle ij\rangle \sigma} t_{ij} \sum_{kk'} c^{\dagger}_{x k\sigma} c_{x' k'\sigma} e^{\im (k' y' - k y)}
\end{equation*}
Here we can use the neighbour shift properties $y' = y + \delta_y$ where $\delta_y = \pm 1$. Doing so we have 
\[
    e^{\im (k' y' - k y)} = e^{\im (k' (y+\delta_y) -  k y)} = e^{\im (k' -  k)y}e^{\im k'\delta_y}
\]
No wee need to express the neighbourhood sum. Precily we have
\begin{align*}
        H_{\text{hop}} =& -\sum_{\langle ij\rangle \sigma} t_{ij} \sum_{kk'} c^{\dagger}_{x k\sigma} c_{x' k'\sigma}  e^{\im (k -  k')y}e^{\im k'\delta_y}\\
                        =& -\sum_{xy \sigma} \sum_{kk'}  \biggl( t_{x,x+1} \underbrace{c_{x k\sigma}^{\dagger} c_{x+1 k'\sigma}e^{\im k'\cdot (0)}}_{\text{$+x$ hopping, no $\delta_y$}} +  t_{x,x-1} \underbrace{c_{x k\sigma}^{\dagger} c_{x-1 k'\sigma} e^{\im k'\cdot (0)}}_{\text{$-x$ hopping, no $\delta_y$}}\\
                        & ~~~~~~~~~~~~+t_{x,y-1} \underbrace{c_{x k\sigma} c_{x k'\sigma}^{\dagger} e^{\im k'\cdot (-1)}}_{\delta_y = -1} + t_{x,y+1} \underbrace{c_{x k\sigma}^{\dagger} c_{x k'\sigma}e^{\im k'\cdot (1)}}_{\delta_y = 1}\biggr)e^{\im (k -  k')y}
\end{align*}
As we see the $y$ direction is now expressed in the $k$-index, which is unique for each lattice $y$-slice. The information is then conserved. 
We know that system have different material on the $x$ axis. This means $t_{x,y-1} = t_{x,x} = t_{x,y+1}$ because the material are isotropic but every material has a 
different hopping term. Beside we can use the following realtion $1/N_y \sum_{y} e^{\im (k -  k')y} = \delta_{kk'}$. Perfoming both expression leads after a summation over $k'$ to
\begin{equation*}
    H_{\text{hop}} = -\sum_{x k \sigma} t_{x,x+1} c_{x k\sigma}^{\dagger} c_{x+1 k\sigma} + t_{x,x-1} c_{x k\sigma}^{\dagger} c_{x+1 k\sigma} +  t_{x,x} c_{x k\sigma}^{\dagger} c_{xk\sigma}\left(e^{\im k} + e^{-\im k}\right)
\end{equation*}
And now we can reintroduce an arbitray second coordinate $x'$ to describe the neighbours.
\begin{equation}
    H_{\text{hop}} = -\sum_{xx' k \sigma} t_{x,x'} c_{x k\sigma}^{\dagger} c_{x' k\sigma} \left(\delta_{x+1,x'} + \delta_{x-1,x'} + \delta_{x,x'} 2 \cos(k)\right)
\end{equation}
The chemical potential term is more easly given. In fact the number opertor yields to use two operators $c^{\dagger}c$ at a same coordinate $i$.
\begin{equation}
    H_{\mu} = -\mu \sum_{x kk' \sigma} c_{x k\sigma}^{\dagger} c_{x k'\sigma} e^{\im (k-k')y} = \sum_{xx'k\sigma} \mu c_{xk\sigma}^{\dagger} c_{xk'\sigma} \delta_{xx'}     
\end{equation}
We finaly have for the terms involving $c^{\dagger}_{xx'\sigma} c_{xx'\sigma}$:
\begin{equation*}
    H_{\text{hop}} + H_{\mu} = \sum_{xx'k\sigma} \epsilon_{xx'k\sigma} c_{xk\sigma}^{\dagger} c_{x'k\sigma}
\end{equation*}
using
\[
    \epsilon_{xx'k\sigma}= - t_{xx'} \left(\delta_{x+1, x'} + \delta_{x-1,x'}\right) - \left(t_{xx'}2\cos(k) + \mu\right) \delta_{xx'}
\]

Moving on to the potential term we have to introduce a new notation. $i\pm\hat{x} = i\pm 1$  $i\pm\hat{y} = i\pm N_x$. \rem{Check if $-$ or $+$.}.
For the brevety we use $f(a) = V_{i,a}F_{i,a} c^{\dagger}_{a\downarrow} c^{\dagger}_{i\uparrow}$
\begin{align*}
    H_V =& V \sum_{\langle ij\rangle} F_{ij} c^{\dagger}_{i\downarrow} c^{\dagger}_{j\uparrow} + \text{h.c.} \\
    =& \sum_{i} f(i-1) + f(i+1) + f(i+N_x) + f(i-N_x) + \text{h.c.}
\end{align*}
We can now insert our fourier transformation introduced in Eq.\ref{eq:C_dagg_Foruier} and Eq.\ref{eq:C_Foruier} to obtain
\begin{align*}
    H_V =& \sum_{xy} \frac{1}{N_y} \sum_{kk'} \left( V_{x,x+1} F_{x}^{x+} c^{\dagger}_{x+1,k,\downarrow}c^{\dagger}_{x,k',\uparrow} + V_{x,x-1} F_{x}^{x-} c^{\dagger}_{x+1,k,\downarrow} c^{\dagger}_{x,k',\uparrow} \right.\\
        &+ \left.V_{x,x}\left( F_{x}^{y+} e^{-\im k}  - F_{x}^{y-} e^{\im k}\right) c^{\dagger}_{x,k,\downarrow} c^{\dagger}_{x,k',\uparrow} \right) e^{-\im(k + k')y} + \text{h.c.}\\
\end{align*}
Defining 
\begin{equation*}
    \begin{aligned}
    F_{xx'k} =& - V_{x,x+1} F_{x}^{x+} c^{\dagger}_{x+1,k,\downarrow}c^{\dagger}_{x,k',\uparrow} + V_{x,x-1} F_{x}^{x-} c^{\dagger}_{x+1,k,\downarrow} c^{\dagger}_{x,k',\uparrow} \\
        &+ V_{x,x}\left( F_{x}^{y+} e^{-\im k}  - F_{x}^{y-q} e^{\im k}\right)
    \end{aligned}
\end{equation*} 

we obtain a BdG-transformed Hamiltonian
\begin{align*}
    H =& \sum_{xx'k} D_{xk}^{\dagger} H_{xx'k} D_{x'k}\\
      =& \sum_{xx'k} \begin{pmatrix}
        c^{\dagger}_{xk\uparrow} & c_{x,-k,\downarrow}
      \end{pmatrix}
        \begin{pmatrix}
            \epsilon_{xx'k\uparrow} & F_{xx'k}\\
            -F_{xx'k}^{\ast} & -\epsilon_{xx'k\downarrow}
        \end{pmatrix}
        \begin{pmatrix}
            c_{x'k\uparrow}\\
            c^{\dagger}_{x',-k,\downarrow}
        \end{pmatrix}
\end{align*}
The summation over all $x,x'$ can be represented in a new matrix.
\[
    H = \sum_k D_k^{\dagger} H_k D_k
\]
involving the $4Nx \times 4Nx$ matrix $H_k$ and the $4Nx$-dimensional vector $D_k$.
\[
    H_k = \begin{pmatrix}
        H_{11k} &\dots & H_{1N_xk}\\
        \vdots&\ddots&\\
        H_{N_x1k} & & H_{N_xN_xk}
    \end{pmatrix}
\]
As before the $y$ information is stored in the $k$-index, which is unique for each lattice $y$-slice. This said, we 
can diagonalize $N_y$ times a $4N_x \times 4N_x$ matrix. $H_k$ represent the interaction of a $y$-line with itself.
The eigenvalues are the same for each $y$-slices and physical
quantities are going to be expressed with this summation over $k$ and the eigenvalues, -vectors of each $k$ ($y$-slice).\\

On the other hand the vector we use to carry the creation and annihilation operators is given as 
\[
    D_k^{\dagger} = \begin{pmatrix}
        c^{\dagger}_{1k\uparrow} & c_{1,-k,\downarrow}& \dots &c^{\dagger}_{N_x k\uparrow} & c_{N_x,-k,\downarrow}
    \end{pmatrix}
\]

\subsection{BdG-transformation}
The eigenvalues equation is similar to Eq.\ref{eq:BdG_eigenVal_H}
\begin{equation}\label{eq:BdG_eigenVal_H_k}
    H_k \mathfrak{X}_{nk} = E_{nk} \mathfrak{X}_{nk}
\end{equation}
The eigenvectors and -values are given as
\begin{equation*}
    \mathfrak{X}_{nk} = \begin{pmatrix}
        \mathfrak{x}_{n1k}\\
        \vdots\\
        \mathfrak{x}_{nN_xk}\\
    \end{pmatrix},~~ \mathfrak{x}_{nik} = \begin{pmatrix}
        u_{nik}\\
        v_{nik}
    \end{pmatrix}
\end{equation*}
if we stick to the formalism we already derived in the earlier Sec.\label{sec:Diagonalization} we obtain similar 
eigenvectors where $u_{nik}$ corresponds to $c$. We are now going to transform the $c$ operators. First we need
to define $\mathfrak{X}_k = [\mathfrak{X}_{1k}, ..,\mathfrak{X}_{2Nk}]$ and $\mathfrak{g}_k\in\mathbb{H}^{2N}$ along with $D_{k} := \mathfrak{X}_k\mathfrak{g}_k$, \rem{recheck all}
delivering $\mathfrak{g}_k = \mathfrak{X}^{\dagger}_k D_{k}$ which is equivalent to:
\begin{equation}
    \mathfrak{g}_{nk} = u_{nk}c_{k\uparrow} + v_{nk}c^{\dagger}_{-k\downarrow}
\end{equation}
which then delivers 
\begin{center}
    \begin{minipage}{0.4\textwidth}
    \begin{equation}\label{eq:BdG_transf_c}
        c_{k\uparrow} = \sum_{n\in\natset{2N}} u_{nk}\mathfrak{g}_{nk} 
    \end{equation}
    \end{minipage}\hspace{0.05\textwidth}
    \begin{minipage}{0.03\textwidth}
        \begin{tikzpicture}
            \coordinate (a) at (0,0.5);
            \coordinate (b) at (0,-0.5);
            \draw[-] (b) -- (a);
            \filldraw[color=black, fill = white , thin] (a) circle (0.05);
            \filldraw[color=black, fill = white , thin] (b) circle (0.05);
        \end{tikzpicture}
    \end{minipage}
    \begin{minipage}{0.45\textwidth}
        \begin{equation}
        \label{eq:BdG_transf_c_dagg}
            c_{-k\downarrow}^{\dagger} =\sum_{n\in\natset{2N}} v_{nk}\mathfrak{g}_{nk} 
        \end{equation}
    \end{minipage}
    \end{center}

\subsection{Pairing amplitudes}
We first need to $F_i^{\pm x}$ unsing the fourier transformation. Then we are going to use our BdG-transformation as we did earlier to be 
able to solve these parameters self-consistently using the eigenvectors and -values.\\

Here we are goinng to transfrom in both $x$ and $y$ direction. \rem{why not just y}?
\begin{equation*}
    F_i^{\pm x} = \langle c_{i\uparrow} c_{i\pm x,\downarrow}\rangle =
    \frac{1}{\underbrace{N_x\cdot N_y}_{N}} \sum_{\bm{k}\bm{k}'} \langle c_{\bm{k}\uparrow} c_{\bm{k}'\downarrow} \rangle e^{\im(\bm{k} + \bm{k}')\bm{r}_i} e^{\pm\im\bm{k}'_x}
\end{equation*}
The expecation value is independant of the site $i$. This means that we can achieve a site description by making the average of the value over the system. This means 
we can write for the total pairing amplitude on an axis:
\begin{equation*}
    F^{\pm x} = 
    \frac{1}{N}\sum_i \frac{1}{N} \sum_{\bm{k}\bm{k}'} \langle c_{\bm{k}\uparrow} c_{\bm{k}'\downarrow} \rangle e^{\im(\bm{k} + \bm{k}')\bm{r}_i} e^{\pm\im\bm{k}'_x}
\end{equation*}
which after the sumation over $i$ results as we covered earlier as $1/N \sum_{i} e^{\im(\bm{k} + \bm{k}')\bm{r}_i} = \delta_{\bm{k},-\bm{k}'}$
\begin{equation*}
    F^{\pm x} = 
    \frac{1}{N} \sum_{\bm{k}} \langle c_{\bm{k}\uparrow} c_{\bm{k}\downarrow} \rangle e^{\mp\im\bm{k}_x}
\end{equation*}
Now that we have simplifed the fourier transform we can incorporate the BdG-transformation. The process is very similar to Eq.\ref{eq:transfo_c_up_c_down_BdG} and yields
\begin{equation}\label{eq:transfo_F_x}
    \begin{aligned}
    F^{\pm x} =& \frac{1}{N}\sum_{\bm{k}}\sum_{nn'} u_{nk}v_{nk}^{\ast} \langle \mathfrak{g}_{n\bm{k}} \mathfrak{g}_{n'\bm{k}}^{\dagger} \rangle e^{\mp\im\bm{k}_x}\\
        =& \frac{1}{N} \sum_{n,\bm{k}} u_{nk} v_{nk}^{\ast}\left(1 - f(E_{nk})\right) e^{\mp \im \bm{k}_x}
    \end{aligned}
\end{equation}
In the very same way we obtain 
\begin{equation}
    F^{\pm y}= \frac{1}{N} \sum_{n,\bm{k}} u_{nk} v_{nk}^{\ast}\left(1 - f(E_{n\bm{k}})\right) e^{\mp \im \bm{k}_y}.
\end{equation}

\subsection{Advanced order parameters}
In the more simple desciption of the superconductivity we outlied how the superconducting order parameter $\Delta$ depends on the pairing amplitude $F$.
Because the potential was isotropic, we simply had $\Delta = U \langle c_{i\uparrow}c_{i\downarrow}\rangle$. Here however the potential is anisotropic 
and we obtain a linear compination of the pairing amplitudes in the different directions of the lattice. Achieving different combinations of the $F$s 
we obtiain different superconducting states. Here is an exaustive list. \rem{repalce V with U for consistency?}
\begin{alignat}{3}
    &\Delta_{s,i} = V F_{s,i} &= \frac{V}{4} \left(F_{i}^{x+(S)} + F_{i}^{x-(S)} + F_{i}^{y+(S)} + F_{i}^{y-(S)}\right)\\
    &\Delta_{d,i} = V F_{d,i} &= \frac{V}{4} \left(F_{i}^{x+(S)} + F_{i}^{x-(S)} - F_{i}^{y+(S)} - F_{i}^{y-(S)}\right)\\
    &\Delta_{p_x,i} = V F_{p_x,i} &= \frac{V}{2} \left(F_{i}^{x+(T)} - F_{i}^{x-(T)}\right)\\
    &\Delta_{p_y,i} = V F_{p_y,i} &= \frac{V}{2} \left(F_{i}^{y+(T)} - F_{i}^{y-(T)}\right)
\end{alignat}
$F_s$, $F_d$, $F_{p_x}$ and $F_{p_y}$ are the pairing amplitudes for the $s$, $d$ (also called $d_{x^2-y^2}$ because of its expression), $p_x$ and $p_y$-wave superconductivity.
The $S$ and $T$ are the singlet and triplet expressions of the pairing amplitudes. More precily we define them as
\begin{align}
    F_{ij}^{(S)} = \frac{F_{ij} + F_{ji}}{2}\\
    F_{ij}^{(T)} = \frac{F_{ij} - F_{ji}}{2}
\end{align}
Where we shorten the expression using $F_{ij}^{\uparrow\downarrow(S)} = F_{ij}^{(S)}$.
\paragraph{Symmetry discussion} $~$ We see that these parameter depends on the spin and the momentum. Therfore it's a good idea to look at their respective
behaviour under exchange of these variables.\\

The discussion for the spin echange is quite straight forward.
\begin{align*}
    F_{ij}^{\uparrow\downarrow(S)} =& \frac{F_{ij}^{\uparrow\downarrow} + F_{ji}^{\uparrow\downarrow}}{2} = \frac{\langle c_{i\uparrow}c_{j\downarrow}\rangle + \langle c_{j\uparrow}c_{i\downarrow}\rangle}{2}\\
    F_{ij}^{\uparrow\downarrow(T)} =& \frac{F_{ij}^{\uparrow\downarrow} - F_{ji}^{\uparrow\downarrow}}{2} = \frac{\langle c_{i\uparrow}c_{j\downarrow}\rangle - \langle c_{j\uparrow}c_{i\downarrow}\rangle}{2}
\end{align*}
\rem{What's happening when we invert both spin?} using $\langle c_{i\uparrow}c_{j\downarrow}\rangle = -\langle c_{i\downarrow}c_{j\uparrow}\rangle$. Using the linearity of the expression we obtain
\begin{align*}
    F_{ij}^{\uparrow\downarrow(S)} = - F_{ij}^{\downarrow\uparrow(S)}\\
    F_{ij}^{\uparrow\downarrow(T)} = F_{ij}^{\downarrow\uparrow(T)}.
\end{align*}
This means that the singlet wave-pairing amplitude is antisymmetric under spin exchange and the triplet wave pairing is symmertic under spin excahnge. Their names
find place in the analogy of the wavefunction formalism.\\

For the momentum exchange we re going to take a look at the fourier transformation of the pairing amplitudes. Using the transformation we made to 
reach Eq.\ref{eq:transfo_F_x} with any translation $ x \rightarrow \bm{r}$, $r\in \{\bm{e}_x,\bm{e}_y\}$ we obtain
\begin{align*}
    F_{i,i+ \bm{r}} = \frac{1}{N} \sum_{\bm{k}} \langle c_{\bm{k}\uparrow} c_{-\bm{k},\downarrow}\rangle e^{-\im\bm{k}\bm{r}}.
\end{align*}
Changing the sign of the momentum we obtain  $\frac{1}{N} \sum_{\bm{k}} \langle c_{-\bm{k},\uparrow} c_{\bm{k},\downarrow}\rangle e^{\im\bm{k}\bm{r}} =  F_{i,i- \bm{r}}$
because of the $\delta_{\bm{k},\bm{k}'}$ trick. For this reason $F_{i,i+ \bm{r}} +  F_{i,i- \bm{r}}$ is symmertic under momentum exchange while $F_{i,i+ \bm{r}} -  F_{i,i- \bm{r}}$ is antisymmetric.\\
Refering back to the order parameter definiton we see that the $s$ and $d$-wave are symmertic under momentum exchange, where the $p_x$ and $p_y$-wave are antisymmetric in such exchange.\\

\subsection{Implementation of the advanced superconductivity}
\end{document}  