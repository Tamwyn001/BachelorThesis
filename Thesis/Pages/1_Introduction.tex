\documentclass[../main.tex]{subfile}
\begin{document}
\section{Introduction}

Superconductivity has today a wide range of applications like transportation with levitating trains. These are mostly implemented in China~\cite{Roque2024} and Japan around 2027~\cite{Nishijima2013}.
Important imaging methods have emerged, like terahertz imaging. This frequency range is between the normal usecases of 
optical components (400-800 THz) and electrical devices (up to the GHz range), giving the opertunity to picture a wider range of materials.
Further nuclear magnetical resonnance imaging (NMRI) takes advantage of the response of the nuclear spin to an external 
magnetical field to provide a non intrusive spectroscopy technique. To provide stronger field and better imaging definition, superconducting magnets are used \cite{Nakamura2015}. In fact they
allow stable and strong fields up to arround 24T \cite{Hashi2015}, whereas conventional permanent magnets are limited around 2T \cite{Haishi2005}.\\
Beside superconducting magnetic energy storage (SMES) can store arround one and a hundred MW of energy, whith a discharge time of a few seconds. This makes
this technology interesting for the grid stabilisation, providing a very responsive energy storage system \cite{Tixador2008}.
SMES is based on a superconducting coil within a cryostat that stores the energy in the coil and releases it when needed. Due the the superconducting nature of the coil,
an energy converson efficiency of 95\% is observed \cite{Tixador2008}.\\

This very promissing state of matter has a main drawback, the freezing low critical temperature. Over the past century, researchers 
have been searching for materials that could display superconducting properties at higher temperatures \cite{Bednorz1986} (Nobel im 1987).
\cite{PhysRevLett.58.908} found critical temperature above the liquid nitrogen temperature.
Special dopping methodes were proven to improve $T_C$ \cite{Doiron-Leyraud2007} in 2007.
In 2015 researcher found that appling a substrat to single-layer FeSe films could improve the critical temperature \cite{Ge2015}.
And a lot of other studies that extended the range of materials that could be superconducting at higher temperatures \cite{Einaga2016}\cite{Drozdov_2019} \cite{lee2023}.\\
These progress were among other, achieved by increasing considerably the pressure, even to gigapascals in some studies \cite{Drozdov_2019}.\\


Further unconventional d-wave superconductivity has been proven to be stable at higher temperatures than its conventional BCS-homologue.
The applications we provided above need liquid hydrogen to stay in the temperatures that allows superconductivity. D-wave
would work with liquid nitrogene for exemple, which is a strong economic argument.\\

Superconductors were proven to exhibit diffent characteristics in comparaison to normal conductors as
Meissner and Ochsenfeld discovered in 1933 \cite{MeissnerOchsenfeld1933}. 
The experimented tin Sn and lead Pb beeing subject to a magnetic field bellow a crtical temperature $T_c$. They observed that the magnetic field lines on the surface
of the superconductor were ordering in such a way, that no magnetical flux lines were able to penetrate the superconductor ($\bm{B} = 0$ inside). This has been called the Meissner effect
and presents the superconductors as perfect diamagnets.\\

\paragraph{Notation}$~$\\
Fist a few words about the notation. When having multiple indices under a symbol we ommit the comma to make the notation more readable. 
For example, $x_{ij}$ is the same as $x_{i,j}$. However when we have complexer indices such as $x_{i-1, j}$, we keep the comma between all
the indicies. For the sumation over natural numbers we use the notation $\natset{N} = \{1,..,N\}_{\mathbb{N}}$.\\

We are going to use Hilbert-space's operators. As often in physics they will be mutliplied with the wavefunction. This is the case for the Hamiltonian 
operator for example. However the second quantisation opertors works diffently. We will give a proper definition when the moment comes.
\end{document}

