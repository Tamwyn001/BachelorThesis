\documentclass[../main.tex]{subfile}
\begin{document}

\section{Theoretical Background}
In order to describe the superconductors we are going to introduce the second quantization formalism based on \cite{Folk2014}.
This framework allows us to picture the wave function of a system using creation and annihilation operators over
energy states of the system. This simplifies a lot the notation. The mathematical foundation of this formalism lies
in the Hilbert space and requires as well its dual space. To picture the many-particles interaction we need as well
to introduce the Fock space. We aim to give a practical understanding over the formal mathematical rigour.
In this sense some derivations might not be as rigorous as a proper mathematical treatment could deliver.\\

We want to work on fermionic systems. The formalism stays the same for 
bosons, but the results are fundamentally different. One can mention the Pauli principle as an example which
only applies on fermions. After some definition we are going to see how to derive it with the help of the second quantization.\\ 

Please note that we aim to describe the operators with a hat: $\hat{\circ}$.
Later the expression will however involve a dense notation, such that we will drop the hat for the readability. The reader will stay
informed about the nature of each symbol. \\

\subsection{Bosons and fermions}
We consider without loss of generality the following hamiltonian
\begin{equation}\label{eq:MasterHamiltonian}
    \hat{H} = \hat{H}_0 + \hat{H}_I.
\end{equation}
It is a very generic definition that uses a single particle operator $\hat{H}_0$ and the interaction operator $\hat{H}_I$. The single 
particle part sums up the kinetic and potential energy of each particle. The interaction part gives room for complex relations
between the particles.\\ 
\[
    \hat{H}_0 = \sum_{i\in \natset{N}} \hat{h}_i(x_i)~,~~ \hat{h}_i(x_i) = -\frac{\hbar^2}{2m}\nabla_i^2 + \hat{U}(x_i).
\]
The potential part may depend on the particle's position $\bm{r}$ or spin $s$. 
We use $x_i := (\bm{r}, s) \in \mathcal{X}\subseteq\mathbb{R}^{3}\times\mathbb{S}$ to group this information. For example, we have for an electron:
$\mathbb{S}= {-\frac{1}{2},\frac{1}{2}}$.\\

Further we describe a quantum state that a particle can occupy with a wavefunction $\phi_{\nu}(x)$, 
which is related to a certain energy $\epsilon_n \in\mathbb{R}$. This energy depends on 
the wave vector and the spin of the particle: $\nu = (\bm{k}, \sigma)$. The fundamental equation of
quantum mechanics relates the wavefunction with the hamiltonian using the energy of the state:
\[
    \hat{h} \phi_{\nu}(x) = \epsilon_\nu \phi_{\nu}(x)
\]
The wavefunction lies in the Hilbert space $\mathcal{H}$. Therefore, $\phi_{\nu}(x)$ are eigenfunctions or -states of
the Hamiltonian with eigenvalues $\epsilon_{\nu}$. Further the wavefunction should build an orthonormal basis:
\[
    \int_\mathcal{X} \phi_{\nu'}(x) \phi_{\nu}(x) \dd x = \delta_{\nu'\nu}.
\]
$\nu$ and $\nu'$ are two different states. We introduced here the Kronecker delta $\delta_{\nu'\nu}$, which is one when the two indices
are equal, and zero otherwise. Because the spin $s$ is not continuous, one can understand the integral in the following way:
\[
    \int_\mathcal{X} \dd x = \sum_{s\in \mathbb{S}} \int_{\mathbb{R}^3} \dd^3 r
\]  
where $ \int_{\mathbb{R}^3}\dd^3 r = \int_{\mathbb{R}}\int_{\mathbb{R}}\int_{\mathbb{R}} \dd r_1 \dd r_2 \dd r_3$.
In other words we integrate over all possible states.\\

Now that we can picture a single particle, the next question is: How to describe the collective behaviour of a collection of such particles?
This is done with a many-body wavefunction, that sums up all possible combinations of wavefunctions in the system and should stay normalized. 
A combination is illustrated as the product of each possible wavefunction of the particle in a certain state. These particles
can be swapped, and therefore we need to consider all the combinations.
We aim to work with fermions but give a quick insight with bosons for comparison. We admit having $N \in \mathbb{N}$ particles in the system.\\

\paragraph{Bosons}$~$\\

Boson are more flexible because they are not restrained by the Pauli principle.
Their many-particle wavefunction is symmetric (exponent $S$) under swap of two particles.
\[
    \Phi^{(S)}(x_1,..,x_N) = \left(N!\prod_{N}(n_{\nu})!\right)^{-\frac{1}{2}} \sum_{P\in S_n} P \phi_{\nu_1}(x_1)\cdot ..\cdot \phi_{\nu_N}(x_N).
\]
This represents an eigenfunction of the non-interacting bosonic-Hamiltonian.
We used $n_{\nu}$, which represents the number of particle in the state $\nu$. Therefore, we usually call it the occupation number of the state $\nu$.
For bosons this integer has no constraint in general.
The permutation set $S_n$ contains all the possible combinations of $x_i$ in the state $\nu_j$ for $i,j\in\natset{N}$. $P$ describes a permutation
in the $x_i$.\\

For example with two particles we have $x_1$ and $x_2\in\mathcal{X}$ and the later part of the expression reads 
$\phi_{\nu_1}(x_1) \cdot \phi_{\nu_2}(x_2) + \phi_{\nu_1}(x_2) \cdot \phi_{\nu_2}(x_1)$ and is thanks to the prefactor, normalized.
We now see, the permutation aims to describe each particle (an $x_i$) in each possible state $\nu_i$.

\paragraph{Fermions}$~$\\

Fermions are a bit different. Their many-particle wavefunction is antisymmetric under swap of two particles. We denote it as
\[
    \Phi^{(A)}(x_1,..,x_N) = \left(N!\right)^{-\frac{1}{2}} \sum_{P\in S_n} \signum{P}\cdot P \phi_{\nu_1}(x_1)\cdot ..\cdot \phi_{\nu_N}(x_N),
\]
which is an eigenfunction of the non-interacting fermionic-Hamiltonian.
$\text{Sgn}$ represents the signum function. Applied on a permutation $P$, it is one if $P$ is even and minus one if $P$ is odd.\\
We already know that the Pauli principle implies that we can find up to one particle in each energy state. We therefore have $n_\nu \in \{0,1\}$. The normalization
factor is the same but the product over the $n_{\nu}$ is always one, we have $0!=1$ and $\prod 1 = 1$.  \\

At this point one might have recognized the formula of the determinant
\[
    \Phi^{(A)}(x_1,..,x_N) = \left(N!\right)^{-\frac{1}{2}} \text{det}\begin{pmatrix}
        \varphi_{\nu_1}(x_1)& \cdot\cdot &\varphi_{\nu_1}(x_N)\\
        \vdots&  &\vdots\\
        \varphi_{\nu_N}(x_1)& \cdot\cdot &\varphi_{\nu_N}(x_N)\\

    \end{pmatrix}.
\]
We usually describe this expression as the Slater determinant. A determinant vanishes if two rows or columns are identical. If they are the same, we have two particles 
in the same state. This means that the probability of finding two fermions in the same state is zero.
This is the Pauli principle. Only one or no particle may occupy each state. From this we get the constraint for $n_{\nu}$.\\

However following this method may lead to a major problem. The many-particle wavefunction of fermions is defined up to a sign. For instance if we consider
two particles ``having'' $x_1$ and $x_2$, we have two possible state $\nu_1$ and $\nu_2$. Two possible solution are
\begin{align*}
    &\Phi^{(A_1)} = \frac{1}{\sqrt{2}} \bigl(\varphi_{\nu_1} (x_1)\varphi_{\nu_2} (x_2) - \varphi_{\nu_1} (x_2)\varphi_{\nu_2} (x_1) \bigr)\\
    \text{or~} &\Phi^{(A_2)} = \frac{1}{\sqrt{2}} \bigl(\varphi_{\nu_1} (x_2)\varphi_{\nu_2}(x_1) - \varphi_{\nu_1} (x_1)\varphi_{\nu_2} (x_2)\bigr)\\
    &~~~~~~~=-\Phi^{(A_1)}.
\end{align*}
This sign difference may lead to computation errors. To solve this we must give a labelling to our states when we count them, and keep it when we 
come to build the Slater determinant.\\

These bosonic and fermionic wavefunctions are eigenstate of the Hamiltonian $\hat{H}_0$ and the corresponding eigenvalue $E_{\nu}$
is given by summing the energy of each state times its occupation number: $E_{\nu} = \sum_{\nu} \epsilon_{\nu} n_{\nu}$.
The orthogonal property of the single-particle wavefunction propagates itself:
\[
    \int_{\mathcal{X}^N} \Phi_a^{\ast}(x_1,..,x_N) \Phi_b(x_1,..,x_N) \dd^N x = \delta_{ab}.
\]
Therefore, we can expend any many-particle wavefunction $\Psi$ as the linear combination of these:
\[
    \Psi = \sum_a f_a \Phi_a(x_1,..,x_N)
\]
where $f_a$ is a coefficient and $a$ a labelling.\\

What we just discussed is the so-called first quantization- or wavefunction formalism. Now we intend to introduce a more compact description of our system. 

\subsection{The second quantization}
For a better description of the many-particle system we introduce a simpler notation. The second quantization lies on three important objects. 
States, which are described as ``kets''. We put any relevant information (e.g. quantum numbers) in the ket: $|\bm{k}, \sigma,..\rangle$. 
Then we need two operators that act on these states to allow evolutions in the system. 
The second quantization has two main operators. They can create and annihilates a state.\\

\paragraph{States} $~$\\

In this section we describe a state as the number of particle that occupies each single-particle state. As explained before, we need to
order the state: $1<2<..<N$. We can then describe the wavefunction as follows $|n_{1},..,n_{N}\rangle$.\\

Further the state where no particle are present is called the vacuum state, and we denote it as $|0_{\nu_1},..,0_{\nu_N}\rangle = |0\rangle$.

\subsubsection{Second quantization for fermions}$~$\\

It is now the time to define the operators for the fermionic case.
\paragraph{Creation operator $c_\nu^{\dagger}$}$~$\\

The creation operator adds a particle in the concerned state and introduce an additional phase:
\[
    c_{\nu}^{\dagger} |n_{1},..,n_{\nu},..\rangle = (-1)^{\sum_{\mu<\nu}n_{\mu}} (1-n_{\nu})|n_{1},..,n_{\nu}+1,..\rangle
\]
We intentionally discarded the hat on the $c$ as said before.
We notice the $ (1-n_{\nu})$ term, which avoids creating a particle in an already occupied state. This is the way we express
the Pauli principle. 
Further we can construct a state by applying this operator one after another on the vacuum state. To avoid the minus one adding a negative sign, we start from the 
end, and add the particle backwards in the order of the state:
\[
    |n_{1},..,n_{N}\rangle = (c_{1}^{\dagger})^{n_{1}}\cdot ..\cdot (c_{N}^{\dagger})^{n_{N}} |0\rangle
\]  

\paragraph{Annihilation operator $c_\nu$}$~$\\

Likewise, the annihilation operator destroys a particle in the corresponding state. The operator reads
\[
    c_{\nu} |n_{1},..,n_{\nu},..\rangle = (-1)^{\sum_{\mu<\nu}n_{\mu}} (n_{\nu})|n_{1},..,n_{\nu}-1,..\rangle.
\]
We can easily recognize that due to the $n_{\nu}$-term, destroying a particle that does not exist gives zero, 
so it is only possible to annihilate existing particles. Now that we can bring the particles into new states, we can introduce some rules
the operators yield.

The anticommutator of two operators reads $[A,B]_{+}$ or $\{A,B\} := AB + BA$ and is an operator as well.
We are going to stick with $[A,B]_{+}$ since it is more consistent with the commutator notation $[A,B]_{-}$ (or simply $[A,B]$).\\

The following results are obtained by separating the $\nu = \mu$ from the $\nu \neq \mu$. We must also say that the dagger $\dagger$ 
should be understood as the complex transpose of the operator: $(AB)^{\dagger} = B^{\dagger}A^{\dagger}$.
\begin{align}
    [c_{\nu},c_{\mu}]_+ =&~ 0 \label{eq:Fermion1} \\
    [c^{\dagger}_{\nu},c^{\dagger}_{\mu}]_+ =&~0 \label{eq:Fermion2} \\
    [c^{\dagger}_{\nu},c_{\mu}]_+ =&~ \delta_{\mu,\nu}\label{eq:Fermion3} 
\end{align} 
Fermionic operators ``anticommute''.
We can combine the creation and annihilation operators to count the number of particles in a state:
\[
    c_{\nu}^{\dagger} c_{\nu} |n_{1},..,n_{\nu},..\rangle = n_{\nu}|n_{1},..,n_{\nu},..\rangle.
\]  
From this we can define the number operator $\hat{n}_{\nu}:= c_{\nu}^{\dagger} c_{\nu}$ which we can combine in the total number operator
\[
    \hat{N} = \sum_{\nu} \hat{n}_{\nu}~,~~\text{where logically}~ N = \sum_{\nu} n_{\nu}.
\]
If we apply the total number operator on the system wavefunction, we obtain the total number of particles.
We kept the hat on the operator to avoid the confusion with the actual number $n_{\nu}$ of particles.\\

\paragraph{Second quantization description of the single- and two- particle operators}$~$\\
% We first need to make an important observation between the Slater determinant and the single particle state to
% understand the following. 

The next step is to translate the Hamiltonian in our formalism.
First we introduce two basis element $|\Phi_{\alpha}\rangle$ and $|\Phi_{\beta}\rangle$, which can be many-particles eigenstate of the system.
We can also call them Slater determinants.
Further we introduce the probability of the configuration $|\Phi_{\alpha}\rangle$ to scatter into the $|\Phi_{\beta}\rangle$ due to the action of an operator $A$ (momentum, potental, interactions,..).
This is described by the matrix element $\langle\Phi_{\alpha}|A|\Phi_{\beta}\rangle$ which involves the single particle states $|\alpha_1\rangle, .., |\alpha_N\rangle$ of $|\Phi_{\alpha}\rangle$
and $|\beta_1\rangle, ..,|\beta_N\rangle$ of $|\Phi_{\beta}\rangle$.
\[
    \langle\Phi_{\alpha}|A|\Phi_{\beta}\rangle = \sum_{ij} C_{ij}\langle\alpha_i|A|\beta_j\rangle
\]
involving some constants $C_{ij}$. This describes the overlap of the two configurations, after that we modified $|\Phi_{\beta}\rangle$ with $A$.
On the right-hand side (r.h.s) we introduced the braket scalar product notation. The bra $\langle\alpha|$ lives in the dual space
of the Hilbert space. One reads it as the complex transpose of $|\alpha\rangle$.\\

We recall the single particle Hamiltonian we introduced earlier. Its second quantization representation reads
\begin{equation}\label{eq:SecondquantizationSingleParticle}
    \hat{H}_0 = \sum_{i \in \natset{N}} \hat{h}(x_i) ~ \leadsto ~~ \sum_{\alpha\beta} \langle \alpha|\hat{h}|\beta\rangle c_{\alpha}^{\dagger}c_{\beta}
\end{equation}
where $\alpha$ and $\beta$ are single-particle states of the system. $c_{\alpha}^{\dagger}c_{\beta}$ tries to transfer a fermion
from the state $|\beta\rangle$ to the state $|\alpha\rangle$. However, this transition can be restricted by physical laws. This is the role of the matrix element.
It returns zero, if the operation does not bring $|\alpha\rangle$ into $|\beta\rangle$.\\

The equivalent expression in the wavefunction formalism is
\begin{equation}\label{eq:Translation_TwoStateOverlap}  
    \langle \alpha|\hat{h}|\beta\rangle = \int  \varphi_{\alpha}^{\ast}(x) \hat{h}(x) \varphi_{\beta}(x) \dd x,
\end{equation}
where $x$ still represents the position and the spin of the particle.
$\hat{h}$ is a single particle operator, it means it acts on one particle at a time.
Two states are going to be changed. $|\alpha\rangle$ loses a particle and $|\beta\rangle$ gains one. We say for instance, that the 
configuration before the scattering is $|\Phi\rangle$ and after the scattering is $|\Phi'\rangle$. 
This means, if our two Slater determinant $|\Phi\rangle$ and $|\Phi'\rangle$ differs in more than two states, there are some scattering processes which give zero overlap.\\

In other words, we allow only two states to be modified. Otherwise, the single-particle
states differ and due to their orthogonal properties, we get a zero.\\

Similarly, for the two-particle operator we have
\begin{equation}\label{eq:SecondquantizationDoubleParticle}
    \hat{H}_I = \frac{1}{2}\sum_{i\neq j \in \natset{N}} \hat{v}(x_i, x_j) ~ \leadsto ~~ \frac{1}{2}\sum_{\alpha,\beta,\gamma,\delta} \langle \alpha\beta|\hat{v}|\gamma\delta\rangle c_{\alpha}^{\dagger}c_{\beta}^{\dagger}c_{\gamma}c_{\delta}
\end{equation}
involving a more nested overlap of the four states $\alpha,\beta,\gamma$ and $\delta$:
\begin{equation}\label{eq:Translation_FourStateOverlap}
    \langle\alpha\beta|\hat{v}|\gamma\delta\rangle = \int \int \varphi_{\alpha}^{\ast}(x) \varphi_{\beta}^{\ast}(x') \hat{v}(x,x') \varphi_{\gamma}(x) \varphi_{\delta}(x') \dd x \dd x'.
\end{equation}
This expression modifies four states, so that the overlap of two Slater determinant vanishes, if the determinant differs in at least four states. We kept the hats
to be consistent with the definition of $\hat{H}_0$. The l.h.s of the equation is the matrix element $\langle\Phi_{\alpha}|\hat{v}|\Phi_{\beta}\rangle$ of the operator
 $\hat{v}$, which involves two basis state $|\Phi_{\alpha}\rangle$ and $|\Phi_{\beta}\rangle$.\\
 
What we defined here, is the bridge between the first and the second quantization. In later expressions with proper states we could compute each formalism separately, 
and notice that both lead to the same result. It is important to note that both Eq. \ref{eq:Translation_TwoStateOverlap} and \ref{eq:Translation_FourStateOverlap}
build a scalar.\\

It exists as well a second quantization for bosons, but this is not the aim of this thesis. For additional information, please refer to \cite{Folk2014}.

\subsubsection{Basis transformation}
Until now, we considered the wavefunction in a restricted basis $\{\varphi_{\alpha}(x)\}$. A wavefunction is defined as the overlap between the basis and the state:
\[
    \phi(x) = \langle x | \alpha \rangle.
\]
Let us now define a new, more general operator that creates a state $|\alpha\rangle = a_{\alpha}|0\rangle$ for fermions. (This works as well for bosons).
We assume that we have another basis $\{|\tilde{\alpha}\rangle\}$.
We now want to show, that the new operator $a_{\alpha}$ can be expressed as a linear combination of the other operator  $a_{\tilde{\alpha}}$. This will be a
useful tool, allowing us to jump from a basis to another. We first notice the following identity relation:
\[
    \mathbb{I} = \sum_{\alpha} | \alpha \rangle \langle \alpha| = \sum_{\tilde{\alpha}} | \tilde{\alpha} \rangle \langle \tilde{\alpha}|,
\]
this allows us to write 
\begin{align*}  
    a_{\alpha}^{\dagger}|0\rangle &= | \alpha \rangle =  \sum_{\tilde{\alpha}}\mathcal{I}|\alpha\rangle 
    =\sum_{\tilde{\alpha}}|\tilde{\alpha}\rangle\underbrace{\langle\tilde{\alpha}|\alpha\rangle}_{\text{scalar}} 
    = \sum_{\tilde{\alpha}}\langle\tilde{\alpha}|\alpha\rangle |\tilde{\alpha}\rangle.
\end{align*}
Please note that we write $|\tilde{\alpha}\rangle\langle\tilde{\alpha}||\alpha\rangle = |\tilde{\alpha}\rangle\langle\tilde{\alpha}|\alpha\rangle$
Inverting the indices leads to the same result, which yields to the transformation rules
\begin{align*}
    a_{\alpha}^{\dagger} &= \sum_{\tilde{\alpha}}\langle\tilde{\alpha}|\alpha\rangle a_{\tilde{\alpha}}^{\dagger}\\
    a_{\alpha} &= \sum_{\tilde{\alpha}}\langle\alpha|\tilde{\alpha}\rangle a_{\tilde{\alpha}}.
\end{align*}
Further we can use these relations to express a wavefunction in the basis of another wavefunction. We have
\begin{align*}
    \phi_{\alpha}(x) = \langle x|\alpha\rangle &= \langle x|\left(\sum_{\tilde{\alpha}} \langle \tilde{\alpha}|\alpha\rangle |\tilde{\alpha}\rangle\right) = \sum_{\tilde{\alpha}}  \langle \tilde{\alpha}|\alpha\rangle\langle x|\tilde{\alpha}\rangle
    = \sum_{\tilde{\alpha}} \langle \tilde{\alpha}|\alpha\rangle \varphi_{\tilde{\alpha}}(x).
\end{align*}
Inverting $\alpha$ and $\tilde{\alpha}$ works as well: $\varphi_{\tilde{\alpha}}(x) = \sum_{\alpha} \langle \alpha|\tilde{\alpha}\rangle \phi_{\alpha}(x)$.\\

Moreover, we can show that the basis transformation is unitary. This is an important feature because we can simplify the calculation by changing the basis, and
the result would remain the same if we use unitary transformations. Such transformations play a major role later in the thesis.
We can save the $\langle \tilde{\alpha} | \alpha\rangle$ in a matrix element $D_{\tilde{\alpha}\alpha}$, and prove that the matrix $D$ is unitary. Here we need
$\langle \gamma | \beta\rangle = \langle \beta | \gamma\rangle$.
\begin{align*}
    \langle \tilde{\alpha} | \tilde{\beta}\rangle  &= \sum_{\gamma} \langle \tilde{\alpha}|\gamma\rangle\langle \gamma | \beta\rangle =  \sum_{\gamma} \langle \tilde{\alpha}|\gamma\rangle \langle \beta | \gamma\rangle^{\ast}\\
    &=\sum_{\gamma} D_{\alpha\gamma}D_{\beta\gamma}^{\ast} = \sum_{\gamma} D_{\alpha\gamma}D^{\dagger}_{\gamma\beta} = (DD^{\dagger})_{\alpha\beta},
\end{align*} 
where $\langle \tilde{\alpha} | \tilde{\beta}\rangle = \delta_{\tilde{\alpha} \tilde{\beta}}$ such that $DD^{\dagger} = \mathbb{I}$. The matrix $D$ is therefore unitary.\\

The last important step is to show that the basis transformation keeps the anti-, commutation relations. Let us for the seek of readability
use the notation $[A,B]_{\xi} = AB + \xi BA$ where $\xi = -1$ for bosons and $\xi = +1$ for fermions. We have for example, using $[a_{\alpha}, a_{\alpha'}^{\dagger}]_{\xi} = \delta_{\alpha\alpha'}$
\[
    [a_{\tilde{\alpha}}, a_{\tilde{\alpha}'}^{\dagger}]_{\xi} = \sum_{\tilde{\alpha}\tilde{\alpha}'} \langle \tilde{\alpha}|\alpha\rangle\langle\alpha'|\tilde{\alpha}'\rangle [a_{\alpha}, a_{\alpha'}^{\dagger}]_{\xi} = \langle \tilde{\alpha}|\tilde{\alpha}'\rangle = \delta_{\tilde{\alpha}\tilde{\alpha}'}.
\]
The first step follows after splitting the commutator in two parts, insert the transformation and recombine the new commutator. 
The last step involves the orthonormality of the basis.
In the same way one can prove $[a_{\tilde{\alpha}},a_{\tilde{\alpha}'}] = [a_{\tilde{\alpha}}^{\dagger},a_{\tilde{\alpha}'}^{\dagger}] = 0$.
  
\subsubsection{Field operators}
Later in this thesis we are going to use field operators to describe an order parameter of a superconductive system. These are 
creation and annihilation operators that are defined in the $|x\rangle$-space-spin basis regarding another basis $\{|\alpha\rangle\}$. We here give the state as an argument
and not as an index any more. Despite the notation, the following operators mustn't be confused with a wavefunction.
\begin{align}
    \hat{\psi}^{\dagger}(x) &= \sum_{\alpha} \langle\alpha|x\rangle a_{\alpha}^{\dagger} = \sum_{\alpha} \varphi_{\alpha}^{\ast}(x) a_{\alpha}^{\dagger} \label{eq:FieldOp}\\ 
    \hat{\psi}(x) &= \sum_{\alpha} \langle\alpha|x\rangle a_{\alpha} = \sum_{\alpha} \varphi_{\alpha}(x) a_{\alpha}^{\dagger}\label{eq:FieldOpDag}
\end{align}
involving fermionic or bosonic operators $a$. We can then annihilate and create a particle at a spin-space location $x$. Because the $a$ operators
are involved, the commutations properties should be respected. Using the result we had for $[a_{\alpha}, a_{\alpha'}]_{\xi},
[a_{\alpha}, a_{\alpha'}^{\dagger}]_{\xi}$ and $[a_{\alpha}^{\dagger}, a_{\alpha'}^{\dagger}]_{\xi}$ where $\xi = -$ for the bosons and $+$ for the fermions. We find the following commutation relations:
\begin{align*}
    \left[\hat{\psi}(x), \hat{\psi}(x')\right]_{\xi} = & \left[\hat{\psi}(x)^{\dagger}, \hat{\psi}^{\dagger}(x)\right]_{\xi} = 0 \\
    \left[\hat{\psi}(x), \hat{\psi}^{\dagger}(x')\right]_{\xi} = &~ \delta(x-x')
\end{align*}
In the last expression we obtain a $\langle x | x '\rangle$ which can not be normalized the $\{|x\rangle\}$-basis. Instead of a Kronecker delta $\delta_{xx'}$
we get a delta distribution $\delta(x-x')$. This can be justified because the wavefunction is normalized and can be ``seen'' as well as a distribution.
The goal is now to describe the Hamiltonian using this fields operators. Therefore, we rebuild these operators in the Hamiltonian using a $\{|x\rangle\}$ basis
\begin{align*}
    \hat{H}_0 &=  \int \hat{\psi}^{\dagger}(x) \hat{h}(x)\hat{\psi}(x) \dd x\\
    \hat{H}_I &= \frac{1}{2} \int \int  \hat{\psi}^{\dagger}(x) \hat{\psi}^{\dagger}(x') \hat{v}(x,x')\hat{\psi}(x') \hat{\psi}(x) \dd x' \dd x
\end{align*}

This paragraph closes the discussion of the second quantization. We introduced the creation and annihilation operators and showed how to build a 
second quantized version of the Hamiltonian. The goal is now to apply this formalism on a system of electrons that is going to be relevant in a later part of this work.


\subsection{Case of study: electron gas with interactions}
As we are later going to study, the electrons are allowed to interact with each other. The most simple interacting system involving electrons is the electron-gas.
During these processes virtual photons mediate interactions between electrons. we are going to use the formalism we introduced to 
find a second quantization representation of the interacting Hamiltonian.\\

The system we are studying is a cube of side length $L$ with volume $\Omega = L^3$ containing $N$ electrons. Further we consider a periodic boundary condition for an arbitrary 
position vector $\bm{r} = (x,y,z)$:
\[
    (L,y,z) = (0,y,z)~,~~(x,L,z) = (x,0,z)~,~~(x,y,L) = (x,y,0).
\]  
We use the general form of the Hamiltonian introduced in the beginning under the Eq. \ref{eq:MasterHamiltonian}. We consider a Coulomb potential and a kinetic energy term.
\begin{align}
    \hat{H}_0 &= \hat{T}+\hat{U} =- \sum_{i\in \natset{N}} \frac{\hbar^2}{2m}\nabla_i^2 + \sum_{i\in \natset{N}} \hat{u}(x_i)\\
    \hat{H}_I &= \hat{V}= \sum_{i\neq j \in \natset{N}} \hat{v}(x_i, x_j),
\end{align} 
where the pairwise potential is just a Coulomb potential $\hat{v}(x_i, x_j) = \frac{e^2}{4\pi\epsilon_0|x_i - x_j|} = v(|x_i - x_j|)$.\\
We aim to describe the Hamiltonian in the momentum space, which is more convenient for the second quantization. Therefore, we first
need to find an expression for the wave vector $\bm{k}$. We describe a state $\alpha = (\bm{k}, \sigma)$ at a spin-space coordinate $x = (\bm{r}, s)$.
 First we take a plane wave solution of the Schrödinger equation:
\[
    \psi_{\bm{k}}(\bm{r}) = \frac{1}{\sqrt{\Omega}} e^{\im \bm{k}\cdot\bm{r}} \chi_\sigma(s)
\]
The periodicity of the system allows us to write for each dimension $\psi(x=0) = \psi(x=L)$. Using these boundary conditions we obtain 
$1= e^{\im k_x L}$ in all directions. The result reflects itself in the wave vector $\bm{k}$, which becomes quantized:
\[
    \bm{k} = \frac{2\pi}{L}(n_x, n_y, n_z)~,~~n_i\in \mathbb{Z}.
\]
As we saw earlier, the eigenfunctions build a complete basis:
\begin{align*}
    \int \psi^{\ast}_{\alpha'}(x) \psi_{\alpha}(x) \dd x &= \int \sum_{s} \psi^{\ast}_{\alpha'}(x) \psi_{\alpha}(x) \dd^3r \\
    &=\frac{1}{\Omega} \int e^{\im\bm{r}\left(\bm{k}- \bm{k}'\right)} \underbrace{\sum_s \delta_{s\sigma}\delta_{s\sigma'}}_{\delta_{\sigma\sigma'}} \dd^3 r\\
    &= \delta_{\bm{k}\bm{k}'}\delta_{\sigma\sigma'} = \delta_{\alpha\alpha'}.
\end{align*}    
The integral over the exponential function diverges if $\bm{k} \neq \bm{k}'$, so we use the case $\bm{k} = \bm{k}'$ and set the integral to zero otherwise.
The kinetic energy operator is a single-particle operator. We have according to Eq. \ref{eq:SecondquantizationSingleParticle}:
\[
    T = \sum_{\alpha\alpha'} \langle \alpha|\frac{\hat{\bm{p}}^2}{2m}|\alpha'\rangle c_{\alpha}^{\dagger}c_{\alpha'},
\]
with $\hat{\bm{p}}$ the impulse operator in all space directions: $\hat{\bm{p}} = -\im \hbar \nabla$. To compute this expression we use its 
first quantized form (see Eq. \ref{eq:Translation_TwoStateOverlap}) involving the $\delta_{\sigma\sigma'}$ trick we introduced
in the derivation of the complete basis. We also use  $\bm{k}' = \bm{k}$ and hide the $\frac{1}{\Omega}$ is the $\delta_{\bm{k}\bm{k}'}$. The result reads
\begin{equation}\label{eq:interactingGasT}
    T = \sum_{\alpha,\alpha'} \delta_{\alpha\alpha'} \frac{\hbar \bm{k}^2}{2m} c_{\alpha'}^{\dagger} c_{\alpha} 
    = \sum_{\alpha} \underbrace{\frac{\hbar \bm{k}^2}{2m}}_{=:\epsilon_{\bm{k}}} c_{\alpha}^{\dagger} c_{\alpha}.
\end{equation}
We recognize the occupation number operator $\hat{n}_{\alpha} = c_{\alpha}^{\dagger} c_{\alpha}$ and the energy of the state $\epsilon_{\bm{k}}$. This variable plays a central role later.
We obtain a quite meaningful result: The non-interacting energy part of the system is the product of the energy of a 
state with the number of particle within that state, summed over all states.
The single-particle operator engage a $\hat{u}$-term as well. This can be used by a single-particle potential such as an external electric field 
in the case of the electrons. We will ignore this term.\\

For the interaction potential we have a two-particles operator. This is described by Eq. \ref{eq:SecondquantizationDoubleParticle} and
requires Eq. \ref{eq:Translation_FourStateOverlap} to be solved:
\[
    \hat{V} = \frac{1}{2} \sum_{\alpha \beta \gamma \delta} \langle \alpha \beta |\hat{v}|\gamma\delta\rangle 
    c_{\alpha}^{\dagger}c_{\beta}^{\dagger}c_{\gamma}c_{\delta}.
\]

We can first work on the matrix element using $v(\bm{r}, \bm{r'})= v(\bm{r} - \bm{r'})$:
\[
    \langle \alpha \beta |\hat{v}|\gamma\delta\rangle = \frac{1}{\Omega^2} \delta_{\sigma_{\alpha}\sigma_{\gamma}}\delta_{\sigma_{\beta}\sigma_{\delta}} \int \int  e^{\im \bm{r}(\bm{k}_{\gamma}-\bm{k}_{\alpha})}  e^{\im \bm{r}'(\bm{k}_{\delta}-\bm{k}_{\beta})} v(\bm{r} - \bm{r'}) \dd \bm{r'} \dd \bm{r}.
\]
After making a substitution $\bm{R} = \bm{r} - \bm{r}'$, adding and subtracting a $(\bm{k}_{\gamma}-\bm{k}_{\alpha})\bm{r}'$, we obtain:
\[
    \langle \alpha \beta |v|\gamma\delta\rangle = \frac{1}{\Omega^2} \delta_{\sigma_{\alpha}\sigma_{\gamma}}\delta_{\sigma_{\beta}\sigma_{\delta}}
     \underbrace{\int e^{-\im\bm{R}(\bm{k}_{\alpha}- \bm{k}_{\delta})} v(\bm{R}) \dd \bm{R}}_{v_{\bm{k}_{\delta}- \bm{k}_{\alpha}}}
    \underbrace{\int e^{\im\bm{r}'\left(\bm{k}_{\gamma}-\bm{k}_{\beta} + \bm{k}_{\delta} -\bm{k}_{\alpha}\right)}\dd \bm{r}'}_{= \delta_{\bm{k}_{\gamma}-\bm{k}_{\beta} + \bm{k}_{\delta} , \bm{k}_{\alpha}}}.
\]
Here we see that the first underbraced term is the Fourier transform of a Coulomb potential $v$ and the second term is a Kronecker delta.
From this we derive a cumbersome equation
\[
    \hat{V} = \frac{1}{2\Omega} \sum_{\substack{\bm{k}_{\alpha}\bm{k}_{\beta}\bm{k}_{\gamma}\bm{k}_{\delta} \\
         \sigma_{\alpha}\sigma_{\beta}\sigma_{\gamma}\sigma_{\delta}}}
    \delta_{\sigma_{\alpha}\sigma_{\gamma}}\delta_{\sigma_{\beta}\sigma_{\delta}}
    \delta_{\sigma_{\alpha}\sigma_{\gamma}}\delta_{\sigma_{\beta}\sigma_{\delta}}
    \delta_{ \bm{k}_{\alpha}, \bm{k}_{\gamma}-\bm{k}_{\beta} + \bm{k}_{\delta}}
    v_{\bm{k}_{\alpha}- \bm{k}_{\delta}} c_{\bm{k}_\alpha\sigma_{\alpha}}^{\dagger}c_{\bm{k}_\beta\sigma_{\beta}}^{\dagger}c_{\bm{k}_\gamma\sigma_{\gamma}}c_{\bm{k}_\delta\sigma_{\delta}}.
\]
We can sum over the $\bm{k}_{\alpha}$, rename $\sigma_{\alpha} \rightarrow\sigma$ and $\sigma_{\beta} \rightarrow\sigma'$
 and sum up over $\sigma_{\gamma}$ and $\sigma_{\delta}$ to simplify the Kronecker deltas.
\[
    \hat{V} = \frac{1}{2\Omega} \sum{\sigma\sigma'} \sum_{\bm{k}_{\beta}\bm{k}_{\gamma}\bm{k}_{\delta}} v_{\bm{k}_{\gamma}- \bm{k}_{\beta}} c^{\dagger}_{\bm{k}_{\delta}+ \bm{k}_{\gamma}-\bm{k}_{\beta},\sigma}
        c^{\dagger}_{\beta\sigma'} c_{\gamma\sigma'} c_{\delta\sigma'} ,
\] 
where using $\bm{k}_{\alpha} = \bm{k}_{\gamma}-\bm{k}_{\beta} + \bm{k}_{\delta}$ from the Kronecker delta we get $v_{\bm{k}_{\alpha}- \bm{k}_{\delta}} = v_{\bm{k}_{\gamma}-\bm{k}_{\beta} + \cancel{\bm{k}_{\delta}-\bm{k}_{\delta}}}$.
Then we can introduce the following variable transformations:
\[
    \bm{k}_{\delta} \rightarrow \bm{k} ~,~~~ \bm{k}_{\gamma} \rightarrow \bm{k}' ~,~~~ \bm{k}_{\beta} \rightarrow \bm{k}' - \bm{q},
\]
which yields:
\begin{align*}
    \bm{k}_{\delta} + \bm{k}_{\gamma} - \bm{k}_{\beta} &= \bm{k}+ \bm{q},\\
    \bm{k}_{\gamma} - \bm{k}_{\beta} &=  -\bm{q},
\end{align*}
and we finally get our second quantized interaction operator:
\[
    \hat{V} = \frac{1}{2\Omega} \sum_{\bm{q}}v_{q} \sum_{\substack{\bm{k}\sigma\\\bm{k}'\sigma'}} c^{\dagger}_{\bm{k}+\bm{q},\sigma}c^{\dagger}_{\bm{k}'-\bm{q},\sigma'}c_{\bm{k}'\sigma'}c_{\bm{k}\sigma}.
\]
This describes an electron transferring a momentum $\bm{q}$ to another electron. The formula tells that we first destroy both electrons as they are before the interaction.
Then we create two in the new states, where one lost some momentum that has been transferred the other electron. Here the $v$-term modulates the strength of the process. 
The Coulomb force is known to decline with the inverse of the distance $\bm{r}-\bm{r}'$
between the electrons, but do not depend on their respective positions $\bm{r}$ and $\bm{r}'$. This exchange is therefore invariant under translations.
The following Feynman-diagram is a good illustration of this process.\\
\begin{figure}[H]
    \centering
    \begin{tikzpicture}
        \begin{feynman}
          \vertex (a) at (3,0);
          \vertex (b) at (0,0);
          \vertex (c) at (-1,2){\(\bm{k}'-\bm{q},\sigma'\)};
          \vertex (d) at (-1,-2){\(\bm{k}',\sigma'\)};*
          \vertex (c2) at (4,2){\(\bm{k}+\bm{q},\sigma\)};
          \vertex (d2) at (4,-2){\(\bm{k},\sigma\)};
          \diagram* {
            (d) -- [fermion] (b) -- [fermion] (c),
            (d2) -- [fermion] (a) -- [fermion] (c2),
            (b)-- [photon, momentum'=\(\bm{q}\)] (a)
        };
        \end{feynman}
    \end{tikzpicture}

    \caption{The interaction of two electrons modulated by a photon of momentum $\bm{q}$. The leftmost starts in the state  $(\bm{k}', \sigma')$ and gives 
    a momentum $\bm{q}$ to the rightmost electron. The time is represented on the horizontal axis and the space on the vertical one.}
    \label{fig:photon_exchange}
\end{figure}
This closes the introduction theory on the second quantization. We saw how we can express integrals over wave functions as braket scalar products. 
We introduce two operators that create and annihilate particles in a certain state. Using this formalism we were able to compute the 
Hamiltonian of the system in the momentum space. We found that the non-interacting-part relies on the energy of each state times their occupation number.
In the last part we showed that an interaction between two electrons can be described as a momentum transfer between them,
 which is modulated by the Fourier transform of the Coulomb potential.\\

The goal of this thesis is to study how the superconductivity behaves in the proximity of an altermagnet. 
Over the next chapter we are going to explain how a superconductive state works. The altermagnet will be presented in a later part of this work.\\

%------------------------%
\section{Superconductivity}
Superconductivity can be illustrated as a phase transition of a metal when its temperature lies under a crystal temperature $T_C$. In the superconductive state the material 
becomes a perfect diamagnet and its resistivity vanishes. If we apply a magnetic field on superconductive metal, we will observe some shielding currents that arise on its surface.
They are labelled as Meissner currents. We have as well an internal magnetic field, that cancels exactly the one
that is applied outside. The magnetic susceptibility in the body is then $\chi = -1$.\\
A second property that found multiple technical use cases is the friction-less flow of a current in the metal. The current can flow for a very long time without losing energy. 
In the first chapter of \cite{FossheimSudbo2004} Fossheim and Sudbø even calculated a significant decay of this current that exceeds the age of the universe.
The superconductive state is also described as a Meissner state.\\ 

Suppose that we now heat the material at the critical temperature $T_C$, some fluctuation effects appear and break the superconductive state.
We usually distinguish type I and type II superconductors. The type I superconductors
lose abruptly their magnetization over $T_C$. Type II have a mix of ordinary and superconductive properties at intermediate temperatures.
 In this mixed states the magnetization slowly decreases while heating the metal,
 until we can not measure any Meissner state any more over $T_C$. Above an intermediate field strength, we observe as well some vortices
  and flux lines that pierce the material.\\
 \begin{figure}[H]\centering

    \begin{tikzpicture}
        \begin{axis}[
            axis lines=middle,
            width=7cm,   % Adjust the width
            height=5cm,  % Adjust the height
            xtick={3},
            minor y tick num = 0,
            ytick={0.5,1},
            ymin=0, ymax=1.3,
            xmin=2, xmax=3.5,
            xlabel = \(T\),
            ylabel = {\(H\)},
            legend style={draw=none},
            unbounded coords = jump,
            yticklabels = {$H_{C1}$,$H_{C2}$},
            xticklabels = {$T_{C}$},
            every axis/.append style={line width=1pt}
        ]
    
            % The green parabola
            \addplot [
                domain=0:3.5, 
                samples=4000, 
                color=TamGreen,
            ]
            {sqrt(-x+3)};
    
            % The black parabola
            \addplot [
                domain=0:3.5,
                samples=4000,
                color=black,
                thick,
            ]
            {sqrt(-x+3)/2};
    
            % Add a node for labelling
            \node at (axis cs: 2.25, 0.175) [black] {$S_{\text{Me}}$};
            \node at (axis cs: 2.4, 0.55) [TamGreen] {$S_{\text{Mi}}$};
            \node at (axis cs: 2.8, 0.8) [black] {$N$};

        \end{axis}
    \end{tikzpicture}
    
        \caption{The phase diagram of a superconductive material regarding the temperature and the external applied magnetic field strength. Type I superconductor can only be in 
        the Meissner state $S_{\text{Me}}$ or in the normal state $N$. Type II superconductor additionally have mixed state $S_{\text{Mi}}$ between the two. The critical values depend on the material.
        Increasing the field strength lowers the critical temperature. Inspired by \cite{FossheimSudbo2004}.}
    \end{figure}
 
    \begin{figure}[H]\centering

        \begin{tikzpicture}
            \begin{axis}[
                axis lines=middle,
                width=5cm,   % Adjust the width
                height=5cm,  % Adjust the height
                xtick={3},
                minor y tick num = 0,
                ytick=\empty,
                ymin=0, ymax=4,
                xmin=0, xmax=4,
                xlabel = \(H\),
                ylabel = {\(-M\)},
                legend style={draw=none},
                unbounded coords = jump,
                xticklabels = {$H_{C}$},
                every axis/.append style={line width=1pt}
            ]
        
                % The green parabola
                \addplot [
                    domain=0:3.5, 
                    samples=100, 
                    color=black,
                ]
                {x < 3 ? x : 0};
        
                % The black parabola
                \addplot [
                    domain=1.5:3,
                    samples=100,
                    color=TamLightGreen,
                    thick,
                ]
                {-x + 3};

                \node at (axis cs: 2.5, 1.5) [black] {$S_{\text{Me}}$};
                \node at (axis cs: 2, 0.3) [TamLightGreen] {$S_{\text{In}}$};
                \node at (axis cs: 3.5, 2) [black] {$N$};
                \draw[-, line width=.5pt, line cap=round, dash pattern=on 0pt off 2\pgflinewidth] (axis cs: 1.5,0) -- (axis cs: 1.5,1.5);
            \end{axis}
        \end{tikzpicture}
        \hspace{1cm}
        \begin{tikzpicture}
            \begin{axis}[
                axis lines=middle,
                width=7cm,   % Adjust the width
                height=5cm,  % Adjust the height
                xtick={3,6},
                minor y tick num = 0,
                ytick=\empty,
                ymin=0, ymax=4,
                xmin=0, xmax=7,
                xlabel = \(H\),
                ylabel = {\(-M\)},
                legend style={draw=none},
                unbounded coords = jump,
                xticklabels = {$H_{C1}$,$H_{C2}$},
                every axis/.append style={line width=1pt}
            ]
        
                % The green parabola
                \addplot [
                    domain=0:3, 
                    samples=100, 
                    color=black,
                ]
                {x};
        
                % The black parabola
                \addplot [
                    domain=3:6,
                    samples=100,
                    color=TamGreen,
                    thick,
                ]
                {exp(-2.1*x+7.4)};
        
                % Add a node for labelling
                \node at (axis cs: 5, 1) [black] {$N$};
                \node at (axis cs: 3.5, 0.3) [TamGreen] {$S_{\text{Mi}}$};
                \node at (axis cs: 2, 1) [black] {$S_{\text{Me}}$};
                \draw[-, line width=.5pt, line cap=round, dash pattern=on 0pt off 2\pgflinewidth] (axis cs: 3,0) -- (axis cs: 3,3);
            \end{axis}
        \end{tikzpicture}
        
            \caption{The left plot shows how the magnetization of a type I grows with the external field strength to cancel the applied field inside the material. Above
            the critical field strength $H_C$ the material becomes a normal metal and the field flows through it. If the material has a non-zero demagnetization factor,
            the field starts to penetrate the body at an intermediate field strength, leading to an intermediate (not mixed!) state $S_{\text{In}}$.
             The field penetrates in form of
            flux lines, along the normal to the surface. On the other hand we have a figure
            that after showing the cancellation of the field inside the body, highlights the presence of a mixed state in a type II superconductor. The field penetrates 
            the material in the form of vortices, and finally let all the external field stream in the material.
            Inspired by \cite{FossheimSudbo2004}.}
        \end{figure}
During the suppression of the superconductive state more and more magnetic field 
is permitted inside the material. Assuming that some particles are responsible for the superconductivity, the field will penetrate more easily where we observe
a lower density of these particles.\\

The Meissner state is a thermodynamic state. This means that we can completely identify it with a set of variable that we are going to derive.
We can show that the free energy of the superconductive state is lower than the normal state. 
This results in a lower entropy compared to the normal state (\cite{FossheimSudbo2004} chap. 1.6).
Along with the derivation of the superconductivity, we are going to follow closely the work of Fossheim and Sudbø \cite{FossheimSudbo2004} from chapter 2 to 4.
For readability reason we set the reduced Planck constant $\hbar = 1$ in the following. Further they added the chemical potential to the energy of the state such
that now $\epsilon_k - \mu \rightarrow \epsilon_k $, which was originally defined at Eq. \ref{eq:interactingGasT}.\\

The Hamiltonian of the system is described by the solid state physics. We consider the energy of the electrons and the ions in a lattice and the interaction between them resulting in 
the following, very generic expression of the Hamiltonian:
\[
    H = H_{e^-e^-} + H_{e^-\text{ion}} + H_{\text{ion ion}}.
\]
Each term consist of a kinetic and potential energy term. For a more mathematical approach we consider a system of $N$ electrons and $L$ ions and define:
\begin{align*}
    H_{e^-e^-} &= \sum_{i \in \natset{N}} \frac{p_i^2}{2m} + \sum_{ij\in\natset{N}} V_{\text{Coulomb}}^{e^-e^-}(\bm{r}_i-\bm{r}_j), \\
    H_{\text{ion ion}} &=  \sum_{i \in \natset{M}} \frac{p_i^2}{2M} + \sum_{ij\in\natset{L}} V_{\text{Coulomb}}^{\text{ion-ion}}(\bm{R}_i-\bm{R}_j), \\
    H_{e^-\text{ion}} &= \sum_{i\in \natset{N},j\in \natset{L}} V_{\text{Coulomb}}^{e^-\text{ion}}(\bm{r}_i-\bm{R}_j).
\end{align*}
We have $m$ and $M$ as the mass of the electron and the ion. $\bm{r}$ and $\bm{R}$ are the position of the electron and the ion. The ion-ion potential
freezes the ions into the lattice. 
First we are going to introduce some concepts by describing a non-interacting electron and then improve it to include the interactions.
Usually the potential of a lattice is periodic. If so, according to the Bloch theorem, the wave function of a particle moving in the system is a plane wave modulated
by a periodic function. Eigenstates of the corresponding Hamiltonian are called Bloch states.

\subsection{The non-interacting electron gas}
In this case of study the Hamiltonian only includes a kinetic term
\begin{equation}\label{eq:NonInteractingHamiltonian}
    H=\sum_{\bm{k},\sigma} \epsilon_{\bm{k}} c_{\bm{k}\sigma}^{\dagger}c_{\bm{k}\sigma}.
\end{equation}
We assume that there exists a ground state $|0\rangle$, where the system is filled up with a certain amount of electron until the Fermi-energy $\epsilon_F$ is reached. Associated with this
energy we find a wave vector $\bm{k}_F$, the Fermi-momentum. The set of energy up to $\epsilon_F$ is called the Fermi-sea, as an analogy to the level zero of the topographic maps. 
Put into a mathematical form, the Fermi-sea is defined as:
\begin{equation}
    \hat{n}_{\bm{k},\sigma} |0\rangle = \Theta(\epsilon_F - \epsilon_{\bm{k}})|0\rangle.\label{eq:MomentumDistributionFree}
\end{equation}
We introduced here a very useful tool called the Heaviside step function which is defined as: 
\begin{equation}\label{eq:Heaviside}
    \Theta(x) = \begin{cases}
        1, ~x > 0\\
        0, ~x < 0
    \end{cases} , ~~~ \Theta(-x) = 1 - \Theta(x).
\end{equation}
This means that we find no particles that have an energy higher than the Fermi-energy ($\bm{k}>\bm{k}_F$).\\

Now we want to study how the electron can scatter in different states. The function that we are using is called the propagator, and
gives the probability to find the particle in the state $|\bm{k}',\sigma\rangle$ at a time $t'$ knowing it at $|\bm{k},\sigma\rangle$ and $t$. 
An important fact is that without interaction, the particle shouldn't scatter in another state due to energy conservation. Therefore,
\[
    G_0(\bm{k}, \bm{k}', t'-t) = G_0(\bm{k}, t'-t) \delta_{\bm{k}, \bm{k}'},
\]
which is zero if the wave-vectors between the two time points differs. 
We observe that only the pastime $t'-t$ is relevant. This is due to the time independence of the Hamiltonian.
We are going to use the representation in the frequency space, using a Fourier-transformation.
\begin{equation}
    G_0(\bm{k},\omega) = \int_{\mathbb{R}} e^{\im\omega t} G_0(\bm{k},t)\dd t = \frac{1}{\omega - \epsilon_{\bm{k}} + \im \delta_{\bm{k}}} \label{eq:single_propa}
\end{equation}
where $\delta_{\bm{k}} = \delta \cdot\signum{\epsilon_{\bm{k}} - \epsilon_{F}}$ involving a very small, non-zero number $\delta$. We observe that
this analytical function has a pole given by:
\begin{align*}
    &\omega - \epsilon_{\bm{k}} + \im \delta_{\bm{k}} = 0\\
    \Longleftrightarrow ~& \omega = \epsilon_{\bm{k}} - \im \delta_{\bm{k}},
\end{align*}
where we denote $\im$ as the imaginary unit to avoid confusion with the index $i$.
The variable $\epsilon_{\bm{k}}$ gives the so-called spectrum or dispersion relation of the excitations.  Please remember that we set $\hbar =1$. The imaginary part serves as a damping term and is 
inversely proportional to the lifetime of the particle. As a result of the absence of scattering, $\delta$ is a small number due to the infinitely long lifetime of the electrons.\\

Further the propagator yields important information on the system, when considering the integration over its different arguments. First we take the imaginary 
part of the propagator, called the single particle spectral weight $A(\bm{k}, \omega)$.
\begin{equation}
    \begin{aligned}\label{eq:SingleParticleSpectralWeight}
        A(\bm{k}, \omega) = -\frac{1}{\pi} \mathcal{I}m \left[G_0(\bm{k}, \omega)\right] =~& \frac{1}{\pi} \frac{\delta_{\bm{k}}}{(\omega- \epsilon_{\bm{k}})^2 + \delta_{\bm{k}}}\\
        =~&\delta(\omega- \epsilon_{\bm{k}})
    \end{aligned}
\end{equation}
which informs us about possible available state which may be occupied. We can find a form for the momentum distribution $n(\bm{k})$:
\begin{equation}
    n(\bm{k}) = \int A(\bm{k},\omega) \dd \omega,\label{eq:MomentumDistribution}
\end{equation}
and for the density of state:
\begin{equation}\label{eq:DensityOfState}
    D(\omega) = \int A(\bm{k},\omega) \dd ^3 k~,~~ \text{or for discontinuous state}~ \sum_{\bm{k}} A(\bm{k}, \omega).
\end{equation}

These equations picture the non-interacting electron gas. Now that we put the ground stones, we can make the model more advanced.

\subsection{Fermi-liquid: The interacting case}
Now that we have described the non-interacting system, let us make the model more complex by introducing the interactions.
In an earlier section we saw how
\begin{equation} \label{eq:FermiLiquid_Hamiltonian}
    H = \sum_{\bm{k}\sigma} \epsilon_{\bm{k}} c_{\bm{k},\sigma}^{\dagger}c_{\bm{k},\sigma} + \sum_{\bm{k}\sigma\bm{k}'\sigma'}
        V_{\bm{k}\bm{k}', \bm{q}} c_{\bm{k}-\bm{q},\sigma}^{\dagger}c_{\bm{k}+\bm{q},\sigma'}^{\dagger}c_{\bm{k},\sigma}c_{\bm{k}',\sigma'}
\end{equation}
represent the pairwise interaction of multiple electrons and their respective energy.\\

To extend the model we now want to introduce two new quantities,
the propagator $G$ and the one-particle irreducible self-energy $\Sigma$.
The propagator $G: \mathbb{R}^3\times\mathbb{R}\mapsto \mathbb{C}$ gives the probability amplitude of finding a particle in the state $|\bm{k},\sigma\rangle$ at a time $t$. On the other hand 
for $\Sigma_R, \Sigma_I \in \mathbb{R}$ and $\Sigma\in\mathbb{C}$, we define the complex self energy $\Sigma = \Sigma_R + \im \Sigma_I$. This variable contains 
the lifetime of the particle in this state, and the shift of the particle's energy due to the interaction with the surroundings. The framework defines the 
non-interacting energy of the particle as $\epsilon_{\bm{k}}$. When put in an interacting system, the spectrum shifts and becomes 
$\tilde{\epsilon}_{\bm{k}} =  \epsilon_{\bm{k}} + \Sigma_R$. Due to the interactions, the particle then has a much smaller lifetime. $\Sigma_I$ is inversely 
proportional to the particle's lifetime $\tau_{\bm{k}}$. As said, $\Sigma_I$ is very small in the non-interacting case. 
These two quantities are linked through the Dyson equation, which reads
\begin{equation} \label{eq:Dyson}
    \bigl(G(\bm{k}, \omega)\bigr)^{-1} =  \bigl(G_0(\bm{k}, \omega)\bigr)^{-1} - \Sigma(\bm{k},\omega).
\end{equation}

One can use a Fourier-transformation to switch from the time representation to the frequency representation $\omega$. Reordering the equation and using the result
from Eq. \ref{eq:single_propa} we obtain
\[
    G(\bm{k},\omega) = \frac{1}{\omega - \epsilon_{\bm{k}} - \Sigma}.
\]
This function has a pole at $\omega = \epsilon_{\bm{k}} + \Sigma_R + \im \Sigma_I$, where in the none interacting case $\omega = \epsilon_{\bm{k}} + \im\delta_{\bm{k}}$.
This expression makes sense, the particle spectrum is now shifted due to its finite lifetime, as a result of the interaction.
The pole yields to an expression for the spectrum
\begin{equation} \label{eq:pole_spectrum1}
    \omega - \epsilon_{\bm{k}} - \left(\Sigma_R(\bm{k},\omega) + \im \Sigma_I(\bm{k},\omega)\right) = 0
\end{equation}
In complex analysis the order of a pole is given as $n$ if $f(z)$ is meromorphic and has a pole at $z_0$ where
\[
    (z-z_0)^n f(z)
\]
is also meromorphic in the neighbourhood of $z_0$. In our case we are interested in the 0-th order of the pole and first ignore the imaginary part of the pole.
What we get
is an expression for the energy shift:
\[
    \omega = \epsilon_{\bm{k}} + \Sigma_R(\bm{k}, \epsilon_{\bm{k}}) = \tilde{\epsilon}_{\bm{k}}.  
\]
If we take into account a tiny imaginary part of $\Sigma$ we obtain a shifted pole. Performing a Taylor expansion of 
$\Sigma_{R}$ in the neighbourhood of $\omega = \tilde{\epsilon}_{\bm{k}}$ will help us.
\begin{equation*}
    \left.\Sigma_R(\bm{k}, \omega) = \Sigma_R(\bm{k}, \tilde{\epsilon}_{\bm{k}}) + (\omega - \tilde{\epsilon}_{\bm{k}}) \parDif{\Sigma_R}{\omega}\right|_{\omega= \tilde{\epsilon}_{\bm{k}}} + \mathcal{O}(\omega^2)
\end{equation*}
We aim to compute to the first order in $\Sigma_I$ such that we evaluate it at $\omega = \tilde{\epsilon}_{\bm{k}}$. 
Starting from Eq. \ref{eq:pole_spectrum1}, and after inserting the Taylor expansion for $\Sigma_R$, we obtain
\begin{equation}
    \begin{aligned}
        \omega - \underbrace{\bigl(\epsilon_{\bm{k}} + \Sigma_R(\bm{k}, \tilde{\epsilon}_{\bm{k}})\bigr)}_{=  \tilde{\epsilon}_{\bm{k}}} 
        - \left.(\omega - \tilde{\epsilon}_{\bm{k}}) \parDif{\Sigma_R}{\omega}\right|_{\omega= \tilde{\epsilon}_{\bm{k}}} - \im \Sigma_I(\bm{k}, \tilde{\epsilon}_{\bm{k}}) &= 0\\
        \Longleftrightarrow ~~ (\omega - \tilde{\epsilon}_{\bm{k}}) \left(1 - \left.\parDif{\Sigma_R}{\omega}\right|_{\omega= \tilde{\epsilon}_{\bm{k}}}\right) - \im \Sigma_I(\bm{k}, \tilde{\epsilon}_{\bm{k}}) &= 0.
    \end{aligned}
\end{equation}
We define the residue of the propagator as 
\begin{equation}
    z_{\bm{k}} = \frac{1}{1- \left.\parDif{\Sigma}{\omega}\right|_{\omega = \tilde{\epsilon}_{\bm{k}}}}, \label{eq:residue}
\end{equation}
which we can insert in the inverse lifetime of the electron occupying the state $|\bm{k},\sigma\rangle$
\begin{equation*}
    \frac{1}{\tau_{\bm{k}}} = - z_{\bm{k}} \Sigma_I(\bm{k}, \tilde{\epsilon}_{\bm{k}}).
\end{equation*}
This justifies the statement $ \Sigma_I \propto 1/\tau_{\bm{k}}$
This residue is a decreasing function of the energy, which means that its influence is more important for low energies. 
An interpretation could be that the slow moving electrons have less time to interact with their fast homologues, which results in a
longer lifetime.\\

We recall once again the difference of the propagators to conclude this section:\vspace{12pt}\\
\begin{center}
\begin{minipage}{0.4\textwidth}
    \begin{center}
    Free electron
    \[G_0(\bm{k}, \omega) = \frac{1}{\omega- \epsilon_{\bm{k}}+\im \delta_{\bm{k}}}\]
\end{center}
\end{minipage}
\begin{minipage}{0.05\textwidth}
    \begin{tikzpicture}
        \coordinate (a) at (0,0.5);
        \coordinate (b) at (0,-0.5);
        \draw[-] (b) -- (a);
        \filldraw[color=black, fill = white , thin] (a) circle (0.05);
        \filldraw[color=black, fill = white , thin] (b) circle (0.05);
       
    \end{tikzpicture}
\end{minipage}
\begin{minipage}{0.4\textwidth}
    \begin{center}
        Interacting electron
    \[G(\bm{k}, \omega) = \frac{z_{\bm{k}}}{\omega- \tilde{\epsilon}_{\bm{k}}+\im \frac{1}{\tau_{\bm{k}}}}\]
    \end{center}
\end{minipage}
\end{center}

Interacting electrons represent a degraded version of the free-electron case, characterized by a quasiparticle weight $z_{\bm{k}} < 1$.\\

\paragraph{Quasi-particles}$~$\\

The main question we have now is: What does the residue look like on the Fermi surface? Consider a low energy electron, close to the Fermi-surface and 
after an interaction happened. It turns out, that if there is a $z_{\bm{k}_F}>0$, we find a precise low energy single particle excitation. This excitation is very close
to the exact eigenfunction of a non-interacting Hamiltonian. We call this state akin to the free electron, a quasi-particle.\\

This is it, a Fermi-liquid is a system of interacting electrons and quasi-particles. 
These quasi-particles are not exact eigenstates of the non-interacting Hamiltonian:
We can not consider them as an electron-like excitation in the non-interacting context. In fact,
the interactions allow to scatter some states in and out of the Bloch-State. However, in a Fermi-Liquid, they are well-defined
enough to be interpreted as electron-akin excitation of the interacting system. \\

The above expression for the momentum distribution $n(\bm{k})$ at Eq. \ref{eq:MomentumDistribution} can be plotted,
 and we recognize a gap of $z_{\bm{k}_F}<1$. 
 This property of the Fermi liquid is a central difference with the non-interacting
system. Back then the momentum distribution had a discontinuity from one to zero when crossing the Fermi surface (Eq. \ref{eq:MomentumDistributionFree}).\\

\begin{figure}[H]\centering

    \begin{tikzpicture}
        \begin{axis}[
            axis lines=middle,
            width=10cm,   % Adjust the width
            height=5cm,  % Adjust the height
            xtick=\empty,
            minor y tick num = 0,
            ytick distance=1,
            ymin=-.2, ymax=1.3,
            xmin=-0.5, xmax=5.1,
            xlabel = \(\bm{k}\),
            ylabel = {\(n(\bm{k})\)},
            legend style={draw=none},
        ]
        \pgfplotsset{every axis/.append style={line width=1pt}}

            % Heaviside step function
            \addplot [
                domain=-1:5,
                samples=200,
                color=black,
                thick,
            ]
            {x < 2 ? 1 : 0};
        \addlegendentry{Free electrons}
        %Here the blue parabola is defined
        \addplot [
            domain=-1:2, 
            samples=100, 
            color=TamGreen,
            ]
            {1 /( exp(1.3*(x - 3.1)) +1)};
        \addlegendentry{Quasiparticles}
        \addplot [
            domain=2:5, 
            samples=100, 
            color=TamGreen,
            ]
            {1 /( exp(1.3*(x - 0.8)) +1)};
            \node(source) at (axis cs:2.2, 0.87){};
            \node[anchor = north](kf) at (axis cs:2, 0){\(\bm{k}_F\)};
            \node (destination) at (axis cs:2.2, 0.13){};
            \draw[Stealth-Stealth,  style={line width=0.5pt} ](source)-- node[midway, anchor = west] {\(z_{\bm{k}}\)} (destination);
        \end{axis}

        \end{tikzpicture}
        \caption{The momentum distribution in a Fermi liquid for free electrons and quasi-particles. The quasi-particles have a smaller gap, allowing to have energies beyond the 
        Fermi surface. This gap has exactly the magnitude of the quasiparticle's weight defined in Eq. \ref{eq:residue}. Inspired by \cite{FossheimSudbo2004}.}
    \end{figure}

An important feature to mention, that is the key understanding of the Fermi-liquid, is the one-to-one correspondence of quantum numbers between the interacting and non-interacting electrons. 
This revealed itself to be correct for metals.
This means we can start with simple Hamiltonian like Eq. \ref{eq:NonInteractingHamiltonian} and add new particles by introducing
some perturbations. The obtained quasi-particles are a result of the excitation of the non-interacting system. We then observe this
one-to-one correspondence.\\

The last question one can ask is: Why does this approximation work so well in the most case? The answer lies in the reference system
we choose to perturb around. Keeping in mind the Pauli principle, we notice that the added particles are restricted in their scattering.
This is so strong, that it almost cancels out the ``switching-on'' of the interactions. It is a good decision to perturb around
the free system. In the free case we are limited in the number of particle in the ground state (Pauli principle).
Therefore, \cite{FossheimSudbo2004} points out that thermodynamic quantities change smoothly when we incorporate the interactions
in the system.\\

We want to emphasize one last time that the Fermi-liquid formalism work because of the momentum restrictions
due to the Pauli exclusion principle.


\paragraph{Repulsive interactions}$~$\\

In condensed matter physics the dominant effect is the repulsive interactions between the electrons due to the Coulomb potential.
 We already described it in the Hamiltonian 
$H_{e^-e^-}$ using some pairwise interactions in Eq. \ref{eq:FermiLiquid_Hamiltonian}. Therefore, the goal is now to
 find an expression for the Coulomb potential in momentum space.\\

As we can see in Eq. \ref{eq:FermiLiquid_Hamiltonian} the system is described in the momentum-space. For this reason we need to 
consider the Fourier-transform of the real space potential. We start with the Coulomb potential which is predominant in the solid state physics.
However, the integral is going to diverge for $r\rightarrow 0$. To solve this problem We
introduce the Yakuwa potential which exponentially modulates the Coulomb potential:
\begin{equation}
    V_{\lambda}(r) = \frac{1}{4\pi\epsilon_0} \frac{e^2}{r} e^{-r\lambda,} \label{eq:Yakuma}
\end{equation}
The motivation for introducing the exponential factor is so that the Fourier transform can be more easily performed.
Letting $\lambda\rightarrow 0$ in the result, we obtain after \cite{Cupcake}:
\begin{equation} \label{eq:Pot_repulsive_el}
    V_{\bm{k},\bm{k}',\bm{q}} = V_{\text{el}}(\bm{q}) := \frac{1}{4\pi\epsilon_0} \frac{2\pi e^2}{\bm{q}^2},
\end{equation}
with $\bm{q}$ the momentum transfer during the interaction. We see that in the same way Coulomb potential in its real
 space form does not depend on the respective positions but only on the distance,
this expression is independent of the initial states. 
The Fermi-liquid remains stable to the repulsive processes, because the accessed energy level are all avoid the surface.\\

However, as we are going to see in the next section, attractive interactions also take place due to an exchange of phonons between two electrons. This will be 
an additional step towards the description of the Meissner state.

\subsection{Instability due to attractive interactions}
We are going to show how the attractive interactions destabilize the Fermi-liquid. This is done by proving
that new ground states open. Leon Cooper introduced a very specific context of attraction, that has a strong influence on the stability. Taking this example we are going
to make clear that attractive interactions can in some cases exceed the Coulomb force, resulting in unexpected new eigenstates.\\

Let's assume we have an inert Fermi sea, where the electrons are considered non-interactive. Adding electrons requires to place them
above the sea. The exotic context is due to the interactions. They can only interact if they are within a small frequency cover $\omega_0$
over the Fermi surface, one on each side, facing them through the complete surface. If this is not the case the interaction vanishes.

Fossheim and Sudbø derived a method in \cite{FossheimSudbo2004} from p.67 to 69. They conclude with the fact that allowing such
interaction leads to a total energy of the interacting system $E$ smaller than $2 \epsilon_{F}$. This means that the attraction of the electrons
shift them in a state that lies under the Fermi sea. Further they showed that if the Fermi sea vanishes (one could take electrons out of the system), 
then this attractive pairing disappears. The same is observed as we approach the classical limit.
By forming a pair the electrons effectively acts as a boson in some ways. These pairs are named Cooper pairs by the eponymous physicist.
The Pauli principle does not rule their energies any more.\\

We now understand why attractive processes create instabilities in the liquid. An energy gap opens bellow the energy surface,
reflecting the energy needed to break the new formed Cooper pairs. The shell $\omega_0$ is the maximum frequency delivering
the ``adhesive'' that allows the electron to pair.
Now that we showed the influence of attractive interactions, we seek some candidate processes that are attractive.

\subsection{Attractive forces mediated by phonons}
As known from condensed matter physics, the lattice can have some internal oscillations called phonons resulting from the spring coupling between the ions.
Now we can imagine that due to the Coulomb interactions, an electron can shift the ions producing a phonon. If this phonon travels and influences another electron
on it way, we result in an effective electron-electron interaction thanks to the phonon. A similar case would be the exchange of a photon between two electrons.
We are going to show how this exchange can lead to an attractive interaction.\\

In a dense lattice the ions move much slower around their equilibrium positions than the light electrons who pass by. The electrons move the charges of the ions, inducing
in a small dipole moment. A second electron that also passes in the surrounding is going to feel the dipole moment, and will be 
attracted. Then the ion shifts back in its new equilibrium position, and the dipole moment vanishes long after the first electron passed.\\

Moreover, due to the Coulomb interactions, the electrons aim to put as much distance as they can between them, in a minimal amount of time. Therefore,
we can say that the $\bm{k}$-quantum number should be opposite between the two electron. If we target to put these concepts in a mathematical form, we use
our previous Hamiltonian, and add an electron-phonon interaction term.
\[
    H = \sum_{\bm{k},\sigma} \epsilon_{\bm{k}} c_{\bm{k},\sigma}^{\dagger}c_{\bm{k},\sigma} + \sum_{\bm{k},\sigma,\bm{k}',\sigma'}
        V_{\bm{k}\bm{k}', \bm{q}} c_{\bm{k}-\bm{q},\sigma}^{\dagger}c_{\bm{k}+\bm{q},\sigma'}^{\dagger}c_{\bm{k},\sigma}c_{\bm{k}',\sigma'} + V_{e^-\text{phonon}}
\]
where $V_{e^-\text{phonon}}$ usually depends on the sum picturing the phonons-modes $\lambda$. These modes are similar to the oscillations modes we have in a CO$_2$-molecule for example.
The expression that forms the phonon-depend potential in momentum space reads 
\begin{equation} \label{eq:Pot_phonon_interaction}
    V_{e^-\text{phonon}} = \sum_{\bm{k},\bm{q},\sigma} M_{\bm{q}}\left(a_{-\bm{q}^{\dagger}} + a_{\bm{q}}\right) c_{\bm{k}+\bm{q},\sigma}^{\dagger}c_{\bm{k},\sigma}.
\end{equation}
$M_{\bm{q}}\left(a_{-\bm{q}^{\dagger}} + a_{\bm{q}}\right) $ is a matrix element of the coupling between the electron and the phonon. 
The $a_{\bm{q}}$ and $a_{\bm{q}}^{\dagger}$ are annihilation and creation operators of a phonon with wave vector $\bm{q}$.\\

Let us now take a step aside and recall the quantum oscillator. Its eigenstates are considered as bosonic. This is achieved by showing that the ladder operators
verify the commutation relation $[a,a^{\dagger}] = 1$. On the other side, the phonons are modelled with springs between the ions. For this reason they
behave in the same manner as the quantum oscillator: By extension phonons act like bosons. For this reason they verify:
\[
    \commu{a_{\bm{q}}}{a_{\bm{q}^{\dagger}}} = \delta_{\bm{q},\bm{q}'}
\] 
Their number is however not conserved in a solid. The matrix element $M$ is a function of the eigenfrequency 
of the phonon $\omega_{\bm{q},\lambda}$ and the Fourier transform $\tilde{V}$ of the electrostatic potential $V_{\lambda}(\bm{q})$ between the electron
and the phonon of mode $\lambda$, if included.
\[
    M_{\bm{q},\lambda} = \im (\bm{q}\cdot \bm{\xi}_{\lambda}) \sqrt{\frac{\hbar}{2M\omega_{\bm{q},\lambda}}} \tilde{V}_{\lambda}(\bm{q}).
\]
We do not provide here a detailed calculation of $M_{q,\lambda}$. This pairing is much weaker than the electron-photon interaction. 
Another important fact is that for $\bm{q}\rightarrow 0$ the matrix element $M$ vanishes. $M$ is proportional to $\bm{q}$ which illustrate how the electron-phonon interaction
happens between a point charge and a dipole.\\

The goal is now the implicitly express the phonon exchange with an effective electron-electron process. If we consider two diagrams, one aiming to describe the absorption of a phonon
and one the emission of a phonon, we can combine them to get a new effective interaction like the photon exchange case.
\begin{figure}[H]
    \centering
    \begin{tikzpicture}
        \begin{feynman}
          \vertex (a) at (2,0){};
          \vertex (b) at (0,0);
          \vertex (c) at (-1,2){\(\bm{k}-\bm{q},\sigma\)};
          \vertex (d) at (-1,-2){\(\bm{k},\sigma\)};
          \vertex[left = 2mm of b] (e) {\(M_{\bm{q}}\)};
          \diagram* {
            (d) -- [fermion] (b) -- [fermion] (c),
            (b)-- [scalar, edge label=\(\bm{q}\)] (a),
        };
        \end{feynman}
      \end{tikzpicture} \hspace{2cm}
    \begin{tikzpicture}
        \begin{feynman}
          \vertex (a) at (-2,0){};
          \vertex (b) at (0,0);
          \vertex (c) at (1,2){\(\bm{k}+\bm{q},\sigma\)};
          \vertex (d) at (1,-2){\(\bm{k},\sigma\)};
          \vertex[right = 2mm of b] (e) {\(M_{\bm{q}}\)};
          \diagram* {
            (d) -- [fermion] (b) -- [fermion] (c),
            (a)-- [scalar, edge label=\(\bm{q}\)] (b),
        };
        \end{feynman}
      \end{tikzpicture}
    \caption{The emission (left) and absorption (right) digram of a phonon of wave vector $\bm{q}$ by an electron.
    The matrix element $M_{\bm{q}}$ gives the probability amplitude of emitting or absorbing the phonon. This plays a role
    in the Fermi golden rule, when it comes to give the transition rate of the scattering of the electron in its new state.}
    \end{figure}
If we represent this interaction by linking both $\bm{q}$-edges, i.e. the emitted phonon is absorbed by another electron, the energy of phonon is simply defined as
\[
    H_{\text{phonon}} = \sum_{\bm{q}} \omega_{\bm{q}} a_{\bm{q}}^{\dagger}a_{\bm{q}}.
\]
We can write a propagator which describes the dashed line in a connected context similarly to the photon case described in Fig. \ref{fig:photon_exchange}.
\[
    D_0(\bm{q},\omega) = \frac{2\omega_{\bm{q}}}{\omega^2 - \omega_{\bm{q}}^2 + \im \eta},
\]
involving a very small quantity $\eta$. From this we obtain an expression for the phonon-mediated interaction of two electrons.
This is performed using $V^{(ph)}_{\text{eff}}(\bm{q},\omega) = \mathcal{R}\text{e}\left({|M_{\bm{q}}|^2 D_0(\bm{q},\omega)}\right)$ and discarding the second 
order $\eta$ term which is very small. We get:
\begin{equation}\label{eq:Pot_attractive_el_phonon}
    V^{(ph)}_{\text{eff}}(\bm{q},\omega) = \frac{2|M_{\bm{q}}|^2\omega_{\bm{q}}}{\omega^2 - \omega_{\bm{q}}^2},
\end{equation}
with $\bm{q}$ the momentum transfer. 
We can now use a more complete potential in the Hamiltonian involving both the electrostatic (Eq. \ref{eq:Pot_repulsive_el}) and effective
phonon-mediated interactions (Eq. \ref{eq:Pot_attractive_el_phonon}):
\begin{equation}\label{eq:V_eff_phonon}
    V_{\text{eff}}(\bm{q}, \omega) =  V_{\text{el}}(\bm{q}) + V^{(ph)}_{\text{eff}}(\bm{q},\omega)
     = \frac{1}{4\pi\epsilon_0} \frac{2\pi e^2}{\bm{q}^2} + \frac{2|M_{\bm{q}}|^2\omega_{\bm{q}}}{\omega^2 - \omega_{\bm{q}}^2}.
\end{equation}

Here we have reached a very important point. This new potential can be in some case negative, which means it results in an attractive interaction between the electrons. 
In other words, in some cases the phonon exchange can be attractive and even overcome the strong repulsive Coulomb potential.

\begin{figure}[H]\centering

    \begin{tikzpicture}
        \begin{axis}[
            axis lines=middle,
            width=7cm,   % Adjust the width
            height=5cm,  % Adjust the height
            xtick=\empty,
            ytick=\empty,
            unbounded coords=jump,
            xmin=-5, xmax=5,
            xlabel = \(\omega\),
            ylabel = {\(V_{\text{eff}}(\omega)\)},
            legend style={draw=none},
            ymin = -5, ymax = 5,
            restrict y to domain =-100:5
        ]
        \pgfplotsset{every axis/.append style={line width=1pt}}


        %Here the blue parabola is defined
        \addplot [
            domain=-5:-2, 
            samples=100, 
            color=black,
            ]
            {1 /(-x-2)};
            \addplot [
                domain=2:5, 
                samples=100, 
                color=black,
                ]{1 /(x-2)};
        \addplot [
            domain=-2:-1.4142, 
            samples=100, 
            color=TamGreen,
            ]
            {1 /(-x-2) + 1/(x-2) +2};
        \addplot [
            domain=-1.4142:1.4142, 
            samples=100, 
            color=black,
            ]
            {1 /(-x-2) + 1/(x-2) +2};
        \addplot [
            domain=1.4142:2, 
            samples=100, 
            color=TamGreen,
            ]
            {1 /(-x-2) + 1/(x-2) +2};
        \end{axis}

        \end{tikzpicture}\hspace{1cm}
        \begin{tikzpicture}
            \begin{axis}[
                axis lines=middle,
                width=7cm,   % Adjust the width
                height=5cm,  % Adjust the height
                xtick=\empty,
                ytick=\empty,
                unbounded coords=jump,
                xmin=-5, xmax=5,
                xlabel = \(\omega\),
                ylabel = {\(V_{\text{eff}}(\omega)\)},
                legend style={draw=none},
                ymin = -5, ymax = 5,
                restrict y to domain =-100:5
            ]
            \pgfplotsset{every axis/.append style={line width=1pt}}
        
            \addplot [
                domain=-2.000001:2.0000001,
                samples=200,
                color=TamGreen,
                thick,
            ]
            {abs(x) < 2 ? -3 : 0};
            \end{axis}
            \end{tikzpicture}
        \caption{Left the course of the phonon-mediated effective potential $V_{\text{eff}}$ defined in Eq. \ref{eq:V_eff_phonon}. For some values of $\omega$ the potential is attractive.
        On the right is a simplified version of the potential that revealed itself to be very accurate with the experiments. Inspired by \cite{FossheimSudbo2004}.}
    \end{figure}
\subsection{Construction of the effective Hamiltonian}
We want to restrict ourselves in the case where the effective Hamiltonian is attractive. This happens in a small shell around the Fermi-surface.
If we want to maximize the phase space for the scattering, the state before and after the scattering have to be in this shell. A good idea is 
to consider that the two electrons have opposite wave vectors. The following figure illustrates this process.
\begin{figure}[H]\centering
    \begin{tikzpicture}
        \coordinate (O) at (-3, -3);
        \coordinate (a) at (2.21, -0.981);
        \coordinate (b) at (1.2, 2.078);
        \filldraw[color=TamDarkBlue2, fill = white!0 , thick, dotted](0,0) circle (2.7);
        \filldraw[color=TamDarkBlue, fill = white!0, very thick](0,0) circle (1.9);
        \draw[->, thick, color = black!70] (0,0) -- (a) node[midway,anchor = south]{$\bm{k}'$};
        \draw[->, thick, color = black!70] (0,0) -- (b) node[midway,anchor = west]{$\bm{k}$};
        \draw[->, thick, color = TamGreen] (a) -- ++(2,-0.3) node[midway, anchor = south]{$\bm{q}$};
        \draw[->, thick, color = black!70] (b) -- ++(-2,0.3) node[midway, anchor = south]{$-\bm{q}$};
        \coordinate (a) at (2.21, -0.981);
        \draw[->, thick] (O) -- ++(.75,0) node[anchor = west]{$\bm{k_x}$};
        \draw[->, thick] (O) -- ++(0,.75) node[anchor = south]{$\bm{k_y}$};
        \filldraw[color=TamDarkBlue, fill = white!0](0,0) circle (0.05);
    \end{tikzpicture}
    \hspace{0.7cm}
    \begin{tikzpicture}
        \coordinate (a) at (1.697, 1.697);
        \coordinate (b) at (-1.697, -1.697);
        \coordinate (shift) at (0,-3);
        \filldraw[color=TamDarkBlue2, fill = white!0 , thick, dotted](0,0) circle (2.7);
        \filldraw[color=TamDarkBlue, fill = white!0, very thick](0,0) circle (1.9);
        \draw[->, thick, color = black!70] (0,0) -- (a) node[midway,anchor = west]{$\bm{k}$};
        \draw[->, thick, color = black!70] (0,0) -- (b) node[midway,anchor = north]{$\bm{k}'$};
        \draw[->, thick, color = black!80] (a) -- ++(-2.3,0.8) node[midway, anchor = south]{$-\bm{q}$};
        \draw[->, thick, color = black!80] (b) -- ++(2.3,-0.8) node[midway, anchor = north]{$\bm{q}$};
        \filldraw[color=white, fill = white!0 , thick, dotted] (shift) circle (0.1);
        \filldraw[color=TamDarkBlue, fill = white!0](0,0) circle (0.05);
    \end{tikzpicture}
    \caption{The scattering process of two electrons with opposite wave vectors $\bm{k}$ and $\bm{k}'$. We have a momentum transfer $\bm{q}$ between them. 
    The thick line illustrates the Fermi surface and the dotted one the thin shell. As we see, if the electrons have opposite wave vectors, and the $\bm{k}$ electron
    scatters into the shell, then the $\bm{k}'$ electron scatters in the shell as well. This is not always the case if the wave vectors do not agree, as we can see on the left figure.
    The right figure points out the maximization process. With opposite wave vectors,
    we get the largest possibility spectrum for the scattering. Inspired by \cite{FossheimSudbo2004}.}
\end{figure}
Further the attraction is a short range effect, so if we want to consider it, we must think that the electrons are very close to each other.
This requires the electrons to have opposite spins due to the Pauli principle. Latter we are going to introduce the tight binding formalism. In this context we want
the electrons to be on the same lattice site to be able to form a pair. This shell approximation turns out to be a good model.\\

Now we allow us to rename some variables:
\[
    \bm{k} + \bm{q} \longrightarrow \bm{k} ~,~~ \bm{k} \longrightarrow \bm{k}'.
\]
The Hamiltonian that follows from these transformations is called the BCS-reduced Hamiltonian and first found by Bardeen, Cooper and Schrieffer in 1957 \cite{BCS1957}.
\begin{equation} \label{eq:BCS_ReducedHamiltonian}
    H = \sum_{\bm{k},\sigma} \epsilon_{\bm{k}} c_{\bm{k},\sigma}^{\dagger}c_{\bm{k},\sigma} +
     \sum_{\bm{k},\bm{k}',\sigma} V_{\bm{k}\bm{k}'} c_{\bm{k},\sigma}^{\dagger}c_{-\bm{k},\overline{\sigma}}^{\dagger}c_{-\bm{k}',\overline{\sigma}}c_{\bm{k}',\sigma},
\end{equation}
where $\overline{\sigma}$ is the opposite spin of $\sigma$. 
The two rightmost operators destroy the electron with opposite wave vectors and spin before the interaction and creates two new with opposite wave vectors, including the
 loss/gain of momentum, and opposite spin after the interaction. To convince oneself, we can just look at the expression before the variable transformation. The potential $V_{\bm{k}\bm{k}'}$ 
 that modulates the strength of this pairing is now a matrix element that acts if the wave vectors are close to the Fermi-surface. The electrons have to move in opposite directions with opposite spins.
Due to the retardation processes we introduced earlier, there remains a distortion in the lattice long after the electrons passed. Due to the inducting dipole moment,
the other electron is attracted towards the distortion. As we also saw, the Coulomb repulsion causes a collinear displacement, close to the distortion
that its homologue produced. This phenomenon is called the Cooper-pairing and is a coupling that happens in momentum space.\\

The reader might want to see a picture of what is happing in the real space. To achieve a representation two simple statements are enough. 
First, the Coulomb force aims to maximize the distance between the electrons in minimal time.
 This is achieved by moving them collinearly in opposite directions. Second, due to the retardation process, it is energetically more favourable
for the electron to move along the distortion of the lattice. The result is illustrated in Fig. \ref{fig:CooperPairing}. 
\begin{figure}[H]
    
    \begin{tikzpicture}
        \pgfmathsetmacro{\spacing}{1}
        \pgfmathsetmacro{\lineA}{0}
        \foreach \x in {0,1,2,3,4,5}{
            \foreach \sign in {0,1}
            {
                \ifthenelse{\equal{\sign}{0}}{
                \draw[-] ($(\x*\spacing,0)$) -- ($(\x*\spacing,-6*\spacing)$);
                }
                {
                \draw[-] ($(0,-\x*\spacing)$) -- ($(5*\spacing,-\x*\spacing)$);
                }
            }
        }
        \draw[-] (0,-6*\spacing) -- (5*\spacing,-6*\spacing);
        \foreach \ystable in {0,-3,-6}{
            \foreach \x in {0,1,2,3,4,5}
            {
                \filldraw[fill=TamLightGY!70,thick] ($(\x*\spacing, \lineA+ \ystable*\spacing)$) circle (0.05);
            }
        }
        \pgfmathsetmacro{\distor}{0.04}
        \pgfmathsetmacro{\yRealtiv}{\spacing*1.5}
        \foreach \distortion in {0,1}
        {
            \foreach \sign in {1,-1}
            {   
            \pgfmathsetmacro{\yDistor}{0}
                \foreach \x in {0,1,2,3,4,5}
                {
                    \ifthenelse{\equal{\distortion}{1}}{\pgfmathsetmacro{\X}{5-\x}}{\pgfmathsetmacro{\X}{\x}}
                    \filldraw[fill=TamGreen!70,thick] ($(\X*\spacing, \lineA-\yRealtiv - \sign*\spacing/5 -  \distortion*3 )$) circle (0.05);
                    % \draw[->,thick] ($(\X*\spacing, \lineA-\yRealtiv -\sign*\distor*\x - \sign*\spacing/5 -  \distortion*3 )$) -- ($(\X*\spacing, \lineA-\yRealtiv +\sign*\distor*\x  -0.3*\sign*\distor -0.3*\sign - \sign*\spacing/5 -  \distortion*3 )$);
                }
            }
        }
        \draw[->,semithick] (5.5,-1.5) -- (-1,-1.5) node[anchor = east]{$e^-$};
        \draw[->,semithick] (-0.5,-4.5) -- (6,-4.5) node[anchor = west]{$e^-$};

    \end{tikzpicture}
    \centering
    \caption{A lattice of ions, where two electrons propagate with opposite directions. Inspired by \cite{FossheimSudbo2004}.} \label{fig:CooperPairing}
\end{figure}
Here the moved ions in blue are shifting back to their equilibrium position long after the passage of the second electron.
However, it is important to remember, that when the second electron passes, it experiences almost the same distortion everywhere.
An interesting fact is that these interactions are the source of the superconductivity, but also the main origin of resistivity in clean materials.
    
\subsection{On our way to the BCS-theory}
After this introduction on the phonon coupling between the electrons in the momentum space, or Cooper-pairing,
we aim to describe the energy of the superconductor in a mean-field approach. The goal of it is to reduce the description
involving neighbours, to an on-site representation in the mean field of the other sites. Therefore, we are going to describe 
a one-body problem which is easier to compute. As known, mean-fields approaches require self-consistent equations to be comprehensive.\\

The first step is to introduce the following expectation values:
\begin{align}
    b_{\bm{k}} = &\langle c_{-\bm{k}\downarrow}c_{\bm{k}\uparrow}\rangle \label{eq:ExpectBCS} \\
    b_{\bm{k}}^{\dagger} = &\langle c_{\bm{k}\uparrow}^{\dagger}c_{-\bm{k}\downarrow}^{\dagger}\rangle  \label{eq:ExpectBCSDag}
\end{align}
which lead to a new expression for the $c$ operators:
\begin{equation}
    c_{-\bm{k}\downarrow}c_{\bm{k}\uparrow} = b_{\bm{k}} + \underbrace{c_{-\bm{k}\downarrow}c_{\bm{k}\uparrow} - b_{\bm{k}}}_{\delta_{b_{\bm{k}}}},
\end{equation}
where we can see the $\delta_{b_{\bm{k}}}$ as a deviation, or fluctuation term. If we introduce it back into the BCS-reduced Hamiltonian of Eq. \ref{eq:BCS_ReducedHamiltonian},
we can write
\[
    H = \sum_{\bm{k}\sigma} \epsilon_{\bm{k}} c_{\bm{k}\sigma}^{\dagger}c_{\bm{k}\sigma} + \sum_{\bm{k}\bm{k}'} V_{\bm{k}\bm{k}'} \bigl( b_{\bm{k}}^{\dagger} + \delta_{b_{\bm{k}}}^{\dagger}\bigr)\bigl( b_{\bm{k}} + \delta_{b_{\bm{k}}}\bigr).
\]
We can compute the product of the two terms in parentheses and forget the terms in $\mathcal{O}\left(\delta_{b_{\bm{k}}}^2\right)$, because the fluctuation are small.
We then obtain the following expression
\[
    H = \sum_{\bm{k}\sigma} \epsilon_{\bm{k}} c_{\bm{k}\sigma}^{\dagger}c_{\bm{k}\sigma} + \sum_{\bm{k},\bm{k}'} V_{\bm{k}\bm{k}'} \left( b_{\bm{k}}^{\dagger}c_{-\bm{k}'\downarrow}c_{\bm{k}'\uparrow}  + b_{\bm{k}'}^{\dagger} c_{\bm{k}\uparrow} ^{\dagger}c_{-\bm{k}\downarrow}^{\dagger} -  b_{\bm{k}}^{\dagger} b_{\bm{k}'}\right).
\]

The next step is to define the superconducting gap parameter $\Delta$, which is a key thermodynamic variable to the description of the Cooper-pairs:
\begin{align}
    \Delta_{\bm{k}} := &-\sum_{\bm{k}'} V_{\bm{k}\bm{k}'} b_{\bm{k}}^{\dagger},\\
    \Delta^{\dagger}_{\bm{k}} := &-\sum_{\bm{k}'} V_{\bm{k}\bm{k}'} b_{\bm{k}'}.\label{eq:DeltaBaseDef}
\end{align}
This brings our Hamiltonian in another form:
\begin{equation} \label{eq:HamiltonianBCS1}
    H = \sum_{\bm{k}\sigma} \epsilon_{\bm{k}} c_{\bm{k}\sigma}^{\dagger}c_{\bm{k}\sigma} - \sum_{\bm{k}} \left( \Delta_{\bm{k}}^{\dagger} c_{-\bm{k}'\downarrow}c_{\bm{k}'\uparrow}  + \Delta_{\bm{k}}^{\dagger} c_{\bm{k}\uparrow} ^{\dagger}c_{-\bm{k}\downarrow}^{\dagger} -  b_{\bm{k}}^{\dagger} \Delta_{\bm{k}}^{\dagger} \right).
\end{equation}
We took the liberty to split the sum, rename the $\bm{k}'$ to $\bm{k}$ and recombine the sum. We notice that this form involves a lot of creation and annihilation terms, and this is not 
is rather uncommon for an effective non-interacting electron gas. Furthermore, we remember that we aim a one particle description in the mean field of its neighbours.
This complexity will lead to some difficulties to express the quasi-particle spectrum. A good solution is 
to rotate the basis of the $c$ operators, to land in a basis that diagonalizes the Hamiltonian and therefore minimizes the number of operators.\\

The transformation involves two new fermionic operators $\eta$ and $\gamma$ that therefore respect Eq. \ref{eq:Fermion1} to \ref{eq:Fermion3} and reads in a matrix form:
\begin{equation}\label{eq:RotationBasis_c1}
    \begin{pmatrix}
        c_{\bm{k}\uparrow}\\
        c_{-\bm{k}\downarrow}^{\dagger}
    \end{pmatrix} = 
    \begin{pmatrix}
        \cos(\theta) & -\sin(\theta)\\
        \sin(\theta) & \cos(\theta)
    \end{pmatrix}
    \begin{pmatrix}
        \eta_{\bm{k}}\\
        \gamma_{\bm{k}}
    \end{pmatrix},
\end{equation}
along with the conjugate transpose of each component of the l.h.s vector, as well in a matrix-vector form:
\begin{equation}\label{eq:RotationBasis_c2}
    \begin{pmatrix}
        c_{\bm{k}\uparrow}^{\dagger} \\
        c_{-\bm{k}\downarrow}
    \end{pmatrix} = 
    \begin{pmatrix}
        \cos(\theta) & -\sin(\theta)\\
        \sin(\theta) & \cos(\theta)
    \end{pmatrix}
    \begin{pmatrix}
        \eta_{\bm{k}}^{\dagger} \\
        \gamma_{\bm{k}}^{\dagger}
    \end{pmatrix}.
\end{equation}
We can reintroduce these into the Hamiltonian of Eq. \ref{eq:HamiltonianBCS1}. 
Some multiplications involve $\cos(\theta)^2 - \sin(\theta)^2 = \cos(2\theta)$ and $\cos(\theta)^2 + \sin(\theta)^2 = 1$. 
Further due to the anticommutations we have $\eta \gamma^{\dagger} = - \gamma^{\dagger}\eta$ and so on for $\gamma \eta^{\dagger} = - \eta^{\dagger}\gamma$.
We obtain the following expression
\begin{equation}
    \begin{aligned}
        H =& \sum_{\bm{k}} \epsilon_{\bm{k}} + \Delta_{\bm{k}}b_{\bm{k}}^{\dagger}\\
          +& \sum_{\bm{k}} \left[\epsilon_{\bm{k}}\cos(2\theta) - \cos(\theta)\sin(\theta)\left(\Delta_{\bm{k}}^{\dagger}+ \Delta_{\bm{k}}\right)\right] \eta_{\bm{k}}^{\dagger}\eta_{\bm{k}}\\
          -& \sum_{\bm{k}} \left[\epsilon_{\bm{k}}\cos(2\theta) -\sin(\theta)\cos(\theta)\left(\Delta_{\bm{k}}^{\dagger}+ \Delta_{\bm{k}}\right)\right]\gamma^{\dagger}_{\bm{k}}\eta_{\bm{k}}\\
          -& \sum_{\bm{k}} \left[\Delta_{\bm{k}} \cos(\theta)^2 - \Delta_{\bm{k}}^{\dagger}\sin(\theta)^2 + 2\epsilon_{\bm{k}} \cos(\theta)\sin(\theta) \right]\eta^{\dagger}_{\bm{k}}\gamma_{\bm{k}}\\
          -& \sum_{\bm{k}} \left[\Delta_{\bm{k}} \cos(\theta)^2 - \Delta_{\bm{k}}^{\dagger}\sin(\theta)^2 + 2\epsilon_{\bm{k}} \cos(\theta)\sin(\theta) \right]\gamma^{\dagger}_{\bm{k}}\eta_{\bm{k}}.     
    \end{aligned}
\end{equation}
The goal is to diagonalize the Hamiltonian in the $(\gamma, \eta)$ basis. Therefore, we have to choose $\theta$, such that the terms with $\gamma^{\dagger}\eta$ and $\eta^{\dagger}\gamma$ vanish.
A difficulty that we may encounter along with this idea is that $\Delta$ is an order parameter and own a complex phase fluctuation. In other words one can write
$\Delta = |\Delta|e^{\im\tau}$ where $\tau$ has some fluctuations, but we are going to ignore them.\\

We can set
\begin{equation*}
    \Delta_{\bm{k}} = \Delta_{\bm{k}}^{\dagger}~~~~ \text{and} ~~~~ \tan(2\theta) = -\frac{\Delta_{\bm{k}}}{\epsilon_{\bm{k}}}
\end{equation*}
and introduce two new variables called the coherence factors:
\begin{align}
    v_{\bm{k}}^2 := \sin(\theta)^2 &= \frac{1}{2} \left(1 - \frac{\epsilon_{\bm{k}}}{E_{\bm{k}}}\right), \label{eq:coherenceFac_v}\\
    u_{\bm{k}}^2 := \cos(\theta)^2 &= \frac{1}{2} \left(1 + \frac{\epsilon_{\bm{k}}}{E_{\bm{k}}}\right)  \label{eq:coherenceFac_u},
\end{align}
along with a new energy $E_{\bm{k}} = \sqrt{\epsilon_{\bm{k}}^2 + |\Delta_{\bm{k}}|^2}$. These factors play an important
role in NMR as well as in the ultrasound propagation in superconductors. Cooper and Schrieffer made some correct
predictions in an advanced many-body system. This is labelled as one of the greatest achievements in condensed matter physics in the 20th century \cite{FossheimSudbo2004}.
They are going as well to play a central role in the simulations later.\\

We obtain a Hamiltonian that looks very similar to a free fermion quasiparticle gas:
\begin{equation}\label{eq:HamiltonianBCS2}
H = \underbrace{\sum_{\bm{k}} \left(\epsilon_{\bm{k}} + \Delta_{\bm{k}} b_{\bm{k}}^{\dagger}\right)}_{\substack{=:H_0~ \text{constant}\\\text{ mean-field term}}}  
+ \underbrace{\sum_{\bm{k}} E_{\bm{k}}\left(\eta_{\bm{k}}^{\dagger}\eta_{\bm{k}} - 
\gamma_{\bm{k}}^{\dagger}\gamma_{\bm{k}}\right)}_{\substack{\text{Spinless fermion system with} 
\\\text{two types of fermions of}\\\text{energies }E_{\bm{k}}\text{ and }-E_{\bm{k}}.}}.
\end{equation}
This expression forgets the interactions to describe two types of particles in a mean field. 
As \cite{FossheimSudbo2004} described it in chapter 3, p.83-84, one can note that we do not have any spins involved. This is due
to the fact that we describe the quasiparticles as a linear combination of electrons and holes with opposite spins (see Eq. \ref{eq:RotationBasis_c1}).
The two relevant degrees of freedom are distinct linear combinations of particle-hole 
pairs with spin-singlet components. However, the resulting quasiparticles are not spin-singlets, as they carry a non-zero spin.
The degrees of freedom are therefore preserved.\\

The energy $E_{\bm{k}}$ present a gap in the energy of the quasiparticles when looking at the Fermi surface $\epsilon_{\bm{k}}=0$ (we remember that $\mu$ is present in $\epsilon_{\bm{k}}$).
This is due to the fact that the expectation values we introduced in \ref{eq:ExpectBCS} and \ref{eq:ExpectBCSDag} do not vanish. The gap is a result of the Cooper-pairing. These expectation values
are the order parameters of the Meissner state and shouldn't be confused with the superconducting gap $\Delta$. They can however be zero both at the same time. \\

This free particle context is a good opportunity to introduce the grand canonical ensemble of the Hamiltonian. 
Even if we could give the direct result, deriving this ensemble involves a lot of important steps, so for
the seek of completeness, we are going to go step to step through it.\\

We define the particle number operator $N = \sum_{\bm{k}}\left( n_{\eta\bm{k}} + n_{\gamma\bm{k}}\right)$, that counts the quasiparticles of both class. 
Further the possible states $\bm{k}$ are in a ``continuous'' set $\mathfrak{K}= \left\{\bm{k}_1,\bm{k}_2,..\right\}$. The set $\{n_{\bm{k}}\}$ gathers
the different occupation numbers of all the states $\bm{k}$ under one configuration. The configurations are stored in an ensemble $\mathcal{C}$. 
Further we consider fermions, so the occupation numbers of the particle
 type $\eta$ or $\gamma$ in state $\bm{k}$ are labelled $n_{\eta\bm{k}}$ and $n_{\gamma\bm{k}}$ 
 and equals either 0 or 1.
\begin{align*}
    Z_G &= \text{Tr}\left(e^{-\beta( H)}\right)\\
    &= \sum_{\{n_{\bm{k}}\}\in\mathcal{C}} e^{-\beta H_0} \langle \{n_{\bm{k}}\}|\exp\left(\sum_{\bm{k'}}-\beta E_{\bm{k'}}\eta_{\bm{k'}}^{\dagger}\eta_{\bm{k'}} + \beta E_{\bm{k'}}\gamma_{\bm{k'}}^{\dagger}\gamma_{\bm{k'}}\right)|\{n_{\bm{k}}\}\rangle\\
    &= \sum_{\{n_{\eta\bm{k}}\}} \sum_{\{n_{\gamma\bm{k}}\}} e^{-\beta H_0}\exp\left(\sum_{\bm{k}}-\beta \left[E_{\bm{k}}\eta_{\bm{k}}^{\dagger}\eta_{\bm{k}}\right]+ \beta \left[E_{\bm{k}}\gamma_{\bm{k}}^{\dagger}\gamma_{\bm{k}}\right]\right)\\
    &= e^{-\beta H_0} \sum_{\substack{n_{\eta\bm{k}_1},\\ n_{\eta\bm{k}_2},..}} ~~\sum_{\substack{n_{\gamma\bm{k}_1}, \\n_{\gamma\bm{k}_2},..}}\exp\left(\sum_{\bm{k}}-\beta \left[E_{\bm{k}}n_{\eta\bm{k}}\right]+ \beta \left[E_{\bm{k}}n_{\gamma\bm{k}}\right]\right)\\
    &= e^{-\beta H_0} \sum_{\substack{n_{\eta\bm{k}_1},\\ n_{\eta\bm{k}_2},..}} \prod_{\bm{k}}\exp\left(-\beta\left[ E_{\bm{k}}n_{\eta\bm{k}} \right]\right)  \sum_{\substack{n_{\gamma\bm{k}_1}, \\n_{\gamma\bm{k}_2},..}}\prod_{\bm{k}} \exp\left(\beta\left[E_{\bm{k}}n_{\gamma\bm{k}}\right]\right) \\
    &= e^{-\beta H_0} \sum_{n_{\eta\bm{k}_1}}\exp\left(-\beta\left[E_{\bm{k_1}}n_{\eta\bm{k}_1} \right]\right)\sum_{n_{\eta\bm{k}_2}}\exp\left(-\beta\left[E_{\bm{k_2}}n_{\eta\bm{k}_2} \right]\right)\cdot..  \\
    &~~~~~~~~~~~~\sum_{n_{\gamma\bm{k}_1}} \exp\left(\beta\left[E_{\bm{k}_1}n_{\gamma\bm{k}_1}\right]\right)\sum_{n_{\gamma\bm{k}_2}} \exp\left(\beta\left[E_{\bm{k}_2}n_{\gamma\bm{k}_2}\right]\right)\cdot..\\
    &= e^{-\beta H_0} \prod_{\bm{k}}  \left(1 + e^{-\beta E_{\bm{k}}}\right) \left(1 + e^{\beta E_{\bm{k}}}\right) \\
\end{align*}
This partition function becomes useful if we want to derive the free energy of the system. It will help us to solve the self-consistent equation for the gap.\\
According to \cite{FuchsBaugaertel2023} p.99 the free energy can be derived from the partition function as:
\[
    F = -\frac{1}{\beta}\ln(Z_G) = H_0 -\frac{1}{\beta}\sum_{\bm{k}} + \ln\left(1 + e^{-\beta E_{\bm{k}}}\right)\ln\left(1 + e^{\beta E_{\bm{k}}}\right).
\]
Following \cite{FossheimSudbo2004} we obtain in the limit of low temperatures ($\beta\rightarrow \infty$):
\begin{align*}
    F &= H_0 + \sum_{\bm{k}} E_{\bm{k}} \theta\left(-E_{\bm{k}}\right) + E_{\bm{k}} \theta\left(E_{\bm{k}}\right)\\
      &= H_0 + \sum_{\bm{k}} E_{\bm{k}} - E_{\bm{k}}\theta\left(E_{\bm{k}}\right) + E_{\bm{k}} \theta\left(E_{\bm{k}}\right)\\
      &= \sum_{\bm{k}} \epsilon_{\bm{k}} + \Delta_{\bm{k}}b_{\bm{k}}^{\dagger} E_{\bm{k}},
\end{align*}
where we used the properties from the Heaviside step-function described in \ref{eq:Heaviside}.
The minimization of the free energy $F$ should self-consitently determine the gap $\Delta_{\bm{k}}$. This is a statement that depends neither on the momentum space structure,
nor on the attraction of the potential \cite{FossheimSudbo2004}. $E_{\bm{k}}$ has an implicit $\Delta_{\bm{k}}$-dependence.
We compute the derivation $\parDif{F}{\Delta_{\bm{k}}}$ and search the argument of the zero-position. Deriving one $\bm{k} $ from the sum is enough. We demand the following
to be satisfied:
\begin{equation}
    \parDif{F}{\Delta_{\bm{k}}} = 0 ~,~~~ \parDif{F}{\Delta_{\bm{k}}^{\dagger}} = 0.
\end{equation}
This leads to
\begin{equation*}
    b_{\bm{k}}^{\dagger} = \Delta_{\bm{k}} \underbrace{\frac{\tanh(\beta E_{\bm{k}}/2)}{E_{\bm{k}}}}_{\chi(\bm{k})},
\end{equation*}
defining $\chi(\bm{k})$ as the pair-susceptibility, and gives how capable the system is to create Cooper-pairs \cite{FossheimSudbo2004}. If we allow us to relabel 
$\bm{k}\rightarrow\bm{k}'$, then multiply on both sides
with $-\sum_{\bm{k}} V_{\bm{k}\bm{k}'}$ and introduce Eq. \ref{eq:DeltaBaseDef}, we obtain the self-consistent equation for the gap. This is usually designated as the BCS gap equation.
\begin{equation}\label{eq:BCS_gap_eq}
    \Delta_{\bm{k}} = -\sum_{\bm{k}} V_{\bm{k}\bm{k}'} \Delta_{\bm{k}'}\frac{\tanh(\beta E_{\bm{k}'}/2)}{E_{\bm{k}'}}.
\end{equation}

Fossheim and Sudbø \cite{FossheimSudbo2004} emphasize that this equation is not limited to phonon mediated interaction, as long we
do not specify the potential. Further Cooper-pairs can condensate, because in the pair, the spin of the electrons adds up to zero.
The pair is not ruled any more by the Pauli principle and can take part into the superconducting condensate.
Even if the formation temperature of Cooper-pairs in higher than the critical temperature $T_C$, these mean field approach makes these temperatures agree.\\

As we introduced earlier the gap has a complex phase fluctuation. This phase is hard to vary in good metals because the number of carrier electrons is high.
On the other side in poor metals, the fluctuations that break the pair are more easily reached, i.e. at a lower temperature than in good metal. Superconductivity
is therefore more stable in metals of good value.\\

In this section we saw how to reduce the many-body Hamiltonian to a single-body problem in the means field of its neighbours. The system that follows from this
is a non-interacting gas of electrons and holes. This involves the introduction of the 
superconducting gap parameter that we can thank to the statistical mechanics formalism express in a self-consistent way.


\subsection{Generalized gap equation, $s$-wave and $d$-wave superconductivity}
As we introduced in an earlier section, the formalism follows from the attractive phonon exchange but can be generalized. The superconductivity
is the result of the pairing of the electrons into Cooper-pairs. These pairs shift into a condensate and take part to a coherent matter-wave.
Making a link to the coherent light wave, superconductivity can be seen as the \textit{analogos} of lasers like Fossheim and Sudbø 
highlight it in \cite{FossheimSudbo2004}.\\

To achieve such generalization we let the formalism be open to other kind of interactions and do not restrict it to the thin shell around the Fermi-surface.
The potential matrix-element can therefore take a more complex form than introduced in Eq. \ref{eq:V_eff_phonon}.\\

We recall that the electrons move into a lattice. We can assume that this crystal owns some symmetries that are reflected in its crystallographic (complete) basis $\{g_{\eta}(\bm{k})\}$.
Similar to the Bloch state, we could imagine that the symmetries emphasize some physical quantities. Assuming this, we propose a form for the potential that takes advantage
of the basis.
\[
    V_{\bm{k}\bm{k}'} = \sum_{\eta} \lambda_{\eta} g_{\eta}(\bm{k})g_{\eta}(\bm{k}')
\]
$\eta$ is also often labelled as the channel \cite{FossheimSudbo2004}. Assuming we have a square lattice, 
\begin{figure}[H]
    \centering
    \begin{tikzpicture}
        \filldraw[color=black, fill = white , thick](0,0) circle (0.1);
        \filldraw[color=black, fill = TamLightGreen , thick](1,0) circle (0.1);
        \filldraw[color=black, fill = TamLightGreen , thick](-1,0) circle (0.1);
        \filldraw[color=black, fill = TamLightGreen , thick](0,1) circle (0.1);
        \filldraw[color=black, fill = TamLightGreen , thick](0,-1) circle (0.1);
        
        \filldraw[color=black, fill = TamYellow , thick](1,1) circle (0.1);
        \filldraw[color=black, fill = TamYellow , thick](-1,1) circle (0.1);
        \filldraw[color=black, fill = TamYellow , thick](1,-1) circle (0.1);
        \filldraw[color=black, fill = TamYellow , thick](-1,-1) circle (0.1);
    \end{tikzpicture}
\end{figure}
with interaction strength $U/2$ on site, for nearest neighbours $2V$ and $4W$ for second-nearest neighbours. We can for each site describe their position with a shift vector 
to the central site. Using
\[
    f(k,k')=\cos(k)\cos(k')+\sin(k)\sin(k'),
\]
Fourier transforming these displacements leads to the following:
\[
\begin{aligned}
    V_{\bm{k}\bm{k}'} = U + &~2V\left(f(k_x,k_x')+ f(k_y,k_y')\right)\\
    + &~4W \left(f(k_x,k_x') \cdot f(k_y,k_y')\right)
\end{aligned} 
\]
We can then introduce the basis-functions $\{g_{\eta}(\bm{k})\}$ we talked about a few paragraphs ago.
\begin{align*}
    g_{1}(\bm{k}) =& \frac{1}{2\pi}\\
    g_{2}(\bm{k}) =& \frac{1}{2\pi}\bigl( \cos(k_x) + \cos(k_y)\bigr) \tag{$s$-wave}\\
    g_{3}(\bm{k}) =& \frac{1}{2\pi}\cos(k_x) \cos(k_y)\\
    g_{4}(\bm{k}) =& \frac{1}{2\pi}\bigl( \cos(k_x) - \cos(k_y)\bigr) \tag{$d$-wave}\\
    g_{5}(\bm{k}) =& \frac{1}{2\pi}\sin(k_x) \sin(k_y)\\
\end{align*}
from which involves the following lambdas $\lambda_1 = 2U\pi^2, ~ \lambda_2=\lambda_4 = 4V\pi^2, ~ \lambda_3=\lambda_5=4W \pi^2$, and for greater
$\eta$ we set $\lambda_{\eta\ge6} =0$.\\

For $\eta \in \{1,2\}$, the corresponding $g$ is the identity under all symmetries of $C_{v4}$ and for $\eta \in \{3,4,5\}$ we observe a change of sign under  $\pi/2$ rotations.
The same behaviour is found in the spherical harmonics for the respective  quantum numbers $l=0$ and $l=2$. From this we can deduce the use of the terms $s$-wave and $d$-wave.
$s$-wave is isotropic in the momentum space whereas the $d$-wave has a different contour for its part with positive phase
\begin{figure}[H]
    \centering
    \begin{tikzpicture}
        \coordinate (A) at (0,0);
        \coordinate (B) at (2.5,0);
        \coordinate (C) at (7.5,0);
        \filldraw[draw = TamLightGreen ,fill=TamLightGreen!30]  (A) circle (1.3);
        \fill[fill=TamYellow!50] (B)++(-0.5,0) arc[x radius = 2, y radius = 1.5 , start angle=180, end angle=-180];
        \fill[fill=TamYellow!50] (B)++(0.0,0) arc[x radius = 1.5, y radius = 2 , start angle=180, end angle=-180];
        
        \fill[fill=red!20, thick] (C) arc[x radius = 1, y radius = 2 , start angle=180, end angle=-180] ;
        \filldraw[draw=TamDarkBlue,fill=TamDarkBlue!30] (C)++(-1,0) arc[x radius = 2, y radius = 1 , start angle=180, end angle=-180] ;
        \draw[draw=red] (C) arc[x radius = 1, y radius = 2 , start angle=180, end angle=-180] ;

        \node[anchor=center] at ($(A) - (0,2.5)$) {$s$-wave};
        \node[anchor=center] at ($(B) - (-1.5,2.5)$) {ext. $s$-wave};
        \node[anchor=center] at ($(C) - (-1,2.5)$) {$d$-wave};

    \end{tikzpicture}
    \caption{Different gap contours in the momentum space. The $s$-wave is isotropic and extended $s$-wave has a more complex
     surface (see appendix \ref{sec:Unconventional_order_parameters}).
    The $d$-wave has a splitting for its part with positive and negative phase. We observe a degeneracy where the two parts overlap.}
\end{figure}

Further we can reintroduce the BCS gap equation \ref{eq:BCS_gap_eq} and obtain a lattice-dependent form for the gap:
\begin{equation}
    \begin{aligned}
    \Delta_{\bm{k}} =& -\sum_{\bm{k}'} V_{\bm{k}\bm{k}'}\Delta_{\bm{k}'}\chi(E_{\bm{k}})\\
    =& -\sum_{\eta\in\natset{5}} \lambda_{\eta} g_{\eta}(\bm{k}) \underbrace{\sum_{\bm{k}'} g_{\eta}(\bm{k}') \Delta_{\bm{k}'}\chi(E_{\bm{k}'})}_{=: \Delta_{\eta}/\lambda_{\eta}}\\
    =& \sum_{\eta\in\natset{5}}\Delta_{\eta} g_{\eta}(\bm{k}).
    \end{aligned}
\end{equation}
This is a nice form to have as Fossheim and Sudbø \cite{FossheimSudbo2004} point out p.92. The gap is a linear combination of the physical quantities $g_{\eta}(\bm{k})$.
The newly introduced $\Delta_{\eta}$ are independent of the wave vector $\bm{k}$ but are function of the temperature. We can express these in the basis 
of another $\eta'$ ($\lambda_{\eta\ge6} =0$):
\begin{align*}
    \Delta_{\eta} = &~\sum_{\eta' \in \natset{5}} \Delta_{\eta'} \mathcal{M}_{\eta,\eta'}\\
    \mathcal{M}_{\eta,\eta'} =& -\lambda_{\eta}\sum_{\bm{k}} g_{\eta}(\bm{k})g_{\eta'}(\bm{k}) \chi(E_{\bm{k}}).
\end{align*}
These are numerical cumbersome to compute, but taking $U>0$, $V<0$ and $W=0$ in the square lattice we obtain no attraction in the 
$\lambda_1$-channel but attraction in the s- and $d$-wave channels. This tells us a lot like \cite{FossheimSudbo2004} analysed it. 
First the gap is highly associated to the Fourier transform of the wave function of the Cooper-pairs.    
Further the gap accommodate itself ``to be zero in the channel where the wave function is non-zero for zero separation between the electrons''\cite{FossheimSudbo2004} p.92.
The s- and $d$-wave channels are preferred in order to avoid the on-site Coulomb repulsion (i.e. the Coulomb force between the two members of the pair).
This can be put in comparison with the special
case of the phonon pairing. In the latest the electrons used retardation processes to cancel out the repulsion. They avoided themselves in time. 
Here the avoidance takes place in space.\\

The manifestation of the retardation effects are actually a direct consequence of restricting ourselves to a thin shell around the Fermi-surface.
Logically if we have a Cooper pair formation that does not follow this restriction, the whole Brillouin zone should be taken into account to compute the gap.\\

Further we can express the Fourier transform of the gap as $\Delta(\bm{r}) = \sum_{\bm{k}} \Delta_{\bm{k}} e^{i\bm{k}\bm{r}}$ which leads $\Delta(0) = \sum_{\bm{k}} \delta_{\bm{k}}$.
However, a repulsive interaction is obtained for $\Delta(\bm{0}) \neq 0$, so we need to find $\Delta_{\bm{k}}$s whose sum satisfies this. 
This can be done using the $s$-wave and $d$-wave channels, i.e. $\Delta_{\bm{k}} = \Delta_0(T)g_2(\bm{k})$ and $\Delta_{\bm{k}} = \Delta_0(T)g_4(\bm{k})$.
We already motivated superconductivity and its difficulty to maintain due to the phase fluctuations above the (freezing) critical temperature $T_C$. 
However, superconductivity involving $d$-wave channels is believed \cite{FossheimSudbo2004} p.92 to be found in materials with high $T_C$.
$\Delta_{\bm{k}} = \Delta_0(T)g_4(\bm{k})$ is therefore a good case of study.

\paragraph{Density of Meissner states} $~$\\
We imagine some fluctuation described by $g(\bm{k})$ 
that only depend on $E_{\bm{k}} = \sqrt{\epsilon_{\bm{k}}^2 + \Delta_{\bm{k}}^2}$. Further we assume $\Delta$ being $\bm{k}$-
independent for now. We can introduce the normal density of states.
\[
    D_n(\epsilon) = \frac{1}{N} \sum_{\bm{k}} \delta(\epsilon - \epsilon_{\bm{k}})  
\]
with $N$ the number of Fourier modes. This is the same as the number of lattice sites for us \cite{FossheimSudbo2004} p.93.   
\[
    \sum_{\bm{k}} g(\bm{k})= \int D_n(\epsilon)g\left(\sqrt{\epsilon^2-\Delta^2}\right) \dd \epsilon
\]
as well as a variable transformation $E = \sqrt{\epsilon^2 + \Delta^2}$ which leads to
\[
    \dd E = \frac{\epsilon}{\sqrt{\epsilon^2 + \Delta^2}} \dd \epsilon ~~ \Longleftrightarrow ~~ \dd \epsilon = \frac{E}{\epsilon} \dd E,
\]
\[
    \sum_{\bm{k}} g(\bm{k})=\int D_n(\epsilon)g(E)\frac{E}{\epsilon} \dd E.
\]
Further we have $\parDif{\epsilon}{E} = \frac{E}{\sqrt{E^2-\Delta^2}} =\frac{E}{\epsilon} $ so that
\[
    \sum_{\bm{k}} g(\bm{k}) = \int g(E)\underbrace{D_n(\epsilon)\parDif{\epsilon}{E}}_{D_s(E)} \dd E.
\]

Then we use a common method in solid state physics to simplify the expression. We assume that $D_n(\epsilon)$ is slowly varying so that $D_n(\epsilon)\approx D_n(0)$
where $\epsilon$ is measured relative to the Fermi surface. Hopefully we do not have any sharpness around the Fermi-surface \cite{FossheimSudbo2004} p.93. 
We can therefore exclude the Van Hove singularities close to the Fermi-surface.
\begin{align*}
    \frac{D_s(E)}{D_n(0)} \frac{E}{\sqrt{E^2-\Delta^2}} =& \frac{E}{\sqrt{E^2-\Delta^2}}\frac{D_n(\epsilon)}{D_n(0)}\\
    =&\frac{E}{\sqrt{E^2-\Delta^2}} \Theta (E- |\Delta|)
\end{align*}
We use the Heaviside step-function because the last term is either one or zero. This is our final expression for the density of the Meissner 
state when the gap is $\bm{k}$-independent.
For the $\bm{k}$ dependence of the gap we can use the spectral weight introduced in Eq. \ref{eq:SingleParticleSpectralWeight}.
In analogy to the propagator or Green function introduced through the Dyson equation at Eq. \ref{eq:Dyson} for the electrons, we can define one for the 
superconducting condensate that involves time-dependent fermionic operators.
\[
    G(\bm{k}, t; \sigma) = - \im \langle 0 | c_{\bm{k}\sigma}(t) c_{\bm{k}\sigma}^{\dagger}(0) | 0 \rangle
\]
involving the BCS superconducting ground state $|0\rangle$ which is an eigenstate of the Hamiltonian of Eq. \ref{eq:HamiltonianBCS2}. As before we can consider
the Fourier transform of the propagator $G(\bm{k}, \omega; \sigma)$  
\[
    G(\bm{k}, \omega; \sigma) = \frac{1}{2\pi} \int e^{\im\omega t} G(\bm{k}, \omega; \sigma)\dd t.
\] 
We recall that Cooper pairs are made of two electrons with opposite spin. We can first take a 
look at the density of state of the pair-member with $\sigma =~ \uparrow$. 
However, we here are only considering spin singlet pairing, so proving that the spectral weight $A$ is spin independent should
show that computing it for $\sigma =~ \uparrow$ is enough as Fossheim and Sudbø argued in \cite{FossheimSudbo2004} p.94.\\

So without loss of generality we set $\sigma = ~ \uparrow$ and reintroduce the rotation of the basis of the $c$ operators 
we performed in Eqs.\ref{eq:RotationBasis_c1} and \ref{eq:RotationBasis_c2}.
\begin{align*}
    G(\bm{k},t; \uparrow) = -\im \langle 0 |&~\left(\cos(\theta) \eta_{\bm{k}}^{\dagger}(t) - \sin(\theta) \gamma_{\bm{k}}^{\dagger}(t)\right)\\
    &\cdot\bigl(\cos(\theta) \eta_{\bm{k}}(0) - \sin(\theta) \gamma_{\bm{k}}(0)\bigr) | 0 \rangle
\end{align*}
The goal of these rotated operators was to diagonalize the hamiltonian. It follows
\[
    \langle 0|\eta_{\bm{k}}^{\dagger}\gamma_{\bm{k}}^{\dagger} | 0\rangle = 0 =  \langle 0|\eta_{\bm{k}}\gamma_{\bm{k}} | 0\rangle
\] 
and therefore
\begin{equation}
    \begin{aligned}
        G(\bm{k},t; \uparrow) =& -\im \cos(\theta)^2 \langle 0 | \eta_{\bm{k}}^{\dagger}(t)\eta_{\bm{k}}(0) | 0 \rangle \\
        & -\im\sin(\theta)^2 \langle 0 | \eta_{\bm{k}}^{\dagger}(t)\eta_{\bm{k}}(0) | 0 \rangle .
    \end{aligned}
\end{equation}
We can nicely split the propagator in the sum of free $\eta$- and $\gamma$-particles in the superconducting state.
Employing to coherence factors of Eqs.\ref{eq:coherenceFac_u} and \ref{eq:coherenceFac_v} in the Fourier transform of we obtain
\[
    G(\bm{k}, \omega; \uparrow) = \frac{u_{\bm{k}}^2}{\omega - E_{\bm{k}} + \im\delta_{\bm{k}}} + \frac{v_{\bm{k}}^2}{\omega + E_{\bm{k}} - \im\delta_{\bm{k}}}.
\]
From this we achieve an expression for the spectral weight:
\[
    A(\bm{k}, \omega; \uparrow) = u^2_{\bm{k}}\delta(\omega - E_{\bm{k}}) + v^2_{\bm{k}}\delta(\omega + E_{\bm{k}}).
\]
This value is spin independent and using the arguments we've just given in such a case, we find
\begin{align*}
    D_s(\omega) =& \frac{1}{N} \sum_{\bm{k}} A(\bm{k},\omega)\\
        =& \frac{1}{N} \sum_{\bm{k}} \left(u^2_{\bm{k}}\delta(\omega - E_{\bm{k}}) + v^2_{\bm{k}}\delta(\omega + E_{\bm{k}})\right).
\end{align*}
The density of state of the superconductive condensate $D_s$ is also spin independent since we are considering spin-singlet pairings. This formula should be used
when computing simulations of the superconducting state.\\ 

\subsection{Transition temperature and energy gap}
The goal of this discussion will be to derive a universal ratio between $\Delta$ and the crystal temperature. In the last section 
we already introduced some expressions for $\Delta_{\bm{k}}(T)$ and $V_{\bm{k}\bm{k}}$. Let us consider the simplest case where the potential stays constant $V_{\bm{k}\bm{k}} = V$.

The phonon modulated interaction has a cover $\omega_0 = \omega_D$, that is the Debye frequency. Inserting this back to the BCS gap equation 
\ref{eq:BCS_gap_eq}, we see that the gap loses its $\bm{k}$-dependence and results as the identity when applying the symmetries
ruling the crystal:
\begin{equation*}
    1 = V\sum_{\bm{k}'} \frac{\tanh\left(\beta E_{\bm{k}'}\right)}{2 E_{\bm{k}'}}.
\end{equation*}
This equation can be easily solved for $T=T_C$ or $T = 0$.\\

Considering $T$ approaching $T_C$ from bellow, we can assume that the gap vanishes. We replace the $\bm{k}$-sum by an integral over the normal 
density of state $D_n(\epsilon)$. Due to the shell the sum occurs in a tiny volume around the Fermi-Surface, so that $D_n(\epsilon)$ is 
evaluated close to the surface. We assume that in this neighbourhood $D_n$ varies slowly, such that avoid some van Hove singularities we 
simply approximate $D_n(\epsilon) \rightarrow D_n(0)$. We choose zero because $\epsilon$ is counted relative to the surface in our early thoughts. 
When introducing $\lambda = V D_n(0)$ we obtain:
\begin{equation*}
    \begin{aligned}
        1 =& \lambda \int_0^{\omega_D} \frac{\tanh\left(\beta\epsilon/2\right)}{\epsilon} \dd \epsilon\\
        =& \lambda \ln\left(\frac{2e^{\gamma}\beta \omega_D}{\pi}\right).
    \end{aligned}
\end{equation*}
Using $\gamma := \lim{m}{\infty} \left(\sum_{l\in\natset{m}} 1/l - \ln(m)\right)$ the Euler-Mascheroni constant. More details are provided
in \cite{FossheimSudbo2004} p.88-89. We obtain
\[
    k_B T_C \approx 1.13\cdot  \omega_D e^{-1/\lambda}
\]
For $T\rightarrow 0$ the gap equation takes a simpler form to solve:
\begin{equation*}
        1 = V \sum_{\bm{k}'} \frac{1}{2 E_{\bm{k}'}} = \lambda\int_0^{\omega_D}  \frac{1}{\sqrt{\epsilon^2 + \Delta^2}}\dd \epsilon
\end{equation*}
and leads to 
\[
    \Delta(T=0) = 2 \omega_D e^{-1/\lambda}
\]
according to the same source. We see how these expressions are closely dependent on $\lambda$. Moreover, we can interpret the essential singularity
at $\lambda\rightarrow 0$ as following: The attractive processes are singular perturbations of the non-interacting electron gas.
$m$ is very demanding to solve even for simple metals and is a function of multiple small details of the system. We aim here 
to acquire qualitative knowledge. Let us now bring the ratio
\[
    \frac{2\Delta(T=0)}{k_B T_C} = \frac{2 \pi}{e^{\gamma}} \approx 3.52
\]  
which is a universal ratio and does not depend any more on the properties of the material. Knowing the critical temperature one can know
the gap at $0\si{\kelvin}$.\\

In this long section we have derived how the Cooper pairs raise from the coupling of two electrons of opposite spin and momentum. Using a mean field approach
and a rotation of the basis we were able to reduce the many-body Hamiltonian to a single-body problem. This allowed use to derive a self-consistent solution for 
the energy gap that the coupling of the electrons left behind. Then we have seen how the gap can be expressed in a basis of the crystallographic symmetries.
Finally, we derived an expression for the density of the Meissner state.\\

Now that we have a better understanding of the superconductivity, we can move on to the next section where we will present a method to self consistently solve
the gap. This will be done by diagonalizing the Hamiltonian and read the superconductivity from the eigenvalues and -vectors. To do so we need to discretize the
system, by introducing the tight binding model.\\
\end{document}
