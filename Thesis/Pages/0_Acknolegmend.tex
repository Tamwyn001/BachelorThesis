\documentclass[../main.tex]{subfile}
\begin{document}
\thispagestyle{empty}
\vspace*{24pt}
\begin{center}
    \LARGE ABSTRACT \normalsize\vspace{24pt}\\
\end{center}
In this work, we simulated the leaking of the Cooper-pairs of a BCS superconductor \cite{FossheimSudbo2004} in an altermagnet \cite{Smejkal2022} using the tight binding and
Bogoliubov-de Gennes formalism. The simulations were performed in a lattice of $40\times30$ sites being used.
We found that the leaking decreases exponentially the more we penetrate the altermagnet, and is modulated by oscillations.
Moreover, the leaking is stronger for lower $|\mu|$, with $\mu$ the chemical potential. The presence of the Cooper pairs were the highest where 
the Fermi-surface shares the most area with the superconducting gap. The use of a straight or diagonal interface
between the superconductor and the altermagnet was found to have an influence on the leaking. Along the interface's normal, the leaking experiences less oscillations
in the case of a diagonal interface. Due to the presence of vacuum on the side of the system, we were able to see some Friedel oscillations, as well as
the presence of Andrev bound states.

We simulated as well the current induced by a phase gradient in the superconducting gap parameter $\Delta$.
The current was found proportional to the phase gradient as predicted by \cite{Orlando2003},
and is stronger the more the Cooper-pairs are present.

Finally, we simulated the $d$-wave superconductivity \cite{Mjos2019} in a superconductor described by the Hartree-Fock approximation.
The same lattice was used, and we found that the pairs are more present at lower $|\mu|$.


\newpage
\thispagestyle{empty}
\vspace*{24pt}
\begin{center}
    \LARGE ZUSAMMENFASSUNG \normalsize\vspace{24pt}\\
\end{center}
In dieser Arbeit simulierten wir das Austreten der Cooper-Paare eines BCS-Supraleiters \cite{FossheimSudbo2004} in einem Altermagneten \cite{Smejkal2022} unter Verwendung des Tight-Binding- und
Bogoliubov-de Gennes Formalismus. Die Simulationen wurden in einem Gitter mit $40 \times 30$ Stellen durchgeführt.
Wir haben festgestellt, dass das Austreten exponentiell abnimmt, je weiter wir in den Altermagneten eindringen, und dass es durch Oszillationen moduliert wird.
Darüber hinaus ist das Austreten bei niedrigerem $|\mu|$ stärker, wobei $\mu$ das chemische Potential ist. Die Anwesenheit der Cooper-Paare war dort am höchsten, wo 
die Fermi-Fläche die größte Fläche mit der supraleitenden Lücke teilt. Die Verwendung einer geraden oder diagonalen Grenzfläche
zwischen dem Supraleiter und dem Altmagneten wurde festgestellt, dass sie einen Einfluss auf das Austreten hat. Entlang der Normalen der Schnittstelle
erfährt das Leck weniger Schwingungen im Fall einer diagonalen Schnittstelle. Aufgrund des Vakuums auf der Seite des Systems konnten wir einige Friedel-Oszillationen und auch
das Vorhandensein von Andrev-gebundenen Zuständen.

Wir haben auch den Strom simuliert, der durch einen Phasengradienten im supraleitenden Lückenparameter $\Delta$ induziert wird.
Es zeigte sich, dass der Strom proportional zum Phasengradienten ist, wie von \cite{Orlando2003} vorhergesagt,
und ist umso stärker, je mehr Cooper-Paare vorhanden sind.

Schließlich haben wir die $d$-Wellen-Supraleitung \cite{Mjos2019} in einem Supraleiter simuliert, der durch die Hartree-Fock-Näherung beschrieben wird.
Es wurde dasselbe Gitter verwendet, und wir fanden heraus, dass die Paare bei niedrigerem $|\mu|$ stärker vorhanden sind.

Übersetzt mit DeepL.com (kostenlose Version)

\newpage
\thispagestyle{empty}
\vspace*{24pt}
\begin{center}
    \LARGE Acknowledgment \normalsize\vspace{24pt}\\
    \rule[3pt]{0.04\textwidth}{0.2pt} $\quad\sim\quad$\rule[3pt]{.04\textwidth}{0.2pt} 
\end{center}
\vspace*{12pt}


This bachelor thesis is the results of five months of work at the NTNU Trondheim in Norway. I would like to thank my supervisor, professor doctor Jacob W. Linder
for his guidance and support during this period, and for introducing me to an interesting branch of physics. Thanks to professor doctor Wolfgang Belzig for 
correcting this work, and Timo Oppl for his review.\\

This thesis concludes for me three years of study at the University of Constance in Germany. The way until here had many pitfalls and difficulties, and I am very grateful to 
you papa and mama, as well as to my brothers, Massa, Idriss and Vegar, for their encouragement and their kindness during our phone calls and visits. The chocolate and drawings you put in these packages
always brought me to smile! I want to express my gratitude to my grandparents for their support, their stories and delicious meals during my visits. I am also thankful
to my uncles for their great hospitality and helping me find the way in the adulthood. All the moments seeing the family felt very peaceful.

Thanks to Kai and Jakob for their friendship and interesting discussions we shared at lunch around a ``Seezeitteller''. Thanks to Kim and Luca 
in Trondheim for our trips and your smiles. Finally, I am very grateful to you Tom, for all the times we went out in the city, built projects,
watched movies, shared breakfast.
To our deep discussions and stupid jokes. You showed me how life can be great when you use of boldness towards the `no balls' situations.
Your friendship is precious to me.

\begin{center}
    \rule{0.1\textwidth}{0.2pt} $\quad\sim\quad$\rule{0.1\textwidth}{0.2pt} 
\end{center}
\end{document}