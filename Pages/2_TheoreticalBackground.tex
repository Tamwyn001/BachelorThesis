\documentclass[../main.tex]{subfile}
\begin{document}
 \normalsize
\section{Theoretical Background}
In order to describe the superconductors we are going the introduce the second quantisation formalism.
It allows us to describe the wavefunction of a system using creation and anihinaltion operators over
energystates of the system and simplify a lot the notation. The mathematical fundation of this formalism
lays in the Hilbert space, its dual space and furthermore we are going to introduce the Foch space.\\

It's also relevant for our study that we are going to work on fermions. The formalism stays the same for 
bosons but the results are fundamentaly different. One can mention the Pauli principle as an exemple which
only applies on fermion is can be derived with the help of the second quantisation.\\ 

\subsection{Bosons and fermions}
We consider without loss of generality the following hamiltonian. 
\[
    \hat{H} = \hat{H}_0 + \hat{H}_I
\]
with the single particle operator $\hat{H}_0$ and the interaction operator $\hat{H}_I$:
\[
    \hat{H}_0 = \sum_{i\in \natset{N}} \hat{h}_i(x_i)~,~~ \hat{h}_i(x_i) = -\frac{\hbar^2}{2m}\nabla_i^2 + \hat{U}(x_i)
\]
Where we introduce the notation $\natset{N} = \{n \in \mathbb{N}: n\leq N\}$. We call it single particle operator 
because the operator only applies on a particle. It may depend from the particle's postion $\bm{r}$ or spin $s$: 
$x_i := (\bm{r}, s) \in \mathcal{X}\subseteq\mathbb{R}^{3}\times\mathbb{S}$. For exemple we have for an electron 
$\mathbb{S}= {-\frac{1}{2},\frac{1}{2}}$ A single particle operator is in this case the sum of the kinetic- 
and potential energy operators.\\

Further we describe a quantum state that a particle can occupy with a wavefunction $\phi_{\nu}(x)$, 
which own a certain energy $\epsilon_n \in\mathbb{R}$. This energy depends on 
the wavevector and the spin of the particle: $\nu = (\bm{k}, \sigma)$. The fundamental equation of
quantum mechanics relates the wave function with the hamiltonian using the energy of the state:
\[
    \hat{h} \phi_{\nu}(x) = \epsilon_\nu \phi_{\nu}(x)
\]
The wave function lay in the Hilbert space [more details?]. Therefor $\phi_{\nu}(x)$ are eigenfunction or -states of
the Hamiltonian with eigenvalues $\epsilon_{\nu}$. Further the wavefunction should build an orthonormal basis:
\[
    \int_\mathcal{X} \phi_{\nu'}(x) \phi_{\nu}(x) \dd x = \delta_{\nu'\nu}.
\]
$\nu$ and $\nu'$ are two different states. We introduced here the korenker delta $\delta_{\nu'\nu}$ which is one when the two indicies
are equal and zero otherwise. Because the spin $s$ is not continuous one can understand the integral in the following way:
\[
    \int_\mathcal{X} \dd x = \sum_{s\in \mathbb{S}} \int_{\mathbb{R}^3} \dd^3 r
\]  
where $ \int_{\mathbb{R}^3}\dd r^3 = \int_{\mathbb{R}}\int_{\mathbb{R}}\int_{\mathbb{R}} \dd r_1 \dd r_2 \dd r_3$.
We integrate over all possible states.\\

Now that we can describe one particle we want to describe a system containing many instance of that particle.
The wavefunction sums up all possible combination of wavefunction in the system and should stay normalsized. 
A combination is discribed as the product of the wavefunction of the particle in a certain state. These particle
can be swaped and therfore we need to consider all combinations.
 We restrict ourselves to fermions and bosons. We admit having $N \in \mathbb{N}$ paricles in the system.\\

\paragraph{Bosons}
The many-particle wavefunction of the bosons is a symetric (exponent $S$) under swap of two particles.
\[
    \Phi^{(S)}(x_1,..,x_N) = \left(N!\prod_{N}(n_{\nu})!\right)^{-\frac{1}{2}} \sum_{P\in S_n} P \phi_{\nu_1}(x_1)\cdot ..\cdot \phi_{\nu_N}(x_N)
\]
where $n_{\nu}$ represents the number of particle in the state $\nu$. Therefor we usaly call it the occupation number of the state $\nu$.
For fermion this integer has no constrain in general.
The permutation set $S_n$ contains all the possbile combinations of $x_i$ in the state $\nu_j$ for $i,j\in\natset{N}$.

\paragraph{Fermions}
Fermions are a bit different, their many-particle fermion wavefunction is antisymmetric under swap of two particles. We denote it as
\[
    \Phi^{(A)}(x_1,..,x_N) = \left(N!\right)^{-\frac{1}{2}} \sum_{P\in S_n} \signum{P}\cdot P \phi_{\nu_1}(x_1)\cdot ..\cdot \phi_{\nu_N}(x_N).
\]
$\text{sgn}$ represents the signum function. Applied on a permutation $P$ it is one if $P$ is even and minus one if $P$ is even.\\
We already know that the Pauli principle implies that it can be up to one particle in each energy state. We therfore have $n_\nu \in \{0,1\}$. The normalsization
factor is the same but the product over the ones vanishes.\\

At this point one might have recognised the formula of the determinant
\[
    \Phi^{(A)}(x_1,..,x_N) = \left(N!\right)^{-\frac{1}{2}} \text{det}\begin{pmatrix}
        \varphi_{\nu_1}(x_1)& \cdot\cdot &\varphi_{\nu_1}(x_N)\\
        \vdots&  &\vdots\\
        \varphi_{\nu_N}(x_1)& \cdot\cdot &\varphi_{\nu_N}(x_N)\\

    \end{pmatrix},
\]
which vanishes if two rows or columms are identic. We usaly describe this expression as the Slatter determinant. This means that the probability of finding two fermions in the same state is zero.
This is the Pauli principle. Only one or no particle may occupy each state.\\

Further we entcounter a major problem. The many-particle wave function of fermions is defined up to a sign. For instance if we consider
two particles ``having'' $x_1$ and $x_2$, we have two possible state $\nu_1$ and $\nu_2$. To possible soltutuion are
\begin{align*}
    &\Phi^{(A_1)} = \frac{1}{\sqrt{2}} \bigl(\varphi_{\nu_1} (x_1)\varphi_{\nu_2} (x_2) - \varphi_{\nu_1} (x_2)\varphi_{\nu_2} (x_1) \bigr)\\
    \text{or~} &\Phi^{(A_2)} = \frac{1}{\sqrt{2}} \bigl(\varphi_{\nu_1} (x_2)\varphi_{\nu_2}(x_1) - \varphi_{\nu_1} (x_1)\varphi_{\nu_2} (x_2)\bigr)\\
    &~~~~~~~=-\Phi^{(A_1)}.
\end{align*}
This sign difference may lead to computation errors. We aim to give a labeling to our states when we count them and keep it when it 
comes to build the Slatter determinant.\\

These bosonic and fermionic wavefunctions are eigenstate of the Hamiltonian $\hat{H}_0$ and the corresponding eigenvalue $E_{\nu}$
is given by summing the energy of each state times its occupation number: $E_{\nu} = \sum_{\nu} \epsilon_{\nu} n_{\nu}$.
For this reason it's important that they build an orthonormal basis:
\[
    \int_{\mathcal{X}^N} \Phi_a^{\ast}(x_1,..,x_N) \Phi_b(x_1,..,x_N) \dd^N x = \delta_{ab}.
\]
Therfore we can expend any many-particle wavefunction $\Psi$ as the linear combination of these:
\[
    \Psi = \sum_a f_a \Phi_a(x_1,..,x_N)
\]
where $f_a$ is a coefficent and $a$ a labeling.\\

What we just discused is the so called first quantisation- or wave function formalism. Now we intend to introduce a better way of 
describing our system.

\subsection{The second quantisation}
For a better description of the many-particle system we introduce a simpler notation. The second quantisation lays on three important objects. States described as ``ket''. We put any relevant information between the ket
e.g. $|\bm{k}, \sigma,..\rangle$. Then we need operators that act on these states to allow interactions in the system. 
We need an operator that creates a states and another that anihilates a state.\\

\paragraph{States} In this section we describe a state as the number of particle that occupies each single-particle state. Therfore
we order the state $1<2<..<N$. We then can describe the wave function as follow $|n_{1},..,n_{N}\rangle$.\\

Further the state where no particle are present is called the vaccum state and we denote it as $|0_{\nu_1},..,0_{\nu_N}\rangle = |0\rangle$.

\subsubsection{Second quantisation for fermions}
\paragraph{Creation operator $c_\nu^{\dagger}$}
The creation operatortor adds a particle in the state that is concernd and rephase the state:
\[
    c_{\nu}^{\dagger} |n_{1},..,n_{\nu},..\rangle = (-1)^{\sum_{\mu<\nu}n_{\mu}} (1-n_{\nu})|n_{1},..,n_{\nu}+1,..\rangle
\]
We notice the $ (1-n_{\nu})$ term which avoid to create a particle at the state, if one already exist. This is the expression of
the Pauli-principle. 
and we can then construct a state by appling this operator after another on the vaccum state. To avoid the minus one to add a negative sign, we start from the 
end and add the particle backwards in the order of the state:
\[
    |n_{1},..,n_{N}\rangle = (c_{1}^{\dagger})^{n_{1}}\cdot ..\cdot (c_{N}^{\dagger})^{n_{N}} |0\rangle
\]  

\paragraph{Anihilation operator $c_\nu$}
Likewise the anihinaltion operator destroys a particle in the corresponding state. The operator reads
\[
    c_{\nu}^{\dagger} |n_{1},..,n_{\nu},..\rangle = (-1)^{\sum_{\mu<\nu}n_{\mu}} (n_{\nu})|n_{1},..,n_{\nu}-1,..\rangle.
\]
We can easly recognise that due to the $n_{\nu}$-term, destroying a particle that doesn't exist gives zero, 
so it's only possible to destroy particle that exist. Further we intend to introduce some few compution 
rules that are going to help us later. \\

The anticommutator of two operator reads $[A,B]_{+}$ or $\{A,B\} := AB + BA$ and is an opertor as well.
We're going to stick with $[A,B]_{+}$ since it's more consitent with the commutatotor notation $[A,B]_{-}$ (or simply $[A,B]$).\\

The follwoing results are obtain by separting the $\nu = \mu$ from the $\nu \neq \mu$. We must also say that the dagger $\dagger$ 
should be understand as the complex transpose of the operator and $(AB)^{\dagger} = B^{\dagger}A^{\dagger}$.
\begin{align*}
    [c_{\nu},c_{\mu}]_+ =&~ 0\\
    [c^{\dagger}_{\nu},c^{\dagger}_{\mu}]_+ =&~0\\
    [c^{\dagger}_{\nu},c_{\mu}]_+ =&~ \delta_{\mu,\nu}\\
\end{align*} 

We can then combine the creation and anihinaltion operator to count the number of particles in a state:
\[
    c_{\nu}^{\dagger} c_{\nu} |n_{1},..,n_{\nu},..\rangle = n_{\nu}|n_{1},..,n_{\nu},..\rangle.
\]  
From this we can define the number operator $\hat{n}_{\nu}:= c_{\nu}^{\dagger} c_{\nu}$ which we can combine in the total number operator
\[
    \hat{N} = \sum_{\nu} \hat{n}_{\nu}~,~~\text{where logicaly}~ N = \sum_{\nu} n_{\nu}
\]
if we apply the operator on a state.
\subsubsection{Second quantisation for bosons}
\subsection{Basis transformation}
\subsection{Interactive electron gas}

The main transformation beween the second quantisation to the first are the followings:
\begin{align*}
    \varphi_{\alpha}(x) =& \langle x|\alpha\rangle\\
    \langle \alpha |V|\beta\rangle =& \int  \varphi_{\alpha}^{\ast}(x) V(x) \varphi_{\beta}(x) \dd x\\
    \langle\alpha\beta|V|\gamma\delta\rangle =& \int \int \varphi_{\alpha}^{\ast}(x) \varphi_{\beta}^{\ast}(x') V(x,x') \varphi_{\gamma}(x) \varphi_{\delta}(x') \dd x \dd x'
\end{align*}
\section{Superconductivity}
Superconductivity can be illustrated as a phase transition of a meterial under a crital temperature. In the superconductive state the material 
become a perfect diamaget and its resistivity vanishes. We then observe some shielding currents that araise on it's surface
and we can let flew a current for a very long time without loosing energy. The superconductive state is also described as Meissner state.\\ 

Suppose that we heat the material to the critical temperature $T_c$, some fluctuation effects araise and break the superconductive state.
The shielding effects reacts different on the material. We usely distinguish type I and type II superconductors. The type I superconductor
loose abruplty their magnetisation over $T_c$. Type II have a mixed state where the magnetisation slowly decreases until we can't measure it anymore.\\

The break of the superconductive state can be described as letting more and more filed 
flew inside of the material. Asuming that some particle are responsible for the superconductivity, the field achieve to penetrate where wo observe 
a lower density of these particles. The penetration is described as some magnetic field vortecies reaching a certain depth in the material.\\

The Meissner state is a thermodynamical state. We can show that the free energy of the superconductive state is higher than the normal state. 
This results in a lower entropy compered to the normal state.

\subsection{Theoretical framework and BCS theory}
The Hamiltonian of the system is discribed by the solid state physics. The consider the energy of the electrons and the ions in a lattice.
\[
    H = H_{e^-e^-} + H_{e^-\text{ion}} + H_{\text{ion ion}}
\]
Each term consist of a kinetik and potential energy term. For a more mathematical approch we consider a system of $N$ electrons and $L$ ions.
\begin{align*}
    H_{e^-e^-} &= \sum_{i \in \natset{N}} \frac{p_i^2}{2m} + \sum_{i,j\in\natset{N}} V_{\text{Coulomb}}^{e^-e^-}(\bm{r}_i-\bm{r}_j) \\
    H_{\text{ion ion}} &=  \sum_{i \in \natset{M}} \frac{p_i^2}{2M} + \sum_{i,j\in\natset{L}} V_{\text{Coulomb}}^{\text{ion-ion}}(\bm{R}_i-\bm{R}_j) \\
    H_{e^-\text{ion}} &= \sum_{i\in \natset{N},j\in \natset{L}} V_{\text{Coulomb}}^{e^-\text{ion}}(\bm{r}_i-\bm{R}_j)
\end{align*}
We have $m$ and $M$ as the mass of the electron and the ion. $\bm{r}$ and $\bm{R}$ are the position of the electron and the ion. The ion-ion potential
freezes the ions into the lattice. 
We first going to introduce some concepts by describing a non-interacting electron and then improve it to include the interactions.
\subsubsection{The non-interacting electron gas}
In this case of study the Hamiltonian only include a kinetic term
\[
    H=\sum_{\bm{k},\sigma} \epsilon_{\bm{k}} c_{\bm{k},\sigma}^{\dagger}c_{\bm{k},\sigma}.
\] 
We asume that it exist a the ground state $|0\rangle$, where the system is filled up with a certain amount of electron until the Fermi-energy $\epsilon_F$ is reached. Associated with this
energy we find a wave vector $\bm{k}_F$, the Fermi-momentum. The set of enery up to $\epsilon_F$ is called the Fermi-sea, as an analogy to the level zero of the topographic maps.
\[
    \hat{n}_{\bm{k},\sigma} |0\rangle = \Theta(\epsilon_F - \epsilon_{\bm{k}})|0\rangle.
\]
We introduced here a very useful tool called the Heavyside-step function wich is defined as: 
\[
    \Theta(x) = \begin{cases}
        1, ~x > 0\\
        0, ~x < 0
    \end{cases} ~ , ~~ \Theta(-x) = 1 - \Theta(x).
\] 
This means that if we count the particle that have an energy heigher than the Fermi-energy ($\bm{k}>\bm{k}_F$) than we get zero.\\

We now want to study how the electron can scatter in different states. The function that we're using is called the propagator and
gives the probailty to find the particle at $|\bm{k}',\sigma\rangle$ at $t'$ know it at  $|\bm{k},\sigma\rangle$, $t$. 
An important fact is that without interaction, the particle shouldn't scatter in another state due to energy conservation. Therfore
\[
    G_0(\bm{k}, \bm{k}', t'-t) = G_0(\bm{k}, t'-t) \delta_{\bm{k}, \bm{k}'}.
\]
which is zero if the wave-vectors between the two timepoint differs. 
We observe that only the past time $t'-t$ is relevant. This is due to the time independent property of the Hamiltonian.
We are going to use the representation in the frequency space, using a Fourier-transformation.
\[
    G_0(\bm{k},\omega) = \int_{\mathbb{R}} e^{\im\omega t} G_0(\bm{k},t)\dd t = \frac{1}{\omega - \epsilon_{\bm{k}} + \im \delta_{\bm{k}}} \label{eq:single_propa}
\]
where $\delta_{\bm{k}} = \delta \cdot\signum{\epsilon_{\bm{k}} - \epsilon_{F}}$ involving a very small non zero number $\delta$. We observe that
this analytical function has a pole given by 
\begin{align*}
    &\omega - \epsilon_{\bm{k}} + \im \delta_{\bm{k}} = 0\\
    \Longleftrightarrow & \omega = \epsilon_{\bm{k}} - \im \delta_{\bm{k}}
\end{align*}
where we denote $\im$ as the imaginary unit to avoid confusion with the index $i$.
The frequency $\omega$ gives the so called spectrum of the exitation from the unique-particle system. The imaginary part serves as a damping term and is 
inversly proportional to the lifetime of the particle. $\delta$ is a small number due to the infinitely long lifetime. This is a direct result of the abscence
of scattering.\\

Further the propagator yields important informations on the system when considering the integration over its different arguments. First we take the imaginary 
part of the propagator called the single particle spectral weight.
\begin{align*}
    A(\bm{k}, \omega) = -\frac{1}{\pi} \mathcal{I}m \left[G_0(\bm{k}, \omega)\right] =~& \frac{1}{\pi} \frac{\delta_{\bm{k}}}{(\omega- \epsilon_{\bm{k}})^2 + \delta_{\bm{k}}}\\
    =~&\delta(\omega- \epsilon_{\bm{k}})
\end{align*}
which informs us about the occupation of a state $|\bm{k}\rangle$ with energy $\omega$. We can find a form for the momentum distribution $n(\bm{k})$
\[
    n(\bm{k}) = \int A(\bm{k},\omega) \dd \omega
\]
and for the density of state
\[
    D(\omega) = \int A(\bm{k},\omega) \dd ^3 k~,~~ \text{or for discontinuous state}~ \sum_{\bm{k}} A(\bm{k}, \omega).
\]
\subsubsection{Fermi-Liquid}
In an earlyer section we saw how
\[
    H = \sum_{\bm{k},\sigma} \epsilon_{\bm{k}} c_{\bm{k},\sigma}^{\dagger}c_{\bm{k},\sigma} + \sum_{\bm{k},\sigma,\bm{k}',\sigma'}
        V_{\bm{k}\bm{k}', \bm{q}} c_{\bm{k}-\bm{q},\sigma}^{\dagger}c_{\bm{k}+\bm{q},\sigma'}^{\dagger}c_{\bm{k},\sigma}c_{\bm{k}',\sigma'}
\]
represent the pairwise interaction of multiple electrons and their respective energy. To extend the model we now want to introduce two new [measures],
the propagator $G$ and the one-particle irreductible self-energy $\Sigma$.\\

The propagator [bildet ab] in the complex space and sgives the probability amplitude of finding a the particle in the state $|\bm{k},\sigma\rangle$ at a time $t$. On the other hand 
$\Sigma = \Sigma_R + \im \Sigma_I$ contains 
the lifetime of the particle in this state and shift of energy of the particle due to the interaction with the surroundings. The frameworks defines the 
non-interacting energy of the particle as $\epsilon_{\bm{k}}$. Whene put in an interacting system the spectrum shifts and becomes 
$\tilde{\epsilon}_{\bm{k}} =  \epsilon_{\bm{k}} + \Sigma_R$. Due to the interactions, the particle then has a much smaller lifetime. $\Sigma_I$ is antiproportional to 
the particle's lifetime $\tau_{\bm{k}}$. We therfore expect $\Sigma_I$ to be realy small in the non interacting case. 
These two \textit{Größe} are linked trough the Dyson equation, which reads
\[
    \bigl(G(\bm{k}, \omega)\bigr)^{-1} =  \bigl(G_0(\bm{k}, \omega)\bigr)^{-1} - \Sigma(\bm{k},\omega).
\]
\textit{sigma R}
One can use a Fourier-transformation to switch from the time representation to the frequency representation $\omega$. Reordering the equation and using the result
from \ref{eq:single_propa} we obtain
\[
    G(\bm{k},\omega) = \frac{1}{\omega - \epsilon_{\bm{k}} - \Sigma}.
\]
\end{document}

