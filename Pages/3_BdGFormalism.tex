\documentclass[../main.tex]{subfile}
\begin{document}

\section{Bogoliubov-de Gennes Formalism}
The Bogoliubov-de Gennes transformation allows us to express the hamiltonian in a diagonal way and finding some quantities 
by looking at the eigenvectors of the hamiltonian. The resulting matrix is expressed in a huge space and is very sparse.\\

To give a taste of it, it will allow us to rewrite our hamiltonian as following
\begin{align}
    H = E_0 - \frac{1}{2} \check{c}^\dagger \check{H} \check{c}, \label{eq:BdG_intro_H}
\end{align}
involving $\check{c} = (\hat{c}_1,..,\hat{c}_N)$, where each $\hat{c}_i$ is a
 vector containing the creation and annihilation operators of a lattice site $i$:
$\hat{c}_i = (c_{i,\uparrow} ,c_{i\downarrow}, c_{i,\uparrow}^{\dagger} ,c_{i,\downarrow}^{\dagger})$.\\

As we see we just describe each site with the four possbile $c$-operators. This means for each lattice site,
we have a $4\times4$-submatrix that reflects the possible combinations of creation and anihinaltion operators of both spins. 
For the readability we are going to drop the comma between the site and spin indices.\\
For exemple if one has (without loss of generality) a chemical potential at the site $i$, 
 then the hamiltonian is discribed in the following way:
\[
    H_{\text{chem},i} = \sum_{\sigma} \mu_i c_{i,\sigma}^{\dagger} c_{i,\sigma}
\]
If we want to discribe it in therm of $\hat{c}_i$ we have:
\[
    H_{\text{chem},i} = \hat{c}_i^{\dagger}\cdot \mu_i \mathcal{I}_4 \cdot\hat{c}_i = \begin{pmatrix}
        c_{i,\uparrow}^{\dagger} \\c_{i\downarrow}^{\dagger}\\ c_{i\uparrow} \\c_{i\downarrow}\\
    \end{pmatrix}\cdot \mu_i
     \begin{pmatrix}
        1 & 0 & 0 & 0\\
        0 & 1 & 0 & 0\\
        0 & 0 & 1 & 0\\
        0 & 0 & 0 & 1
    \end{pmatrix}\begin{pmatrix}c_{i,\uparrow} ,c_{i\downarrow}, c_{i\uparrow}^{\dagger} ,c_{i\downarrow}^{\dagger}\end{pmatrix}
\]
Depending on the interaction we wish to describe, we can figure out what combination of operatos we want and design the $4\times4$ matrix accordingly.
To achieve a full description of the system we can consider the inderaction between to site $i,j$ as a $4\times4$ matrix involving the $\hat{c}_i^{\dagger}$ and $c_j$ operators.
Then we can build a huge matrix $\check{H}$ based on $4\times4$ matrices at $\check{H}_{i,j}$ and the vector we multiply it to is juste the $\hat{c}_i^{\dagger}$ and $c_j$ operators 
stack above one and other forming the above-introduced $\check{c}$ vector. As a result, one gets the first formula introduced in this section \ref{eq:BdG_intro_H}.\\

[..]\\

Asuming we now have the Hamiltonian 
\begin{align*}
    H = E_0 - \frac{1}{2}\sum_{i,j} \hat{c}_i^{\dagger} \hat{H}_{ij} \hat{c}_j  \ref{eq:BdG_sys_H}
\end{align*}
where the on site matrix reads
\[
    \hat{H}_{ij} = \begin{pmatrix}
        \mu_i \mathcal{I}_2 \delta_{ij} + t_{ij} & -\im\sigma_2\Delta \delta_{ij}\\
        \im\sigma_2\Delta^{\ast} \delta_{ij} & -\mu_i \mathcal{I}_2 \delta_{ij} - t_{ij}^{\ast}
    \end{pmatrix} = \begin{pmatrix}
        H_{ij} & \Delta_{ij}\\
        \Delta_{ij}^{\dagger} & -H_{ij}^{\ast}
    \end{pmatrix}
\]
we can further compress our $\hat{c}_i$ operator by introducing 
\[
    \check{c} = (\hat{c}_1,..,\hat{c}_N)
\] 
along with the system Hamiltonian-matrix $\check{H}_{ij} := \hat{H}_{ij}$ wich allows us to rewrite the Hamiltonian $\ref{eq:BdG_sys_H}$ as:
\begin{align}
    H = E_0 - \frac{1}{2} \check{c}^\dagger \check{H} \check{c}.
\end{align}
\end{document}

